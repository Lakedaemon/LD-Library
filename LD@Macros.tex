%%%%%%%%%%%%%%%%%%%%%%%%%%%%%%%%%%%%%%%%%%%%%%%%%%%%%
%                                                                                                                                            %
%                                                         Titres, fontes et macros                                             %
%                                                                                                                                            %
%%%%%%%%%%%%%%%%%%%%%%%%%%%%%%%%%%%%%%%%%%%%%%%%%%%%%

\def\tit{%
\textfont0=\seventeenrm\scriptfont0=\tenrm\def\rm{\fam0\seventeenrm}%
\textfont1=\seventeenib\scriptfont1=\twelveib%
\textfont2=\seventeensy\scriptfont2=\twelvesy\scriptscriptfont2=\ninesy
\textfont3=\seventeenex
\textfont\itfam=\seventeenti
\def\it{\fam\itfam\seventeenti}%
\textfont\bbfam=\seventeenbb \scriptfont\bbfam=\twelvebb
\def\bb{\fam\bbfam\seventeenbb}%
\textfont\msafam=\seventeenmsa\scriptfont\msafam=\twelvemsa
\textfont\scalnfam=\seventeenscaln
\def\pcap{\fam\pcapfam\seventeenpcap}
\normalbaselineskip=25pt\normalbaselines\rm}


%% Pour couper automatiquement un titre trop long

\def\raggedcenter{\leftskip=20pt plus 10em  % reglages initial 4em
     \rightskip=\leftskip
      \parfillskip=0pt
       \spaceskip=.3333em \xspaceskip=.5em
        \pretolerance=9999 \tolerance=9999
         \hyphenpenalty=9999 \exhyphenpenalty=9999 }

\def\titrecentre#1{{\parindent0mm\raggedcenter\tit#1\par}}

%%%%%%%%%%%%%%%%%%%%%%%%%%%%%%%%%%%%%%%%%%%%%%%%%%%%%
%                                                                                                                                            %
%                                                  Paramètres d'une page                                                    %
%                                                                                                                                            %
%%%%%%%%%%%%%%%%%%%%%%%%%%%%%%%%%%%%%%%%%%%%%%%%%%%%%


%%% différents formats de pages
\def\formatpage{
\hsize150truemm
\vsize 240truemm
\hoffset=5truemm
\voffset=-3truemm
\pretolerance=500
\tolerance=1000
\brokenpenalty=5000
\parindent3mm
}
%\formatpage

\def\olspept{
%\magnification 1200
\hsize180truemm\vsize 270truemm\hoffset=-10truemm\voffset=-13truemm
\pretolerance=500\tolerance=1000\brokenpenalty=5000
\parindent3mm
}


\def\dimstand{\hsize 150truemm\vsize 240truemm\hoffset=5truemm\voffset=0truemm}

\def\dimart{\hsize126truemm\vsize186truemm\hoffset16truemm\voffset=24truemm}

%%



\interfootnoteskip=0pt
\let\note=\numberedfootnote
\everyfootnote={\eightpoint\leftskip=5truemm\rightskip5truemm}


%%


\newcount\chstart
\chstart=\pageno
\headline={\ifnum\pageno=\chstart {\hfill} \else {\hss \tenrm --\ \folio\ --\hss}\fi}
\footline={\hfill}
\normalbaselines
\frenchspacing

%%
\newif\ifMidFolio
\MidFoliotrue            % numéro de page au mileu du haut de page par d\'efaut

\newif\ifpagetitre \pagetitretrue
\newtoks\hautpagetitre \hautpagetitre={\hfill}
\newtoks\baspagetitre \baspagetitre={\hfill}
\newtoks\auteurcourant \auteurcourant={\hfill}
\newtoks\titrecourant \titrecourant={\hfill}
\newtoks\hautpagegauche
\newtoks\hautpagedroite
\newtoks\hautpagemilieu
\hautpagemilieu={\tenrm\hfil -- \folio\ -- \hfil}
\hautpagegauche={\ifMidFolio\the\hautpagemilieu\else\tenrm\folio\hfil\the\auteurcourant\hfil\fi}
\hautpagedroite={\ifMidFolio\the\hautpagemilieu\else\hfil\the\titrecourant\hfil\tenrm\folio\fi}
\newtoks\baspagegauche \baspagegauche={\hfil}
\newtoks\baspagedroite \baspagedroite={\hfil}
\headline={\ifpagetitre\the\hautpagetitre
\else\ifodd\pageno\the\hautpagedroite\else\the\hautpagegauche\fi\fi }
\footline={\ifpagetitre\the\baspagetitre
\else\ifodd\pageno\the\baspagedroite
\else\the\baspagegauche\fi\fi \global\pagetitrefalse}

\def\nopagenumber{\headline={\hfill}\footline={\hfill}}
%\def\numpgsuiv{ \chstart=\pageno\pageno=\chstart}

\def\pageblanche{\vfill\eject\pagetitretrue
\null\vfill\eject
\pagetitretrue
}


\def\hautspages#1#2{\auteurcourant={\ninepcap#1}\titrecourant={\nineit#2}\MidFoliofalse}

\ifnum\chstart=\pageno \pagetitretrue\fi
%\ifnum\pageno=\chstart \pagetitretrue \fi





%%%%%%%%%%%%%%%%%%%%%%%%%%%%%%%%%%%%%%%%%%%%%%%%%%%%%
%                                                                                                                                            %
%                                                  Date et heure (française)                                                  %
%                                                                                                                                            %
%%%%%%%%%%%%%%%%%%%%%%%%%%%%%%%%%%%%%%%%%%%%%%%%%%%%%


\countdef\temps=170
\temps=\time
\countdef\nminutes=171{\nminutes = \time}%
\countdef\nheure=172
\def\heure{\begingroup
   \temps = \time \divide\temps by 60
   \nheure = \temps
   \nminutes = \time
   \multiply\temps by 60
   \advance\nminutes by -\temps
   \ifnum\nminutes<10 \toks1 = {0}%
   \else\toks1 = {}%
   \fi
   \number\nheure h\the\toks1 \number\nminutes
\endgroup}%

\def\dateheure{\vglue-10mm\rightline{(\the\day/\the\month/\the\year,\ \heure)}}
\def\date{\the\day/\the\month/\the\year}


%%%%%%%%%%%%%%%%%%%%%%%%%%%%%%%%%%%%%%%%%%%%%%%%%%%%%
%                                                                                                                                            %
%                                       Caractères spéciaux                                                                    %
%                                                                                                                                            %
%%%%%%%%%%%%%%%%%%%%%%%%%%%%%%%%%%%%%%%%%%%%%%%%%%%%%

\def\e{{\rm e}}                                                % constante e
\def\qed{\Math{\rlap{$\sqcap$}\sqcup}}        % rectangle en fin de démonstration
\def\1{1\!\!1}                                                   % fonction caractéristique
\def\ssm{\smallsetminus}                             % opérateur ensembliste "privé de"

\def\swap#1#2{\let\temp#1\let#1#2\let#2\temp}
\swap\epsilon\varepsilon
\swap\theta\vartheta
\swap\rho\varrho
%\swap\sigma\varsigma
%\swap\phi\varphi



%%%%%%%%%%%%%%%%%%%%%%%%%%%%%%%%%%%%%%%%%%%%%%%%%%%%%
%                                                                                                                                            %
%                                                 Bibliographie                                                                     %
%                                                                                                                                            %
%%%%%%%%%%%%%%%%%%%%%%%%%%%%%%%%%%%%%%%%%%%%%%%%%%%%%

%
%%   Paramètres
%

\newif\ifNumRef              % Determine si les references bibliographiques sont numérotées automatiquement    
\NumReftrue                   %                               (option par défaut) 

%
%%   Variables
%

\newcount\NumRef\NumRef=0

%
%%
%%% Nouvelles instructions
%
%

\def\CitRef#1{[\refn{Ref§#1}]}              % \citeref{label}   ->  renvoie  "[numero associe au label]"

\def\DefRef#1 ; #2 ; #3 ; #4 ; #5\par{\ifNumRef\global\advance\NumRef by 1\definexref{Ref§#1}{{\the
                       \NumRef}}{}\else\definexref{Ref§#1}{#1}{}\fi%          \DefRef{label}   ->  definit un [numero] associé au label #1 
                        \item{\CitRef{#1}} #2, {\it #3 \bf #4} #5}%                                         et affiche "[numero] #2, {\it #3 \bf #4} #5"
                

%%%%%%%%%%%%%%%%%%%%%%%%%%%%%%%%%%%%%%%%%%%%%%%%%%%%%
%                                                                                                                                            %
%        Numérotation automatique des paragraphes, des équations et Sommaire              %
%                                                                                                                                            %
%%%%%%%%%%%%%%%%%%%%%%%%%%%%%%%%%%%%%%%%%%%%%%%%%%%%%


%
%%   Paramètres
%

\newif\ifSecLabelEq        % Determine si le numero de section prefixe le label des equations    
\SecLabelEqtrue             %                               (option par défaut) 

%
%% Variables
%

\newcount\titreun\titreun=0
\newcount\titredeux\titredeux=0
\newcount\titretrois\titretrois=0

%
%% Tests
%

%\newif\ifHealthy

%
%% Fontes
%

\def\fontetitreun{\twelvepts\gpdouze\twelvebf}
\def\fontetitredeux{\elevenpts\textfont1=\eleveni\scriptfont1=\ninei\elevenit}
\def\fontetitredeuxb{\elevenpts\eleventi\gponze\textfont1=\elevenib\scriptfont1=\nineib}
\def\fontetitretrois{\textfont0=\elevenrm\scriptfont0=\eightrm\textfont1=\eleveni\scriptfont1=\eighti\scriptscriptfont1=\sixi\elevenit}

%
%% Fonctions internes
%

\def\eqconstruct#1{\ifSecLabelEq\the\titreun.#1\else#1\fi}%Definit la numerotation des equations

\def\init#1{\global#1=0}

\def\incr#1{\global\advance#1 by 1 {\the #1}}

\long\def\Indentation#1#2{\setbox10=\hbox{\fontetitreun#1}
                          \ifdim\wd10 < 4mm
                       \setbox10=\hbox to 4mm{\box10\hfill}
                     \else\ifdim\wd10 < 6mm
                       \setbox10=\hbox to 6mm{\box10\hfill}
                          \else\ifdim\wd10 < 8mm
                       \setbox10=\hbox to 8mm{\box10\hfill}
                     \else\ifdim\wd10 < 12mm
                       \setbox10=\hbox to 12mm{\box10\hfill}
                     \fi\fi\fi\fi
                     \dimen10=\hsize
                     \advance \dimen10 by -\wd10
                     \noindent \box10 %\ignorespaces
                     \hbox{\vtop{\hsize=\dimen10\raggedright\noindent\fontetitreun#2}}}





%%%%%%%%%%%%%%%%%%%%%%%%%%%%%%%%%%%%%%%%%%%%%%%%%%%%%
%                                                                                                                                            %
%                                                 Notes en marge gauche                                                     %
%                                                                                                                                            %
%%%%%%%%%%%%%%%%%%%%%%%%%%%%%%%%%%%%%%%%%%%%%%%%%%%%%


\def\leftnote#1{\vadjust{\setbox1=\vtop{\hsize 20mm\parindent=0pt\eightpoint
\baselineskip=9pt\rightskip=4mm plus 4mm\vskip-4.7mm#1}\hbox{\kern-2cm\smash{\box1}}}}


%%%%%%%%%%%%%%%%%%%%%%%%%%%%%%%%%%%%%%%%%%%%%%%%%%%%%
%                                                                                                                                            %
%                                                 Opérateurs                                                                         %
%                                                                                                                                            %
%%%%%%%%%%%%%%%%%%%%%%%%%%%%%%%%%%%%%%%%%%%%%%%%%%%%%

\def\log{\mathop{\rm log}\nolimits}

\def\ch{\mathop{\rm ch}\nolimits}
\def\sh{\mathop{\rm sh}\nolimits}
\def\th{\mathop{\rm th}\nolimits}
\def\tg{\mathop{\rm tg}\nolimits}
\def\cotan{\mathop{\rm cotan}\nolimits}
\def\arccos{\mathop{\rm Arccos}\nolimits}
\def\arcsin{\mathop{\rm Arcsin}\nolimits}
\def\arctan{\mathop{\rm Arctan}\nolimits}
\def\argth{\mathop{\rm Argth}\nolimits}
\def\argsh{\mathop{\rm Argsh}\nolimits}
\def\argch{\mathop{\rm Argch}\nolimits}


\def\Sim{\mathop{\sim}}
\def\card{\mathop{\rm card}\nolimits}
\def\dim{\mathop{\rm dim}\nolimits}
\def\det{\mathop{\hbox{\rm d\'et}}\nolimits}
\def\mod{\mathop{\rm mod}\nolimits}
\def\sgn{\mathop{\rm sgn}\nolimits}



\def\quote{\leavevmode\raise.24ex\hbox{$\scriptscriptstyle\langle\!\langle\>$}}    % guillemets ouvrants
\def\etouq{\leavevmode\raise.24ex\hbox{$\scriptscriptstyle\>\rangle\!\rangle$}}    % guillemets fermants



%% pour aller a la ligne dans un \proclaim:
\def\PAR{\par}


%pour eviter les rectangles noirs:
%\overfullrule=0pt
% pour que les @ soient des lettres
\def\signeat{\catcode`@=11}



%%%%%%%%%%%%%%%%%%%%%%%%%%%%%%%%%%%%%%%%%%%%%%%%%%%%%
%                                                                                                                                            %
%                                                 Notes en bas de page                                                       %
%                                                                                                                                            %
%%%%%%%%%%%%%%%%%%%%%%%%%%%%%%%%%%%%%%%%%%%%%%%%%%%%%


% Pour que les accents se placent correctement en mode math en corps 8 et 6
\skewchar\eighti='177 \skewchar\sixi='177
\skewchar\eightsy='60 \skewchar\sixsy='60

%
% [Format : \note{texte de la note} avec numerotation automatique]
%
%
\let\footnotea=\footnote
%\catcode`@=11
\message{output routines,}
\newinsert\footins
\def\footnote#1{\let\@sf\empty % parameter #2 (the text) is read later
  \ifhmode\edef\@sf{\spacefactor\the\spacefactor}\/\fi
\hbox{$^{(#1)}$}\@sf\vfootnote{#1.}}
\def\vfootnote#1{\insert\footins\bgroup
  \interlinepenalty\interfootnotelinepenalty
  \splittopskip\ht\strutbox % top baseline for broken footnotes
  \splitmaxdepth\dp\strutbox \floatingpenalty\@MM
  \leftskip\z@skip \rightskip\z@skip \spaceskip\z@skip \xspaceskip\z@skip
  \textindent{#1}\footstrut\futurelet\next\fo@t}
\def\fo@t{\ifcat\bgroup\noexpand\next \let\next\f@@t
  \else\let\next\f@t\fi \next}
\def\f@@t{\bgroup\aftergroup\@foot\let\next}
\def\f@t#1{#1\@foot}
\def\@foot{\strut\egroup}
\def\footstrut{\vbox to\splittopskip{}}
\skip\footins=\bigskipamount % space added when footnote is present
\count\footins=800 % footnote magnification factor (1 to 1)
\dimen\footins=8in % maximum footnotes per page

%
%\anote=\footnote de plain (2 arguments)
%
\def\anote#1#2{\footnotea{\hbox{$^{#1}$}}{\eightpts#2}}
%\catcode`@=12 % the "@" signs are no longer letters

%reference de note
 \def\defrefnote#1{\definexref{#1}{{\the\footnotenumber}}{refnotes}}
 \def\refnote#1{\ref{#1}}



%\def\note#1#2{\footnote{#1}{\noindent\eightpts #2\vskip-10pt\par}}
%\def\note#1#2{\hbox{$^{(#1)}$}\footnote{}{\eightpts$#1.$#2\vskip-10pt\par}}
\def\pclm#1#2{\noi\bf#1\sl\ #2}

\def\footnoterule{\kern -2pt\hrule width 7truecm\kern 2.4pt}

%reference croisee a des numeros de notes
\def\xnotedef#1{\definexref{#1}{\noexpand\number\footnotenumber}{Note}}%
\def\noteref{\ref}


%
% Fin notes
%



%%%%%%%%%%%%%%%%%%%%%%%%%%%%%%%%%%%%%%%%%%%%%%%%%%%%%
%                                                                                                                                            %
%                                  Numérotation des équation (à gauche par défaut)                           %
%                                                                                                                                            %
%%%%%%%%%%%%%%%%%%%%%%%%%%%%%%%%%%%%%%%%%%%%%%%%%%%%%

 % choix numerotation gauche ou droite: defaut=gauche, \numadroite{oui} -> a droite
 
  
\newif\ifNumGauche
\NumGauchetrue

\ifNumGauche\let\eqno=\leqno\let\eqalignno=\leqalignno
\else\let\leqno=\eqno\let\leqalignno=\eqalignno
\fi

%%%%%%%%%%%%%%%%%%%%%%%%%%%%%%%%%%%%%%%%%%%%%%%%%%%%%
%                                                                                                                                            %
%                                  Insertion de dessins et d'illustrations                                             %
%                                                                                                                                            %
%%%%%%%%%%%%%%%%%%%%%%%%%%%%%%%%%%%%%%%%%%%%%%%%%%%%%

\def\picture #1 by #2 (#3){\leavevmode\vbox to #2{
   \hrule width #1 height 0pt depth 0pt
    \vfill
     \special{picture #3}}}



%fleches

\def\grad{\mathop{\rm grad}\nolimits}

\def\gradv{\mathop{\overrightarrow{\rm grad}}\nolimits}
%\def\vec{\overrightarrow}


\def\re{{\Re e\ }}
\def\im{{\Im m\ }}


\def\emptyset{{\varnothing}}


\def\|{\Vert}
\let\vieuxle=\le
\let\vieuxge=\ge
\def\le{\ifHtml\vieuxle\else\leqslant\fi}\def\leq{\leqslant}
\def\ge{\geqslant}\def\geq{\geqslant}
\def\wh{\widehat}
\def\§{\S\thinspace}

%% Variables

\newdimen\oldim
\newbox\olbox
\newcount\olcnt



%%
%%%%
%%%%%%%%%%%%%%%%%%%%%%%%%%%%%%%%%%% OL
%%%%
%%



%%% binome de Newton
\def\bn#1#2{\setbox\olbox\vbox{\hbox{\Math{\scriptstyle #1}}%
\hbox{\Math{\scriptstyle #2}}}
\Math{\left(\vcenter{\hbox to\wd\olbox{\hfill\Math{\scriptstyle
#1}\hfill}\hbox to\wd\olbox{\hfill\Math{\scriptstyle #2}\hfill}}
\right)}}
%%%%%%%%%%%%%%%%%%%%%%%%%%%%%%%%%%%%%%%%%%%%%%%%%%%%%%%%%%%%%%%%%

\def\ps#1\ps{\left\langle #1\right\rangle }
\def\pss#1\pss{\left[#1\right]}

\def\ul{\underline}
\def\ol{\overline}



\def\N#1{\Q|\!\Q|#1\W|\!\W|}


%Insertion de figures et illustrations
\def\picture #1 by #2 (#3){\leavevmode\vbox to #2{
   \hrule width #1 height 0pt depth 0pt
    \vfill
     \special{picture #3}}}


%% pour aller a la ligne dans un \proclaim:
 \def\PAR{\par}

%%%%%%%%%%%%%%%%%%%%%%%%%%%%%%%%%%%%%%%%%%%%%%%%%%%%%%%%%%%%%%%%%%%%
%                                                                                                                                                                                      %
%                                                            Books vs  Slides                                                                                              %
%                                                                                                                                                                                     %
%%%%%%%%%%%%%%%%%%%%%%%%%%%%%%%%%%%%%%%%%%%%%%%%%%%%%%%%%%%%%%%%%%%%

\long\def\book#1\book{\ifbook #1\fi}
\def\Book#1{\ifbook #1\fi}
\long\def\slide#1\slide{\ifbook\relax\else #1\fi}
\def\Slide#1{\ifbook\relax\else #1\fi}



%%%%%%%%%%%%%%%%%%%%%%%%%%%%%%%%%%%%%%%%%%%%%%%%%%%%%%%%%%%%%%%%%%%%
%                                                                                                                                                                                      %
%                                                                         Racourcis                                                                                            %
%                                                                                                                                                                                      %
%%%%%%%%%%%%%%%%%%%%%%%%%%%%%%%%%%%%%%%%%%%%%%%%%%%%%%%%%%%%%%%%%%%%

%\def\Prerequis{\noindent{\bf Pr\'e-requis}}
%\def\Corollaire{\noindent{\bf Corollaire}}
%\def\Attention{\noindent$\underline{\it Attention}$}
\def\Rappel{\noindent{\it Rappel}}
\def\Bullet{\noindent\item{$\bullet$}}
\def\Det{\sbox{D\'et}}
\def\Id{\sbox{Id}}
%\def\Ker{\sbox{Ker}\ }
\def\IM{\sbox{Im}}
\def\Vect{\sbox{Vect}}
\def\Dim{\sbox{Dim}}
\def\rang{\sbox{Rang}\ }
\def\dim{\sbox{dim}\ }
\def\ieme{{\mbox{\sevenrm i\`eme}}}
\def\er{{\mbox{\sevenrm er}}}
\def\nd{{\mbox{\sevenrm nd}}}
\def\Card{\sbox{Card}\ }


\def\Tr{\sbox{Tr}}



%%%%%%%%%%%%%%%%%%%%%%%%%%%%%%%%%%%%%%%%%%%%%%%%%%%%%%%%%%%%%%%%%%%%
%                                                                                                                                                                                      %
%                                                                         Racourcis                                                                                            %
%                                                                                                                                                                                      %
%%%%%%%%%%%%%%%%%%%%%%%%%%%%%%%%%%%%%%%%%%%%%%%%%%%%%%%%%%%%%%%%%%%%

%\newcount\olnumF   
%\global\olnumF=1
%\def\Figure #1, #2.{\centerline{{\bf Figure  \definexref{#1}{\the\olnumF}{Figure}
%\eqrefn{#1}\global\advance\olnumF by 1.\enspace--- }#2}}


\def\Per{{\hbox{\tensy\char63}}}
\def\per{{\hbox{\sevensy\char63}}}
\def\LuciferCross{\ \rput{0}(0,.1){\scalebox{1 -1}{\dag}}\ }
\def\LuciferStar{\ \rput{0}(0,.1){\scalebox{1 -1}{\star}}\ }
\def\App#1\to#2,#3\mapsto#4\ppA{\vtop{\hbox{$#1$}\hbox{$#3$}}\vtop{\hbox{$\to$}\hbox{$\mapsto$}}\vtop{\hbox{$#2$}\hbox{$#4$}}}


%%%%%%%%%%%%%%%%%%%%%%%%%%%%%%%%%%%%%%%%%%%%%%%%%%%%%%%%%%%%%%%%%%%%
%                                                                                                                                                                                      %
%                                                                         Exos                                                                                                    %
%                                                                                                                                                                                      %
%%%%%%%%%%%%%%%%%%%%%%%%%%%%%%%%%%%%%%%%%%%%%%%%%%%%%%%%%%%%%%%%%%%%


\def\pn{\par\noindent}
\def\Ind{{\it Indication : }}

%%%  Numérotation automatique des exercices

\newcount\numeroexo\numeroexo=0
\newcount\numerosectionexo\numerosectionexo=0    
\newif\ifexolabel\exolabeltrue
\newif\ifexosection\exosectionfalse
\def\filiere{PTSI}

\def\exo#1. #2\par{\expandafter\def\csname exo#1\endcsname{
\global\advance\numeroexo by 1%
\ifexosection\definexref{labelexo\filiere#1}{\the\numerosectionexo.\the\numeroexo}{exo}\else\definexref{labelexo\filiere#1}{\the\numeroexo}{exo}\fi\noindent\ifexolabel\eqrefn{labelexo\filiere#1}\else#1\fi) #2\medskip}}

\newcount\numeroprobleme\numeroprobleme=0
\newif\ifproblemelabel\problemelabeltrue
\def\Probleme#1. #2\par{\expandafter\def\csname Probleme#1\endcsname{\global\advance\numeroprobleme by 1%
\definexref{labelprobleme\filiere#1}{\the\numeroprobleme}{probleme}\centerline{\bf Probl\`eme \ifproblemelabel\eqrefn{labelprobleme\filiere#1}\else#1\fi}\medskip\noindent #2\medskip}}


\def\sol#1. #2\par{\expandafter\def\csname sol#1\endcsname{#2\medskip}}


\catcode`@=11



%%%%%%%%%%%%%%%%%%%%%%%%%%%%%%%%%%%%%%%%%%%%%%%%%%%%%%%%%%%%%%%%%%%%
%                                                                                                                                                                                      %
%                                                        Pour se débarasser des espaces de début et de fin                                            %
%                                                                                                                                                                                      %
%%%%%%%%%%%%%%%%%%%%%%%%%%%%%%%%%%%%%%%%%%%%%%%%%%%%%%%%%%%%%%%%%%%%




%% The Solution
\def\trim #1{\expandafter\trim@\expandafter{#1 }#1}
\def\trim@ #1{\trim@@ @#1 @ #1 @ @@}
\def\trim@@ #1@ #2@ #3@@{\trim@@@\empty #2 @}
\def\unbrace#1{#1}
\unbrace{\def\trim@@@ #1 } #2@#3{\expandafter\def
  \expandafter #3\expandafter {#1}}


%
%  synthaxe : \def\ol{ gah } \trim\ol le resultat est dans \ol

%% Test Code
%\def\Test{\afterassignment\Test@ \def\test}
%\def\Test@{\trim\test \afterassignment\Test@@ \def\test@}
%\def\Test@@{\message{\ifx\test\test@ Y\else FAIL:|\meaning\test|\fi}}


%% Testing The Solution
%\Test{}{}
%\Test{ }{}
%\Test{ a }{a}
%\Test{ {}{} }{{}{}}
%\Test{{braces}}{{braces}}
%\Test{ {braces} }{{braces}}
%\Test{ { braces } }{{ braces }}
%\Test{no space and no space}{no space and no space}
%\Test{no space and a space: }{no space and a space:}
%\Test{ :a space and no space}{:a space and no space}
%\Test{ :a space and a space: }{:a space and a space:}
%\Test{ \ifx }{\ifx}
%\Test{ \ifx/ }{\ifx/}



%%%%%%%%%%%%%%%%%%%%%%%%%%%%%%%%%%%%%%%%%%%%%%%%%%%%%%%%%%%%%%%%%%%%
%                                                                                                                                                                                      %
%                                                                                      Equations  numérotées                                                           %
%                                                                                                                                                                                      %
%%%%%%%%%%%%%%%%%%%%%%%%%%%%%%%%%%%%%%%%%%%%%%%%%%%%%%%%%%%%%%%%%%%%



\def\Equation{\@getoptionalarg\@LEquation}
\def\@LEquation#1$$#2$${\ifx\@optionalarg\empty\def\tempol{\the\eqnumber}\advance\eqnumber by1\else\edef\tempol{\@optionalarg}\fi
\def\argoL{#1}\trim\argoL\ifx\argoL\empty\else\definexref{\argoL}{\tempol}{eq}\fi
\ifHtml
$$\displaystyle\eqalignno{&\displaystyle #2&\mbox{(\bf\tempol)}}$$\else
$$#2\leqno{\hbox{(\tempol)}}$$\fi}


%
%\def\Equation{\@getoptionalarg\@LEquation}
%\def\@LEquation#1$$#2$${\ifx\@optionalarg\empty\def\tempol{\the\eqnumber}\advance\eqnumber by1\else\edef\tempol{\@optionalarg}\fi
%\def\argoL{#1}\trim\argoL\ifx\argoL\empty\else\definexref{\argoL}{\tempol}{eq}\fi
%\ifHtml
%$$\HCode{<mtable><mlabeledtr side="left"><mtd>}\mbox{(\tempol)}\HCode{</mtd><mtd><mrow>}#2\HCode
%{</mrow></mtd></mlabeledtr></mtable>}$$\else
%$$#2\leqno{\hbox{(\tempol)}}$$\fi}




%%%%%%%%%%%%%%%%%%%%%%%%%%%%%%%%%%%%%%%%%%%%%%%%%%%%%%%%%%%%%%%%%%%%
%                                                                                                                                                                                      %
%                                                        Définitions, propriétés, théorèmes, lemmes                                                        %
%                                                                                                                                                                                      %
%%%%%%%%%%%%%%%%%%%%%%%%%%%%%%%%%%%%%%%%%%%%%%%%%%%%%%%%%%%%%%%%%%%%


% Pour inverser l'ordre Hypothèses-propriétés. 


\newif\ifInverted\Invertedfalse


% La macro qui se charge du formatage en couleur Html/TeX - Normal/Inversé
\newdimen\OLdimen


%\DefineRGBcolor F0F9E3=VLGreen.
%\DefineRGBcolor E5F9D1=LGreen.
%\DefineRGBcolor 5DA93B=Green.

\def\ColorFrame{\@getoptionalarg\@LColorFrame}%
\def\@LColorFrame#1#2{%
\noindent
\ifHtml % Html
\ifInverted % Inverted
\EndP
\HCode{<div class="ICF#1">}\vbox{\ignorespaces#2\EndP}\HCode{</div>}
\ifx\@optionalarg\empty\else\HCode{<div class="centerline"><span class="ICFHyp#1">}\@optionalarg\HCode{</span></div>}\fi
\Invertedfalse
\else % Inverted false
\EndP
\ifx\@optionalarg\empty\else
\HCode{<div class="centerline"><span class="CFHyp#1">}\@optionalarg\HCode{</span></div>}\fi
\HCode{<div class="CF#1">}\vbox{\noindent\ignorespaces#2\EndP}
\HCode{</div>}
\fi
\else  % Html false
\tikzstyle{body} = [draw=Color#1, fill=ColorL#1, thick,
    rectangle, rounded corners, inner xsep=5pt, inner ysep=10pt]
\tikzstyle{head}= [draw=Color#1, fill=ColorVL#1, thick,
    rectangle, rounded corners, inner xsep=3pt, inner ysep=3pt]
\ifInverted
\ifx\@optionalarg\empty
\OLdimen=\hsize\advance\OLdimen by -15.3pt
\tikzpicture
\node[text width=\OLdimen,body] {\noindent#2} ;
\endtikzpicture
\PAR
\else
\OLdimen=\hsize\advance\OLdimen by -15.3pt
\tikzpicture
\node[text width=\OLdimen,body] (a) {\noindent#2} ;
\node [head] at (a.south) {\noindent\@optionalarg} ;
\endtikzpicture
\PAR
\fi
\vskip15pt
\Invertedfalse
\else % Inverted false
\ifx\@optionalarg\empty
\bigskip\noindent
\OLdimen=\hsize\advance\OLdimen by -15.3pt
\tikzpicture
\node[text width=\OLdimen,body] {\noindent#2} ; 
\endtikzpicture
\PAR
\else
%¨
\OLdimen=\hsize\advance\OLdimen by -15.3pt
\tikzpicture
\node[text width=\OLdimen,body] (a) {\noindent#2} ;
\node [head] at (a.north)  {\noindent\@optionalarg} ;
\endtikzpicture
\PAR
\vskip9pt\fi
\fi\fi}


%Définition

\def\Definition{\@getoptionalarg\@LDefinition}
\def\@LDefinition#1\par{\ifx\@optionalarg\empty\@LColorFrame{Green}{#1}\else
\edef\temp{[\@optionalarg]}\expandafter\ColorFrame\temp{Green}{#1}\fi
}

%Propriété

\def\Propriete{\@getoptionalarg\@LPropriete}
\def\@LPropriete#1\par{\ifx\@optionalarg\empty\@LColorFrame{Orange}{#1}\else
\edef\temp{[\@optionalarg]}\expandafter\ColorFrame\temp{Orange}{#1}\fi
}

%Théorème

\def\Theoreme{\@getoptionalarg\@LTheoreme}
\def\@LTheoreme#1\par{\ifx\@optionalarg\empty\@LColorFrame{Red}{#1}\else
\edef\temp{[\@optionalarg]}\expandafter\ColorFrame\temp{Red}{#1}\fi
}


% Nocolor

\def\Assertion{\@getoptionalarg\@LAssertion}
\def\@LAssertion#1\par{\ifx\@optionalarg\empty\@LColorFrame{Black}{#1}\else
\edef\temp{[\@optionalarg]}\expandafter\ColorFrame\temp{Black}{#1}\fi
}




%%%%%%%%%%%%%%%%%%%%%%%%%%%%%%%%%%%%%%%%%%%%%%%%%%%%%%%%%%%%%%%%%%%%
%                                                                                                                                                                                      %
%                                      Démonstrations, Exemples, Applications, Exercices, Remarques                                       %
%                                                                                                                                                                                      %
%%%%%%%%%%%%%%%%%%%%%%%%%%%%%%%%%%%%%%%%%%%%%%%%%%%%%%%%%%%%%%%%%%%%


\long\def\Demonstration. #1\CQFD{\noindent{\eightpts D\'emonstration. #1}\bigskip}
\long\def\Exemple : #1\par{\noindent{\eightpts{\it Exemple : }#1}\medskip}
\long\def\Exemples : #1\par{\noindent{\eightpts{\it Exemples : }#1}\medskip}
\long\def\Application : #1\par{\noindent{\eightpts{\it Application : }#1}\medskip}
\long\def\Exercice. #1\par{\noindent{\eightpts{\bf Exercice.} #1}\bigskip}
\def\Remarque{\noindent{\it Remarque}}
\def\Remarques{\noindent{\it Remarques}}
\def\Notation{\noindent{\it Notation}}

%%%%%%%%%%%%%%%%%%%%%%%%%%%%%%%%%%%%%%%%%%%%%%%%%%%%%%%%%%%%%%%%%%%%
%                                                                                                                                                                                      %
%                                                                Text inside maths / ensuremaths                                                                 %
%                                                                                                                                                                                      %
%%%%%%%%%%%%%%%%%%%%%%%%%%%%%%%%%%%%%%%%%%%%%%%%%%%%%%%%%%%%%%%%%%%%


\def\mbox#1{{\ifHtml\PauseMathClass\mtexttrue\HCode{<mtext>}\hbox{#1}\HCode{</mtext>}\EndPauseMathClass\else \hbox{#1}\fi}} 
\def\sbox#1{{\ifHtml\PauseMathClass\mtexttrue\HCode{<mtext>&nbsp;}\hbox{#1}\HCode{&nbsp;</mtext>}\EndPauseMathClass\else \hbox{#1}\fi}} 

\def\cced{\ifHtml \HCode{&ccedil;}\else\c c\fi}
\def\Math#1{\ifmmode #1\else $#1$\fi}


% \newtoks\tks 
%       \def\mbox{\ifHtml \afterassignment\mya \else \myb\fi \tks} 
%       \def\mya{\HCode{<mtext>}\hbox{\PauseMathClass\mtexttrue  
%                           \the\tks\EndPauseMathClass}\HCode{</mtext>}} 
%       \def\myb{\hbox{\the\tks}} 


%%%%%%%%%%%%%%%%%%%%%%%%%%%%%%%%%%%%%%%%%%%%%%%%%%%%%%%%%%%%%%%%%%%%
%                                                                                                                                                                                      %
%                                                                                  Colors                                                                                         %
%                                                                                                                                                                                      %
%%%%%%%%%%%%%%%%%%%%%%%%%%%%%%%%%%%%%%%%%%%%%%%%%%%%%%%%%%%%%%%%%%%%

\newcount\olcounta%
\newcount\olcountb%
\newcount\olloopcount%
\def\DefineRGBcolor #1#2#3#4#5#6=#7.{%
\def\olRGB{}%
%
\def\space{ }
\def\ConvertHD##1{%%%%% convertion nombre hexadecimal de deux chiffres vers fraction decimale à 6 chiffres après la virgule
\def\olnumber{##1}
\unless\ifnum\olnumber=255\relax
	\def\oldd{}\olloopcount=0\loop
	\olcounta=\olnumber\multiply\olcounta by 10\divide\olcounta by "FF\edef\oltemp{\oldd\the\olcounta}\edef\oldd{\oltemp}\olcountb=\olnumber
	\multiply\olcountb by 10\multiply\olcounta by"FF\advance\olcountb by-\olcounta
	\edef\olnumber{\the\olcountb}\advance\olloopcount by 1\ifnum\olloopcount<5\repeat
	\edef\oltemp{\olRGB}\edef\olRGB{\oltemp.\oldd}
\else
	\edef\oldd{999999}%
	\edef\olRGB{\olRGB.\oldd}
\fi
}%%-->>resultat dans oldd et \olRGB
\ConvertHD{"#5#6}\edef\tempa{,.\oldd}\ConvertHD{"#3#4}\edef\tempa{,.\oldd\tempa}\ConvertHD{"#1#2}\edef\tempa{.\oldd\tempa}%
\edef\temp{{Color#7}{rgb}{\tempa}}
\expandafter\definecolor\temp
%\newrgbcolor{Color#7}{\olRGB}
}


\def\MakeRGBcolor #1=#2.{%
\DefineRGBcolor #1=#2.
\expandafter\def\csname#2\endcsname##1{
{\csname Color#2\endcsname##1}
#1}}




\def\pagehsize{\hsize}


%%%%%%%%%%%%%%%%%%%%%%%%%%%%%%%%%%%%%%%%%%%%%%%%%%%%%%%%%%%%%%%%%%%%
%                                                                                                                                                                                      %
%                                                                             Racourcis                                                                                        %
%                                                                                                                                                                                      %
%%%%%%%%%%%%%%%%%%%%%%%%%%%%%%%%%%%%%%%%%%%%%%%%%%%%%%%%%%%%%%%%%%%%

\def\ssi{\quad\Longleftrightarrow\quad}       % Equivalent
\def\diff{\mbox{\rm d}}                                             % différentielle 
\def\NULL{\ifHtml{\HCode{}}\else{}\fi}
\def\pn{\ifHtml\noindent\HCode{<br />}\else\par\noindent\fi}





\let\ds=\displaystyle
\let\ss=\scriptstyle
\let\sss=\scriptscriptstyle
\let\|=\parallel

\def\[{[}
\def\]{]}



\def\Center#1{\ifHtml\EndP\HCode{<div align="center">}#1\HCode{</div>}\Par\else\centerline{#1}\fi}
\def\OLSpace{ }
\def\§{\ifHtml\HCode{&\#160;}\else\OLSpace\fi}
%
%%% Code Roadmap
%
%-> Index, Toc, Concept, Definition, Inversed, Colors, Propriete
%