\catcode`@=11\relax
\input LD@Header.tex
\input qhsort.tex
\catcode`@=11\relax
\input LD@Library.tex
\input LD@Akuma@Japanese.tex
\catcode`@=11\relax
\catcode`@=11\relax
\input LD@Luatex.tex


\font\LD@Font@Kana={C:/Windows/Fonts/HGRSKP.ttf} at 10pt %{HGSeikaishotaiPRO} at 10pt
\lccode"2019="0027\relax% to get hyphen with words with apostrophes
\LD@Font@Kana
\long\def\Ignore #1\Ignore{}%
\Ignore
%%
%% Fontes utilisées
%%

\def\FontCSName #1 #2\EndFontCSName{%
	\EA\gdef\CS #1\EC##1{%
		\unless\ifcsname #1##1\endcsname 
			\EA\font\CS #1##1\EC="#2" at ##1pt%
		\fi
		\CS #1##1\EC
	}%
}%
\def\FontCSDef #1.{%
	\for\OLFont:=#1\do{%
		\EA\FontCSName\OLFont\EndFontCSName
	}%
}%

\FontCSDef Arial Arial,ArialB Arial Bold,ArialIT Arial Italic,ArialBIT Arial Bold Italic,Kaisho HGSeikaishotaiPRO,Mincho MS Mincho.%%% Automated but might bring Errors

\def\FontDef #1.{%
	\def\FontName ##1=##2@##3\EndFontName{%
		\EA\global\EA\font\CS ##1\EC="##2" at ##3pt%
	}%
	\for\OLFont:=#1\do{%
		\EA\FontName \OLFont\EndFontName
	}%
}%

%\FontDef Romaji=Arial@5,Grade=Arial@4.






%\font\Fontindex="HGSeikaishotaiPRO" at 10pt


\%font\StrokeFont="Arial:color=FF0F0Fl" at 4pt

%%%%%%%%%%%%%%%%%%%%%%%%%%%%%%%%%%%%%%%%%%%%%%%%%%%%%%%%%%%%%%%%%%%%%%%%%%%%%%%%%%%%%%%%%%%%%%%%%%%%%%%%%%%%%%%%%%%%%%
%%%%%%%%                                                                                                            Fontes
%%%%%%%%%%%%%%%%%%%%%%%%%%%%%%%%%%%%%%%%%%%%%%%%%%%%%%%%%%%%%%%%%%%%%%%%%%%%%%%%%%%%%%%%%%%%%%%%%%%%%%%%%%%%%%%%%%%%%%
% -> Dingbats
%\font\LD@Dingbats@Japan="japanapush" at 12pt%      r=rice bowl,   t=bentou,    v=temple,     w=zen gate,     z=japan
%\font\LD@Dingbats@X="Geisha" at 12pt
%\font\LD@Dingbats@MartialArts="McCoy Dingbat Karate" at 12pt



\font\LD@Font@IndexNumber="Arial" at 6pt
\font\LD@Font@Nihongo="HGSeikaishotaiPRO" at 28pt
\font\LD@Font@BushuArray="HGSeikaishotaiPRO" at 10pt
\font\LD@Font@BushuMeaningsArray="Arial" at 6pt
\font\LD@Font@BigKanji="HGSeikaishotaiPRO" at 66pt
\font\LD@Font@Mincho="MS Mincho" at 18pt
\font\LD@Font@Word="HGSeikaishotaiPRO" at 10pt
\font\LD@Font@Kana="HGSeikaishotaiPRO" at 10pt
\font\LD@Font@WordKana="HGSeikaishotaiPRO" at 10pt
\font\LD@Font@Bushu="HGSeikaishotaiPRO" at 10pt
\font\LD@Font@Meanings="Arial" at 8pt
\font\LD@Font@BushuMeanings="Arial" at 6pt
\font\LD@Font@Numbers="Arial" at 4pt
\font\LD@Font@Index="Arial" at 5pt
\font\LD@Font@IndexItalic="Arial Italic" at 5pt
\font\LD@Font@TinyKaisho="HGSeikaishotaiPRO" at 6pt
\font\LD@Font@TinyArial="Arial" at 4pt
\font\LD@Font@One="Arial" at 8pt
\font\LD@Font@Two="Arial Italic" at 8pt
\newcount\LD@Font@Number\global\LD@Font@Number=0\relax
\def\LD@Font@Set{% 
	\global\advance\LD@Font@Number by 1\relax
	\ifcase\LD@Font@Number
	\or
		\global\let\LD@Font@Sens\LD@Font@One
	\else
		\global\let\LD@Font@Sens\LD@Font@Two
		\global\LD@Font@Number=0\relax
	\fi
}%
\def\LD@Font@Reset{%
	\global\LD@Font@Number=0\relax
	\LD@Font@Set
}%
\LD@Font@TinyKaisho
\Ignore











%%%%%%%%%%%%%%%%%%%%%%%%%%%%%%%%%%%%%%%%%%%%%%%%%%%%%
%												%
%				\LD@Kanjidic@Load						%
%												%
%%%%%%%%%%%%%%%%%%%%%%%%%%%%%%%%%%%%%%%%%%%%%%%%%%%%%
\directlua1{
	KanjiData={}
	IndexToKanji={}
	LD_Kanji_Data={}
}%
\newcount\LD@XML@Entries
\def\LD@Kanjidic@Load{%
	\input KanjiList.utf
	\for\KanjiVar:=\JouyouKanji\do{%
		\EA\def\csname Level\KanjiVar\endcsname{4}%
	}%   Let's sanitize the JLPT !
	\for\KanjiVar:=\KanjiJLPTDeux\do{%
		\EA\def\csname Level\KanjiVar\endcsname{3}%
	}%
	\for\KanjiVar:=\KanjiJLPTTrois\do{%
		\EA\def\csname Level\KanjiVar\endcsname{2}%
	}%
	\for\KanjiVar:=\KanjiJLPTQuatre\do{%
		\EA\def\csname Level\KanjiVar\endcsname{1}%
	}%
	\for\KanjiVar:=\JouyouKanjiNotInJLPTun\do{%
		\EA\def\csname Level\KanjiVar\endcsname{5}%
	}%
	\global\LD@XML@Entries=0\relax
	\def\LD@XML@character@Do{%
		\global\advance\LD@XML@Entries by1\relax
		\ifcsname Level\LD@Kanji\endcsname
			\edef\LD@JLPT{\CS Level\LD@Kanji\EC}%
		\else
			\def\LD@JLPT{5}%
		\fi
		\EA\xdef\CS Data\LD@Kanji\EC{Level={\LD@JLPT} Bushu={\LD@Kanji@Classical} Grade={\LD@Grade} Stroke Count={\LD@StrokeCount} Frequency={\LD@Frequency} On={\LD@Reading@On} Kun={\LD@Reading@Kun} Nanori={\LD@Reading@Nanori} Meanings={\LD@Meaning@Buffer} Grouped={\LD@Reading@Grouped}}%
		\directlua1{
			index="\LD@Kanji"
			KanjiData[index.." Level"]="\LD@JLPT"
			KanjiData[index.." Bushu"]="\LD@Kanji@Classical"
			KanjiData[index.." Grade"]="\LD@Grade"
			KanjiData[index.." Stroke Count"]="\LD@StrokeCount"
			KanjiData[index.." Frequency"]="\LD@Frequency"
			KanjiData[index.." On"]="\LD@Reading@On"
			KanjiData[index.." Kun"]="\LD@Reading@Kun"
			KanjiData[index.." Nanori"]="\LD@Reading@Nanori"
%				["\LD@Kanji Meanings"]="\LD@Meaning@Buffer",
%				["\LD@Kanji Grouped"]="\LD@Reading@Grouped"
		}%
		\EA\xdef\CS\the\LD@XML@Entries\EC{\LD@Kanji}%
		\directlua1{
				index="\LD@XML@Entries"
				IndexToKanji[index]="\LD@Kanji"
		}%
		\unless\ifx\LD@Frequency\LD@Empty
			\EA\xdef\CS KanjiN\LD@Frequency\EC{\LD@Kanji}%
		\fi
	}%
	\input LD@XMLParser.tex
	\input LD@Akuma@KanjidictParser.tex
	\LD@XML@Parser{kanjidic2.xml}%
}%
%%%%%%%%%%%%%%%%%%%%%%%%%%%%%%%%%%%%%%%%%%%%%%%%%%%%%
%												%
%					Yomigata Preprocessing				%
%												%
%%%%%%%%%%%%%%%%%%%%%%%%%%%%%%%%%%%%%%%%%%%%%%%%%%%%%
\gdef\LD@KanjiData@Verbs{%
	|うい|くき|ぐぎ|すし|つち|るり|ぬに|むみ|ぶび|=%
}%
\gdef\LD@KanjiData@RuVerbs{%
	きぎけげしじせぜちぢてでひびぴへべぺりれいえみめにね%
}%
\gdef\LD@KanjiData@GodanRuVerbs{%
	|帰る,かえる|入る,はいる|切る,きる|知る,しる|要る,いる|%
}%
\gdef\LD@KanjiData@Start{%
	|かY,がN|きY,ぎN|くY,ぐN|けY,げN|こY,ごN|さY,ざN|しY,じN|すY,ずN|せY,ぜN|そY,ぞN|たY,だN|ちY,ぢN,じN|つY,づN,ずN|てY,でN|とY,どN|はY,ばN,ぱN|ひY,びN,ぴN|ふY,ぶN,ぷN|へY,べN,ぺN|=%
	|ほY,ぼN,ぽN|カY,ガN|キY,ギN|クY,グN|ケY,ゲN|コY,ゴN|サY,ザN|シY,ジN|スY,ズN|セY,ゼN|ソY,ゾN|タY,ダN|チY,ヂN,ゾN|ツY,ヅN,ズN|テY,デN|トY,ドN|ハY,バN,パN|ヒY,ビN,ピN|フY,ブN,プN|=%
	|ヘY,ベN,ペN|ホY,ボN,ポN|=%
}% N means in the middle Y at beginning
\gdef\LD@KanjiData@End{%
	|つN,っN,つY|くN,っN,くY|っN,つY|ツN,ッN,ツY|ッN,ツY|クN,ッN,クY|=%
}% N means in the middle Y at end
\def\LD@KanjiData@Add#1#2#3{% Adds the reading variation in the #1#2 Control sequence, if it isn't already in. 
	\ifcsname LD@KanjiData@#1#2#3\endcsname
		\edef\LD@Temp@G{,\LD@Temp@C\LD@KanjiData@Middle\LD@Temp@E ,}%
		\edef\LD@Temp@H{\CS LD@KanjiData@#1#2#3\EC\LD@Temp@G}%
		\EA\LD@Split\EA\LD@Temp@H\EA{\LD@Temp@G}\LD@Temp@Trash\LD@Temp@H
		\ifx\LD@Temp@H\LD@Empty
			\EA\xdef \CS LD@KanjiData@#1#2#3\EC{\CS LD@KanjiData@#1#2#3\EC\LD@Temp@C\LD@KanjiData@Middle\LD@Temp@E ,}%
			\directlua1{
				index="#1#2#3"
				LD_Kanji_Data[index] = LD_Kanji_Data[index] .. "\LD@Temp@C\LD@KanjiData@Middle\LD@Temp@E ,"
			}%
		\fi			
	\else
		\EA\xdef \CS LD@KanjiData@#1#2#3\EC{,\LD@Temp@C \LD@KanjiData@Middle\LD@Temp@E ,}%
		\directlua1{
			index="#1#2#3"
			LD_Kanji_Data[index]=",\LD@Temp@C\LD@KanjiData@Middle\LD@Temp@E ,"
		}%
	\fi
}%
\def\LD@KanjiData@Reading@AddToPool#1{%
		% Count the number of Tokens of the On reading \LD@Temp@A			
		\LD@String@Length#1\LD@TempCounter@A
		\ifnum\LD@Counter@A<\LD@TempCounter@A\relax
			\LD@Counter@A=\LD@TempCounter@A\relax
		\fi
		% Adds it to the Reading Pool
		\ifcsname  LD@KanjiData@\the\LD@TempCounter@A\endcsname
			\EA\let\EA\LD@Temp@B\CS LD@KanjiData@\the\LD@TempCounter@A\EC
			\EA\edef\CS LD@KanjiData@\the\LD@TempCounter@A\EC{%
				\LD@Temp@B ,#1%
			}%
		\else
			\EA\edef\CS LD@KanjiData@\the\LD@TempCounter@A\EC{#1}%
		\fi
}%

\def\LD@Yomigana@Pool@Make#1#2#3#4{% #1  is the Kanji, #2 is on reading, #3 is Kun Reading and #4 is Nanori Reading output is the reading pool : LD@KanjiData@+number  max number is indicated by \LD@Counter@A
	%%%% From this point, we pre-process the reading of the kanji
	\LD@Counter@A=0\relax
	%% Takes care of the On lectures
	\for\LD@KanjiData@LoopA:=#2\do{%
		% Hiraganize the reading
		\LD@String@Hiraganize\LD@KanjiData@LoopA\LD@Temp@A % undead but doesn't use \LD@Temp@A
		\LD@KanjiData@Reading@AddToPool\LD@Temp@A
	}%
	%% Takes care of the Kun lectures
	\for\LD@KanjiData@LoopA:=#3\do{% 
		% take care of the "-" if there is one.
		\edef\LD@Temp@A{\LD@KanjiData@LoopA-}%
		\LD@String@Split\LD@Temp@A-\LD@Temp@A\LD@Temp@Trash
		\ifx\LD@Temp@A\LD@Empty	
			% -Kun  -->  -Kun-  -->  {} et Kun-  --> Kun
			\LD@String@Split\LD@Temp@Trash -\LD@Temp@A\LD@Temp@Trash
		\fi
		% remove the "." if there is one.
		\edef\LD@Temp@A{\LD@Temp@A.}%
		\LD@String@Split\LD@Temp@A.\LD@Temp@A\LD@Temp@Trash
		\unless\ifx\LD@Temp@Trash\LD@Empty
		% Verb-> noun Kun support
			\LD@String@Split\LD@Temp@Trash.\LD@Temp@B\LD@Temp@Trash
			\LD@String@LastChar\LD@Temp@B\LD@Temp@B\LD@Temp@C
			\edef\LD@Test{\LD@KanjiData@Verbs |\LD@Temp@C}%
			\EA\LD@String@Split\EA\LD@Test\EA{\EA |\LD@Temp@C}\LD@Temp@Trash\LD@Test
			\ifx\LD@Test\LD@Empty
				%not a verb
				\edef\LD@Temp@D{\LD@Temp@A\LD@Temp@B\LD@Temp@C}%
			\else
				\unless\if る\LD@Temp@C
					% a Godan verb that doesn't end with ru
					\LD@String@Split\LD@Test{}\LD@Temp@C\LD@Temp@Trash
					\edef\LD@Temp@D{\LD@Temp@A\LD@Temp@B\LD@Temp@C}%
				\else
					\edef\LD@Temp@D{\LD@Temp@A\LD@Temp@B}%
					\LD@String@LastChar\LD@Temp@D\LD@Trash\LD@Temp@E
					\edef\LD@Test{\LD@KanjiData@RuVerbs\LD@Temp@E}%
					\EA\LD@String@Split\EA\LD@Test\LD@Temp@E\LD@Trash\LD@Test
					\ifx\LD@Test\LD@Empty
						% Verb ending with aru, uru or oru 
						\edef\LD@Test{#1\LD@Temp@B\LD@Temp@C,\LD@Temp@A\LD@Temp@B\LD@Temp@C}%
						\def\LD@Trash{来る,くる}% 
						\ifx\LD@Trash\LD@Test
							\edef\LD@Temp@D{き}%
						\else
							\def\LD@Trash{為る,する}% 
							\ifx\LD@Trash\LD@Test
								\edef\LD@Temp@D{し}%
							\else
								% Godan Verb ending with aru, uru or oru 
								\edef\LD@Temp@D{\LD@Temp@A\LD@Temp@B り}%
							\fi
						\fi
					\else
						% a verb ending with iru or eru
						\edef\LD@Test{|#1\LD@Temp@B\LD@Temp@C,\LD@Temp@A\LD@Temp@B\LD@Temp@C|}%
						\edef\LD@Trash{\LD@KanjiData@GodanRuVerbs\LD@Test}% 
						\EA\LD@String@Split\EA\LD@Trash\EA{\LD@Test}\LD@Trash\LD@Test
						\ifx\LD@Test\LD@Empty
							% Ichidan ru Verb
							\edef\LD@Temp@D{\LD@Temp@A\LD@Temp@B}%
						\else
							% Godan ru Verb
							\edef\LD@Temp@D{\LD@Temp@A\LD@Temp@B り}%
						\fi
					\fi
				\fi
			 \fi
			\LD@KanjiData@Reading@AddToPool\LD@Temp@D
		\fi
		\LD@KanjiData@Reading@AddToPool\LD@Temp@A
	}%
	%% Takes care of the Nanori lectures
	\for\LD@KanjiData@LoopA:=#4\do{%
		\let\LD@Temp@A\LD@KanjiData@LoopA
		\LD@KanjiData@Reading@AddToPool\LD@Temp@A
	}%
}%
\def\LD@Yomigana@Pool@Process#1{%
	\LD@Counter@B=0\relax
	\loop
		\advance\LD@Counter@B by1\relax
		\ifcsname LD@KanjiData@\the\LD@Counter@B\endcsname
			\edef\LD@Temp@A{\CS LD@KanjiData@\the\LD@Counter@B\EC}%
			\for\LD@KanjiData@LoopA:=\LD@Temp@A\do{%
				\LD@Split\LD@KanjiData@LoopA{}\LD@KanjiData@First\LD@KanjiData@Middle
				% Get the possible beginnings of the Reading into \LD@Temp@A
				\edef\LD@Temp@A{\LD@KanjiData@Start |\LD@KanjiData@First}%
				\EA\LD@Split\EA\LD@Temp@A\EA{\EA|\LD@KanjiData@First}\LD@Temp@Trash\LD@Temp@A
				\ifx\LD@Temp@A\LD@Empty
					\edef\LD@Temp@A{\LD@KanjiData@First Y}%
				\else
					\edef\LD@Temp@A{\LD@KanjiData@First\LD@Temp@A}
					\LD@Split\LD@Temp@A |\LD@Temp@A\LD@Temp@Trash
				\fi
				\ifnum\LD@Counter@B>1\relax
					\LD@String@LastChar\LD@KanjiData@Middle\LD@KanjiData@Middle\LD@KanjiData@Last
					% Get the possible endings of the Reading into \LD@Temp@B
					\edef\LD@Temp@B{\LD@KanjiData@End |\LD@KanjiData@Last}%
					\EA\LD@Split\EA\LD@Temp@B\EA{\EA|\LD@KanjiData@Last}\LD@Temp@Trash\LD@Temp@B
					\ifx\LD@Temp@B\LD@Empty
						\edef\LD@Temp@B{\LD@KanjiData@Last Y,\LD@KanjiData@Last N}%
					\else
						\edef\LD@Temp@B{\LD@KanjiData@Last\LD@Temp@B}
						\LD@Split\LD@Temp@B |\LD@Temp@B\LD@Temp@Trash
					\fi
				\else
					\def\LD@Temp@B{{}Y,{}N}%
					\def\LD@KanjiData@Last{}%
				\fi
				{\for\LD@KanjiData@LoopB:=\LD@Temp@A\do{%
					\EA\LD@Edef@Two\LD@KanjiData@LoopB\LD@Temp@C\LD@Temp@D
					{\for\LD@KanjiData@LoopC:=\LD@Temp@B\do{%
						\EA\LD@Edef@Two\LD@KanjiData@LoopC\LD@Temp@E\LD@Temp@F
						\LD@KanjiData@Add{#1}\LD@Temp@D\LD@Temp@F
						\if Y\LD@Temp@D
							\LD@KanjiData@Add{#1}N\LD@Temp@F
						\fi
					}}%
				}}%
			}%
		\fi
	\ifnum\LD@Counter@B<\LD@Counter@A\relax
	\repeat
}%
\def\LD@Yomigana@Kanji@Preprocess#1#2#3#4{%
	\LD@Yomigana@Pool@Make{#1}{#2}{#3}{#4}%
	\LD@Yomigana@Pool@Process{#1}%
}%
\def\LD@Yomigana@Kanji@Preprocessing{%
	\global\LD@One=0\relax
	\loop
	{%%% grouping is for forgetting local parameters
		\global\advance\LD@One by1\relax
		\edef\LD@Kanji{\CS\the\LD@One\EC}%
		\LD@Data@Set{On}\LD@Kanji\LD@Reading@On
		\ifx\LD@Reading@On\relax
			\def\LD@Reading@On{}%
		\fi
		\LD@Data@Set{Kun}\LD@Kanji\LD@Reading@Kun
		\ifx\LD@Reading@Kun\relax
			\def\LD@Reading@Kun{}%
		\fi
		\LD@Data@Set{Nanori}\LD@Kanji\LD@Reading@Nanori
		\ifx\LD@Reading@Nanory\relax
			\def\LD@Reading@Nanory{}%
		\fi
		\EA\LD@Yomigana@Kanji@Preprocess\EA{\LD@Kanji}\LD@Reading@On\LD@Reading@Kun\LD@Reading@Nanori
	}%
	\ifnum\LD@One<\LD@XML@Entries\relax
	\repeat
	\LD@Time@Benchmark{Kanji-Preprocessing done}
}%
%%%%%%%%%%%%%%%%%%%%%%%%%%%%%%%%%%%%%%%%%%%%%%%%%%%%%
%												%
%					Pseudo-Kanji Preprocessing				%
%												%
%%%%%%%%%%%%%%%%%%%%%%%%%%%%%%%%%%%%%%%%%%%%%%%%%%%%%
%
%、。,.・:;?!゛゜´`¨^ ̄_ヽヾゝゞ〃仝ー―‐/\~‖|…‥''""()〔〕[]{}〈〉《》「」『』【】+−±×÷=≠<>≦≧∞∴♂♀°′″℃¥$¢£%#&*@§☆★○●◎◇◆□■△▲▽▼※〒→←↑↓〓
%
%
\def\LD@Yomigana@Preprocessing@Default{{%
\for\LD@Letter:=A{A,エー,えー,ええ,えい},B{B,ビー,びー,びい,び,ベー},C{C,シー,しー,しい},D{D,ディー,でぃー,でぃ},E{E,イー,えー,えい},F{F,エフ,えふ},G{G,ジー},H{H,エッチ,ハー},%
I{I,アイ,あい},J{J,ジェー,ジェイ,じぇい},K{K,ケー,けい},L{L,エル,える},M{M,エム,えむ},N{N,エヌ,えぬ},O{O,オー},P{P,ピー,ぴー},Q{Q,キュー,きゅう,きゅぜ},R{R,アール,あーる},%
S{S,エス,えす},T{T,ティー,てぃー,てぃぜ,テー},U{U,ユー},V{V,ブイ},W{W},X{X,エックス},Y{Y,わい},Z{Z},a{a,エー,ええ},b{b,ビー,びい},c{c,シー,しい},d{d,ディー},e{e,イー},%
f{f,エフ},g{g,ジー},h{h,エッチ},※{こめじるし},〒{ゆうびんきごう},%
i{i,アイ},j{j},k{k},l{l,エル},m{m,エム},n{n,エヌ},o{o,オー},p{p,ピー,ペー},q{q,キュー},r{r,アール},s{s,エス},t{t,ティー},u{u,ユー},v{v,ブイ},w{w},x{x,エックス},y{y},z{z},%
0{0},1{1,ひゃく},2{2,に},3{3,さん},4{4,し},5{5,ご},6{6,ろく},7{7,しち},8{8,はち},9{9,きゅう},%
+{プラス},α{アルファ},Α {アルファ},β {ベータ},Β{ベータ},Σ{シグマ},Ε{イプシロン,エプシロン},σ{シグマ},λ {ラムダ},Λ{ラムダ},ω{オメガ},Ζ{ゼータ},Ω{オメガ},Η{イータ,エータ},η{イータ},Π{パイ},θ{シータ},%
ε{エプシロン,イプシロン},ζ{ゼータ},η{エータ},θ{テータ,シータ},Θ{シータ,テータ},ι{イオタ},Ι{イオタ},Μ{ミュー},Ν{ニュー},Ο{オミクロン},Ξ{クサイ,クシー,グザイ},Ρ{ロー},Τ{タウ},Υ{ウプシロン,ユプシロン},υ{ウプシロン},%
Φ{ファイ,フィー},φ{ファイ},Χ{カイ,キー},χ{カイ},Ψ{プサイ,プシー},ψ{プサイ},×{ばつ,ペケ},%
κ{カッパー,カッパ},Κ{カッパ,カッパー},μ{ミュー},ν{ニュー},ο{オミクロン},ξ{グザイ,クサイ,クシー},π{パイ},ρ{ロー},τ{タウ},υ{ユプシロン},φ{フィー},χ{キー},ψ{プシー},○{まる},〇{まる},→{やじるし},←{やじるし},%
@{アットマーク,ナルト},.{.}\do{%
	\EA\LD@Edef@Two\LD@Letter\LD@Temp@A\LD@Temp@B
	\EA\xdef \CS LD@KanjiData@\LD@Temp@A YY\EC{,\LD@Temp@B ,}%
	\EA\xdef \CS LD@KanjiData@\LD@Temp@A NY\EC{,\LD@Temp@B ,}%
	\EA\xdef \CS LD@KanjiData@\LD@Temp@A YN\EC{,\LD@Temp@B ,}%
	\EA\xdef \CS LD@KanjiData@\LD@Temp@A NN\EC{,\LD@Temp@B ,}%
	\directlua1{
		LD_Temp_A="\LD@Temp@A"
		LD_Kanji_Data[LD_Temp_A.."YY"]=",\LD@Temp@B ,"
		LD_Kanji_Data[LD_Temp_A.."NY"]=",\LD@Temp@B ,"
		LD_Kanji_Data[LD_Temp_A.."YN"]=",\LD@Temp@B ,"
		LD_Kanji_Data[LD_Temp_A.."NN"]=",\LD@Temp@B ,"		
	}%
}%
\def\LD@KanjiData@Tail{ぁあぃいぅうぇえぉおかがきぎくぐけげこごさざしじすずせぜそぞただちぢっつづてでとどなにぬねのはばぱひびぴふぶぷへべぺほぼぽまみむめもゃやゅゆょよらりるれろゎわゐゑをんゔかけ{}{}{}{}}%
	\ForEach\LD@Letter:=ァアィイゥウェエォオカガキギクグケゲコゴサザシジスズセゼソゾタダチヂッツヅテデトドナニヌネノハバパヒビピフブプヘベペホボポマミムメモャヤュユョヨラリルレロヮワヰヱヲンヴヵヶヷヸヹヺ\do{%
		\LD@Split\LD@KanjiData@Tail{}\LD@KanjiData@Head\LD@KanjiData@Tail
		\unless\ifx\LD@KanjiData@Head\LD@Empty
			\edef\LD@Temp{\LD@KanjiData@Head}%
			% This Takes care of Hiragana
%			\EA\xdef \CS LD@KanjiData@\LD@Temp YY\EC{,\LD@Temp ,}%
%			\EA\xdef \CS LD@KanjiData@\LD@Temp NY\EC{,\LD@Temp ,}%
%			\EA\xdef \CS LD@KanjiData@\LD@Temp YN\EC{,\LD@Temp ,}%
%			\EA\xdef \CS LD@KanjiData@\LD@Temp NN\EC{,\LD@Temp ,}%
			% This takes care of Katakana with hiragana reading
			\EA\xdef \CS LD@KanjiData@\LD@Letter YY\EC{,\LD@Letter ,\LD@Temp ,}%
			\EA\xdef \CS LD@KanjiData@\LD@Letter NY\EC{,\LD@Letter ,\LD@Temp ,}%
			\EA\xdef \CS LD@KanjiData@\LD@Letter YN\EC{,\LD@Letter ,\LD@Temp ,}%
			\EA\xdef \CS LD@KanjiData@\LD@Letter NN\EC{,\LD@Letter ,\LD@Temp ,}%
			\directlua1{
				LD_Letter="\LD@Letter"
				LD_Kanji_Data[LD_Letter.."YY"]=",\LD@Letter ,\LD@Temp ,"	
				LD_Kanji_Data[LD_Letter.."NY"]=",\LD@Letter ,\LD@Temp ,"	
				LD_Kanji_Data[LD_Letter.."YN"]=",\LD@Letter ,\LD@Temp ,"	
				LD_Kanji_Data[LD_Letter.."NN"]=",\LD@Letter ,\LD@Temp ,"	
			}%
		\else
			% This takes care of Katakana without hiragana reading
			\EA\xdef \CS LD@KanjiData@\LD@Letter YY\EC{,\LD@Temp ,}%
			\EA\xdef \CS LD@KanjiData@\LD@Letter NY\EC{,\LD@Temp ,}%
			\EA\xdef \CS LD@KanjiData@\LD@Letter YN\EC{,\LD@Temp ,}%
			\EA\xdef \CS LD@KanjiData@\LD@Letter NN\EC{,\LD@Temp ,}%
			\directlua1{
				LD_Letter="\LD@Letter"
				LD_Kanji_Data[LD_Letter.."YY"]=",\LD@Temp ,"	
				LD_Kanji_Data[LD_Letter.."NY"]=",\LD@Temp ,"	
				LD_Kanji_Data[LD_Letter.."YN"]=",\LD@Temp ,"	
				LD_Kanji_Data[LD_Letter.."NN"]=",\LD@Temp ,"	
			}%
		\fi
		%\EA\EA\EA\AddForeignKanji \EA\LD@Letter\EA{\LD@Temp}%
	}%
	%% ケ gets special treatment, 〆 too
	\EA\xdef \CS LD@KanjiData@ケYY\EC{,ケ,け,げ,か,が,}%
	\EA\xdef \CS LD@KanjiData@ケNY\EC{,ケ,け,か,}%
	\EA\xdef \CS LD@KanjiData@ケYN\EC{,ケ,け,げ,か,が,}%
	\EA\xdef \CS LD@KanjiData@ケNN\EC{,ケ,け,か,}%
	\EA\xdef \CS LD@KanjiData@〆YY\EC{,しめ,じめ,}%
	\EA\xdef \CS LD@KanjiData@〆NY\EC{,しめ,}%
	\EA\xdef \CS LD@KanjiData@〆YN\EC{,しめ,じめ,}%
	\EA\xdef \CS LD@KanjiData@〆NN\EC{,しめ,}%
	\directlua1{
		LD_Kanji_Data["ケYY"] = ",ケ,け,げ,か,が,"	
		LD_Kanji_Data["ケNY"] = ",ケ,け,か,"	
		LD_Kanji_Data["ケYN"] = ",ケ,け,げ,か,が,"	
		LD_Kanji_Data["ケNN"] = ",ケ,け,か,"	
		LD_Kanji_Data["〆YY"] = ",しめ,じめ,"	
		LD_Kanji_Data["〆NY"] = ",しめ,"	
		LD_Kanji_Data["〆YN"] = ",しめ,じめ,"	
		LD_Kanji_Data["〆NN"] = ",しめ,"	
		LD_Kanji_Data["〆NN"] = ",しめ,"
		LD_Yomigana_KurikaeshiSymbols = "々ヽヾゝゞ"
	}%
	\gdef\LD@Yomigana@KurikaeshiSymbols{々ヽヾゝゞ}%
}}%
\def\LD@Yomigana@Preprocessing@JMnedict{%
	\EA\xdef \CS LD@KanjiData@ゐYY\EC{,ゐ,い,}%
	\EA\xdef \CS LD@KanjiData@ゐNY\EC{,ゐ,い,}%
	\EA\xdef \CS LD@KanjiData@ゐYN\EC{,ゐ,い,}%
	\EA\xdef \CS LD@KanjiData@ゐNN\EC{,ゐ,い,}%
	\EA\xdef \CS LD@KanjiData@ゑYY\EC{,ゑ,え,}%
	\EA\xdef \CS LD@KanjiData@ゑNY\EC{,ゑ,え,}%
	\EA\xdef \CS LD@KanjiData@ゑYN\EC{,ゑ,え,}%
	\EA\xdef \CS LD@KanjiData@ゑNN\EC{,ゑ,え,}%
	\EA\xdef \CS LD@KanjiData@をYY\EC{,を,お,}%
	\EA\xdef \CS LD@KanjiData@をNY\EC{,を,お,}%
	\EA\xdef \CS LD@KanjiData@をYN\EC{,を,お,}%
	\EA\xdef \CS LD@KanjiData@をNN\EC{,を,お,}%
	\EA\xdef \CS LD@KanjiData@はYY\EC{,は,ば,ぱ,わ,}%
	\EA\xdef \CS LD@KanjiData@はNY\EC{,は,わ,}%
	\EA\xdef \CS LD@KanjiData@はYN\EC{,は,ば,ぱ,わ,}%
	\EA\xdef \CS LD@KanjiData@はNN\EC{,は,わ,}%
	\EA\xdef \CS LD@KanjiData@ほYY\EC{,ほ,ぼ,ぽ,お,}%
	\EA\xdef \CS LD@KanjiData@ほNY\EC{,ほ,お,}%
	\EA\xdef \CS LD@KanjiData@ほYN\EC{,ほ,ぼ,ぽ,お,}%
	\EA\xdef \CS LD@KanjiData@ほNN\EC{,ほ,お,}%
	\EA\xdef \CS LD@KanjiData@へYY\EC{,へ,べ,ぺ,え,}%
	\EA\xdef \CS LD@KanjiData@へNY\EC{,へ,え,}%
	\EA\xdef \CS LD@KanjiData@へYN\EC{,へ,べ,ぺ,え,}%
	\EA\xdef \CS LD@KanjiData@へNN\EC{,へ,え,}%
	\EA\xdef \CS LD@KanjiData@ひYY\EC{,ひ,び,ぴ,い,}%
	\EA\xdef \CS LD@KanjiData@ひNY\EC{,ひ,い,}%
	\EA\xdef \CS LD@KanjiData@ひYN\EC{,ひ,び,ぴ,い,}%
	\EA\xdef \CS LD@KanjiData@ひNN\EC{,ひ,い,}%
	\EA\xdef \CS LD@KanjiData@ふYY\EC{,ふ,ぶ,ぷ,う,}%
	\EA\xdef \CS LD@KanjiData@ふNY\EC{,ふ,ぶ,ぷ,う,}%
	\EA\xdef \CS LD@KanjiData@ふYN\EC{,ふ,ぶ,ぷ,う,}%
	\EA\xdef \CS LD@KanjiData@ふNN\EC{,ふ,ぶ,ぷ,う,}%
	\directlua1{
		LD_Letter="\LD@Letter"
		LD_Kanji_Data["ゐYY"]=",ゐ,い,"	
		LD_Kanji_Data["ゐNY"]=",ゐ,い,"	
		LD_Kanji_Data["ゐYN"]=",ゐ,い,"	
		LD_Kanji_Data["ゐNN"]=",ゐ,い,"	
		LD_Kanji_Data["ゑYY"]=",ゑ,え,"	
		LD_Kanji_Data["ゑNY"]=",ゑ,え,"	
		LD_Kanji_Data["ゑYN"]=",ゑ,え,"	
		LD_Kanji_Data["ゑNN"]=",ゑ,え,"	
		LD_Kanji_Data["をYY"]=",を,お,"	
		LD_Kanji_Data["をNY"]=",を,お,"	
		LD_Kanji_Data["をYN"]=",を,お,"	
		LD_Kanji_Data["をNN"]=",を,お,"	
		LD_Kanji_Data["はYY"]=",は,ば,ぱ,わ,"	
		LD_Kanji_Data["はNY"]=",は,わ,"	
		LD_Kanji_Data["はYN"]=",は,ば,ぱ,わ,"	
		LD_Kanji_Data["はNN"]=",は,わ,"	
		LD_Kanji_Data["ほYY"]=",ほ,ぼ,ぽ,お,"	
		LD_Kanji_Data["ほNY"]=",ほ,お,"	
		LD_Kanji_Data["ほYN"]=",ほ,ぼ,ぽ,お,"	
		LD_Kanji_Data["ほNN"]=",ほ,お,"	
		LD_Kanji_Data["へYY"]=",へ,べ,ぺ,え,"	
		LD_Kanji_Data["へNY"]=",へ,え,"	
		LD_Kanji_Data["へYN"]=",へ,べ,ぺ,え,"	
		LD_Kanji_Data["へNN"]=",へ,え,"	
		LD_Kanji_Data["ひYY"]=",ひ,び,ぴ,い,"	
		LD_Kanji_Data["ひNY"]=",ひ,い,"	
		LD_Kanji_Data["ひYN"]=",ひ,び,ぴ,い,"	
		LD_Kanji_Data["ひNN"]=",ひ,い,"	
		LD_Kanji_Data["ふYY"]=",ふ,ぶ,ぷ,う,"	
		LD_Kanji_Data["ふNY"]=",ふ,ぶ,ぷ,う,"	
		LD_Kanji_Data["ふYN"]=",ふ,ぶ,ぷ,う,"	
		LD_Kanji_Data["ふNN"]=",ふ,ぶ,ぷ,う,"	
	}%
}%
%%%%%%%%%%%%%%%%%%%%%%%%%%%%%%%%%%%%%%%%%%%%%%%%%%%%%
%												%
%					Regular Compounds					%
%												%
%%%%%%%%%%%%%%%%%%%%%%%%%%%%%%%%%%%%%%%%%%%%%%%%%%%%%
\gdef\LD@Yomigana@Compounds{%
	|今日,きょう|眼鏡,めがね|部屋,へや|何時,いつ|如何,どう|悪戯,いたずら|為替,かわせ|かたぎ|相撲,すもう|大和,やまと|相撲,ずもう|土産,みやげ|海苔,のり|蝦夷,えぞ|海豚,いるか|自棄,やけ|団扇,うちわ|
	|雲雀,ひばり|飛蝗,ばった|酸漿,ほおずき|似非,えせ|上手,うま|山羊,やぎ|山葵,わさび|時雨,しぐれ|従兄弟,いとこ|従姉妹,いとこ|小豆,あずき|梅雨,つゆ|博士,はかせ|烏賊,いか|何方,どちら|何方,どっち|
	|何故,なぜ|昨日,きのう|一昨日,おととい|可愛,かわい|
}%
\directlua1{
		LD_Yomigana_Compounds = [[|今日,きょう|眼鏡,めがね|部屋,へや|何時,いつ|如何,どう|悪戯,いたずら|為替,かわせ|かたぎ|相撲,すもう|大和,やまと|相撲,ずもう|土産,みやげ|海苔,のり|蝦夷,えぞ|海豚,いるか|自棄,やけ|団扇,うちわ|
		|雲雀,ひばり|飛蝗,ばった|酸漿,ほおずき|似非,えせ|上手,うま|山羊,やぎ|山葵,わさび|時雨,しぐれ|従兄弟,いとこ|従姉妹,いとこ|小豆,あずき|梅雨,つゆ|博士,はかせ|烏賊,いか|何方,どちら|何方,どっち|
		|何故,なぜ|昨日,きのう|一昨日,おととい|可愛,かわい|]]
	}%
%%%%%%%%%%%%%%%%%%%%%%%%%%%%%%%%%%%%%%%%%%%%%%%%%%%%%%%%%%%%%%
%														%
%					Yomigana : Compute 							%
%														%
%%%%%%%%%%%%%%%%%%%%%%%%%%%%%%%%%%%%%%%%%%%%%%%%%%%%%%%%%%%%%%

\def\LD@Yomigana#1#2{% initializes things and starts the recursive process
	\edef\LD@Yomigana@L@Kanji{#1}%
	\LD@Yomigana@Kanji@Scan\LD@Yomigana@L@Kanji
	\edef\LD@Yomigana@L@Kana{#2}%
	\LD@Yomigana@Kana@Scan\LD@Yomigana@L@Kana
	\global\def\LD@Yomigana@GlobalFurigana{}%
	\LD@Yomigana@Compute
	% Improved support for repetitions in Kanji String
%	\ifx\LD@Yomigana@Scan@String\LD@Yomigana@L@Kanji
%		\LD@Yomigana@Compute
%	\else
%		\let\LD@Yomigana@L@Kanji\LD@Yomigana@Scan@String% deactivated for performance concerns
%		\LD@Yomigana@Compute
%		\ifnum\LD@Yomigana@Quirks@G@Level>9\relax
%			\let\LD@Yomigana@Scan@Furigana@Saved\LD@Yomigana@GlobalFurigana
%			\let\LD@Yomigana@Scan@Quirks@Saved\LD@Yomigana@Quirks@G@Level
%			\edef\LD@Yomigana@L@Kanji{#1}%
%			\edef\LD@Yomigana@L@Kana{#2}%
%			\LD@Yomigana@Compute
%			\unless\ifnum\LD@Yomigana@Quirks@G@Level<\LD@Yomigana@Scan@Quirks@Saved\relax
%				\global\let\LD@Yomigana@GlobalFurigana\LD@Yomigana@Scan@Furigana@Saved
%				\global\let\LD@Yomigana@Quirks@G@Level\LD@Yomigana@Scan@Quirks@Saved
%			\fi
%		\fi
%	\fi
	\global\let\LD@Yomigana@Scan@Kurikaeshi@Saved\LD@Yomigana@Scan@Kurikaeshi
	\global\let\LD@Yomigana@Kana@Scan@Katakana@Saved\LD@Yomigana@Kana@Scan@Katakana
	\edef\LD@Temp{\LD@Yomigana@GlobalFurigana end|}%
	\LD@Split\LD@Temp{|end|}\LD@Temp\LD@Trash
	\global\let\LD@Yomigana@GlobalFurigana\LD@Temp
}%



\def\LD@Yomigana@Kanji@Kana@Reading@Test@N{%
	\edef\LD@Yomigana@Test{\CS LD@KanjiData@\LD@Yomigana@L@Kanji\LD@Yomigana@L@Start\LD@Yomigana@L@End\EC \LD@Yomigana@L@Kana,}%
	\EA\LD@Split\EA\LD@Yomigana@Test\EA{\EA,\LD@Yomigana@L@Kana,}\LD@Temp@Trash\LD@Yomigana@Test
}%
\def\LD@Yomigana@Kanji@Kana@Reading@Test@Y{%
	\edef\LD@Yomigana@Test{\CS LD@KanjiData@\LD@Yomigana@L@Kanji\LD@Yomigana@L@Start\LD@Yomigana@L@End\EC \LD@Yomigana@L@Kana,}%
	\EA\LD@Split\EA\LD@Yomigana@Test\EA{\EA,\LD@Yomigana@L@Kana,}\LD@Temp@Trash\LD@Yomigana@Test
	\ifx\LD@Yomigana@Test\LD@Empty
		\LD@String@Hiraganize\LD@Yomigana@L@Kana\LD@Test
		\edef\LD@Yomigana@Test{\CS LD@KanjiData@\LD@Yomigana@L@Kanji\LD@Yomigana@L@Start\LD@Yomigana@L@End\EC \LD@Test,}%
		\EA\LD@Split\EA\LD@Yomigana@Test\EA{\EA,\LD@Test,}\LD@Temp@Trash\LD@Yomigana@Test
	\fi
}%



\def\LD@Yomigana@Kurikaeshi@Kana@Process{%
% if it fails try the reading
	\LD@Split\LD@Yomigana@Kurikaeshi@G@Kana{}\LD@Yomigana@TokenA\LD@Yomigana@Temp
	\LD@Split\LD@Yomigana@L@Kana{}\LD@Yomigana@TokenB\LD@Yomigana@Trash
	\ifx\LD@Yomigana@Temp\LD@Yomigana@Trash
		\edef\LD@Yomigana@Temp{\LD@KanjiData@Start |\LD@Yomigana@TokenA}%
		\EA\LD@Split\EA\LD@Yomigana@Temp\EA{\EA |\LD@Yomigana@TokenA}\LD@Yomigana@Trash\LD@Yomigana@Temp
		\unless\ifx\LD@Yomigana@Temp\LD@Empty
			\LD@Split\LD@Yomigana@Temp |\LD@Yomigana@Temp\LD@Yomigana@Trash
		\fi
		\edef\LD@Yomigana@Temp{\LD@Yomigana@TokenA\LD@Yomigana@Temp \LD@Yomigana@TokenB}%
		\EA\LD@Split\EA\LD@Yomigana@Temp\EA{\LD@Yomigana@TokenB}\LD@Yomigana@Trash\LD@Yomigana@Temp
		\ifx\LD@Yomigana@Temp\LD@Empty
			\LD@Yomigana@QuirkCase
		\else
			\ifx\LD@Yomigana@Kurikaeshi@G@Mode\LD@Empty
				\global\advance\LD@Yomigana@Quirks@G@Level by-9\relax
			\else
				\global\advance\LD@Yomigana@Quirks@G@Level by-10\relax
			\fi
			\LD@Yomigana@ValidCase
		\fi
	\else
		\LD@Yomigana@QuirkCase
	\fi
}%
\def\LD@Yomigana@Unknown{%
	\global\advance\LD@Yomigana@Quirks@G@Level by9\relax
   	\ifx\LD@Yomigana@L@Kana\LD@Empty
		\LD@Yomigana@QuirkCase
	\else
		\edef\LD@Yomigana@Temp{\LD@Yomigana@KurikaeshiSymbols\LD@Yomigana@L@Kanji}%
		\EA\LD@Split\EA\LD@Yomigana@Temp\EA{\LD@Yomigana@L@Kanji}\LD@Yomigana@Trash\LD@Yomigana@Temp
		\unless\ifx\LD@Yomigana@Temp\LD@Empty
			\LD@Yomigana@Kurikaeshi@Support
		\else
			\EA\LD@Counter@Temp@A\EA=\EA`\LD@Yomigana@L@Kanji\relax
			\ifnum\LD@Counter@Temp@A<12436\relax
				\ifnum\LD@Counter@Temp@A>12352\relax
					% Hiragana
					\advance\LD@Counter@Temp@A by96\relax
					\EA\ifnum\EA\LD@Counter@A\EA=\EA`\LD@Yomigana@L@Kana\relax
						% With Katakana Reading
						\global\advance\LD@Yomigana@Quirks@G@Level by-9\relax
						\LD@Yomigana@ValidCase
					\else
						\LD@Yomigana@QuirkCase
					\fi
				\else
					\LD@Yomigana@QuirkCase
				\fi
			\else
				\LD@Yomigana@QuirkCase
			\fi
		\fi
	\fi
	\gdef\LD@Yomigana@Kurikaeshi@G@Kanji{}% it is unknown....
}%



%%%%%%%%%%%%%%%%%%%%%%%%%%%%%%%%%%%%%%%%%%%%%%%%%%%%%%%%%%%%%%%%%%%%%%%%%%%%%%%%%%%%%%%%%%%%%%%%%%%%%%%%%%%%%%%%%%%%%%
%%%%%%%%                                                                                                            Stats 
%%%%%%%%%%%%%%%%%%%%%%%%%%%%%%%%%%%%%%%%%%%%%%%%%%%%%%%%%%%%%%%%%%%%%%%%%%%%%%%%%%%%%%%%%%%%%%%%%%%%%%%%%%%%%%%%%%%%%%
\def\JLPTKanji#1{%
\begingroup
	\LD@Counter@A=#1\relax
	\advance\LD@Counter@A by1\relax
	\ifcase\LD@Counter@A
		{\Kaisho6片仮名}\or
		{\Kaisho6平仮名}\or
		JLPT 4\or
		JLPT 3\or
		JLPT 2\or
		JLPT 1\or
		NF\or
		Autre%
	\fi
\endgroup
}%
\def\GradeKanji#1{%
\begingroup
	\LD@Counter@A=#1\relax
	\advance\LD@Counter@A by1\relax
	\ifcase\LD@Counter@A
	 	{\Kaisho6片仮名}\or
		{\Kaisho6平仮名}\or
		CP\or
		CE1\or
		CE2\or
		CM1\or
		CM2\or
		6ème\or
		College\or
		Nom\or
		Autre\or
	\fi
\endgroup
}%
\def\JLPTTango#1{%
\begingroup
	\LD@Counter@A=#1\relax
	\ifcase\LD@Counter@A
	 	\or
		JLPT 4\or
		JLPT 3\or
		JLPT 2\or
		JLPT 1\or
		NF\or
		Autre\or
	\fi
\endgroup
}%
\def\LD@Stats{%
%\immediate\write16{Displaying Stats...}%
\begingroup
	\def\JLPTSwap##1{%
		\LD@Counter@A=##1\relax
		\global\advance\LD@Counter@A by1\relax
		\ifcase\LD@Counter@A
			\def##1{-1}\or
			\def##1{0}\or
			\def##1{4}\or
			\def##1{3}\or
			\def##1{2}\or
			\def##1{1}\or
		\fi
	}%
	\PAR
	\centerline{\ArialB{20}Statistiques}
	\medskip
	{\Arial 8Tableau representant pour chaque niveau de vocabulaire, le nombre de mots {\Kaisho8(単語)} que l'on peut écrire en utilisant les caractères {\Kaisho8(漢字)} d'un niveau inférieur à un seuil fixé.}
	\medskip
	\centerline{%
		\tikzpicture
			\foreach \x in{1,2,...,6}
				\foreach \y in{-1,0,...,6}
					\node[rectangle,draw,text width=1cm,text height=10pt,text depth=2pt,inner sep=0pt,text centered] at (\x*1cm,-\y*12pt) {%
					\let\LDX=\x\JLPTSwap\LDX\LD@Stat@Display{\LDX|\y}};
               		\foreach \x in{1,2,...,6}
				\node[rectangle, text width=1cm,text height=10pt,text depth=2pt,inner sep=0pt, text centered] (a\x)at (\x*1cm,24pt) {\Arial6\JLPTTango{\x}};
			\draw [snake=brace] (a1.north west) -- (a6.north east) node[midway](G){};
			\node [above of=G, node distance=8pt]  {\Kaisho{10}単語};
			\foreach \y in{-1,0,...,6}
			\node[rectangle, text width=0.8cm,text height=10pt,text depth=2pt,inner sep=0pt] at (0cm,-\y*12pt) (b\y) {\noindent\hfill\Arial6\JLPTKanji{\y}\hskip4pt};
			\draw [snake=brace] (b6.south west) -- (b-1.north west) node[midway](H){};
			\node (J) at (H.west) {};\node [left of=J,node distance=4pt](K){};
			\node[above of=K,node distance=5pt] {\Kaisho{10}漢};
			\node[below of=K,node distance=5pt] {\Kaisho{10}字};
		\endtikzpicture
	}%
	\centerline{%
		\tikzpicture
			\foreach \x in{1,2,...,6}
				\foreach \y in{-1,0}
					\node[rectangle,draw, text width=1cm,text height=10pt,text depth=2pt,inner sep=0pt, text centered] at (\x*1cm,-\y*12pt) {%
					\let\LDX=\x\JLPTSwap\LDX\LD@Stat@Display{\LDX|\y}};
			\foreach \x in{1,2,...,6}
				\foreach \y in{1,2,...,9}
					\node[rectangle,draw, text width=1cm,text height=10pt,text depth=2pt,inner sep=0pt, text centered] at (\x*1cm,-\y*12pt) {%
					\let\LDX=\x\JLPTSwap\LDX\LD@Stat@Display{\LDX|G\y}};
               		\foreach \x in{1,2,...,6}
				\node[rectangle, text width=1cm,text height=10pt,text depth=2pt,inner sep=0pt, text centered] (a\x)at (\x*1cm,24pt) {\Arial6\JLPTTango{\x}};
			\draw [snake=brace] (a1.north west) -- (a6.north east) node[midway](G){};
			\node [above of=G, node distance=8pt]  {\Kaisho{10}単語};
			\foreach \y in{-1,0,...,9}
			\node[rectangle, text width=0.9cm,text height=10pt,text depth=2pt,inner sep=0pt] at (0cm,-\y*12pt) (b\y) {\noindent\hfill\Arial6\GradeKanji{\y}\hskip4pt};
			\draw [snake=brace] (b9.south west) -- (b-1.north west) node[midway](H){};
			\node (J) at (H.west) {};\node [left of=J,node distance=4pt](K){};
			\node[above of=K,node distance=5pt] {\Kaisho{10}漢};
			\node[below of=K,node distance=5pt] {\Kaisho{10}字};
		\endtikzpicture
	}%
	\endgroup
}%
%%%%%%%%%%%%%%%%%%%%%%%%%%%%%%%%%%%%%%%%%%%%%%%%%%%%%
%												%
%					Language choice					%
%												%
%%%%%%%%%%%%%%%%%%%%%%%%%%%%%%%%%%%%%%%%%%%%%%%%%%%%%

\def\LD@Akuma@Language{fr}%

%%%%%%%%%%%%%%%%%%%%%%%%%%%%%%%%%%%%%%%%%%%%%%%%%%%%%
%												%
%					Loading Kanjidic					%
%												%
%%%%%%%%%%%%%%%%%%%%%%%%%%%%%%%%%%%%%%%%%%%%%%%%%%%%%

\LD@Kanjidic@Load

%%%%%%%%%%%%%%%%%%%%%%%%%%%%%%%%%%%%%%%%%%%%%%%%%%%%%
%												%
%					Kanjidic Preprocessing				%
%												%
%%%%%%%%%%%%%%%%%%%%%%%%%%%%%%%%%%%%%%%%%%%%%%%%%%%%%

\LD@Yomigana@Kanji@Preprocessing
\LD@Yomigana@Preprocessing@Default




\directlua1{
	
	function oltest(a,b)
		if b==nil then
			tex.print("nil")
		else
			tex.print(a..b)
		end
	end
	
	function LD_Yomigana_Compute(LD_Yomigana_L_Kanji, LD_Yomigana_L_Kana) 
		% initializes things and starts the recursive process
		% QuirkMode  : 0 (regular) 1 (Null quirk) 2 (non Null quirk)
%%%		LD_Yomigana_Kurikaeshi_G_Kana="" % global
%%%		LD_Yomigana_Kurikaeshi_G_Kanji="" % global	
%%%		LD_Yomigana_Kurikaeshi_G_Mode="" % global	
%		LD_Yomigana_L_Start="Y" %  : First ? Y or N   %		LD_Yomigana_L_End="Y" % : Last ? Y or N
		local LD_Yomigana_GlobalFurigana, LD_Yomigana_Quirks_G_Level, LD_Yomigana_Quirks_G_Mode = LD_Yomigana_Process(0 , 0, LD_Yomigana_L_Kanji, LD_Yomigana_L_Kana, unicode.utf8.len(LD_Yomigana_L_Kanji), unicode.utf8.len(LD_Yomigana_L_Kana), "Y", "Y") 
		 return  LD_Yomigana_GlobalFurigana, LD_Yomigana_Quirks_G_Level
	end

	function LD_Yomigana_Process(LD_Yomigana_Quirks_G_Level, LD_Yomigana_Quirks_G_Mode, LD_Yomigana_L_Kanji, LD_Yomigana_L_Kana, LD_Yomigana_Counter_L_Kanji, LD_Yomigana_Counter_L_Kana, LD_Yomigana_L_Start, LD_Yomigana_L_End)
		local LD_Yomigana_GlobalFurigana
		% The Lakedaemonian Recursive Function implementing a kind of Dichotomy.  
		if LD_Yomigana_Counter_L_Kanji == 1 then
			% one Kanji
			LD_Yomigana_GlobalFurigana, LD_Yomigana_Quirks_G_Level, LD_Yomigana_Quirks_G_Mode = LD_Yomigana_OneKanji(LD_Yomigana_Quirks_G_Level, LD_Yomigana_Quirks_G_Mode, LD_Yomigana_L_Kanji, LD_Yomigana_L_Kana, LD_Yomigana_L_Start, LD_Yomigana_L_End) 
		elseif  unicode.utf8.find(LD_Yomigana_Compounds, "|" .. LD_Yomigana_L_Kanji .. "," .. LD_Yomigana_L_Kana .. "|") == nil then
			% Split kanji string and Kana string 
			LD_Yomigana_GlobalFurigana, LD_Yomigana_Quirks_G_Level, LD_Yomigana_Quirks_G_Mode = LD_Yomigana_Dichotomy(LD_Yomigana_Quirks_G_Level, LD_Yomigana_Quirks_G_Mode, LD_Yomigana_L_Kanji, LD_Yomigana_L_Kana, LD_Yomigana_Counter_L_Kanji, LD_Yomigana_Counter_L_Kana, LD_Yomigana_L_Start, LD_Yomigana_L_End)
		else 
			% Support for Regular Compounds. Return stuff
			LD_Yomigana_GlobalFurigana, LD_Yomigana_Quirks_G_Mode = unicode.utf8.rep("*|", LD_Yomigana_Counter_L_Kanji-1) .. "*" .. 	LD_Yomigana_L_Kana .. "|", 0
%			LD_Yomigana_Kurikaeshi_G_Kana = ""
%			LD_Yomigana_Kurikaeshi_G_Kanji = ""
%			LD_Yomigana_Kurikaeshi_G_Mode = ""										
		end
		return LD_Yomigana_GlobalFurigana,LD_Yomigana_Quirks_G_Level,LD_Yomigana_Quirks_G_Mode
	end
	
	function LD_Yomigana_OneKanji(LD_Yomigana_Quirks_G_Level, LD_Yomigana_Quirks_G_Mode, LD_Yomigana_L_Kanji, LD_Yomigana_L_Kana, LD_Yomigana_L_Start, LD_Yomigana_L_End)
		local LD_Yomigana_GlobalFurigana
		%
		% One Kanji case. 
		% This case does the Grunt Work (it compares with the lectures) and happens a lot. 	
		%
		local LD_Yomigana_Test = LD_Kanji_Data[LD_Yomigana_L_Kanji .. LD_Yomigana_L_Start .. LD_Yomigana_L_End] 
		if not(LD_Yomigana_Test == nil) then
			if unicode.utf8.find(LD_Yomigana_Test, "," .. LD_Yomigana_L_Kana .. "," ) == nil then
				% Quirk case
				LD_Yomigana_GlobalFurigana, LD_Yomigana_Quirks_G_Level, LD_Yomigana_Quirks_G_Mode = LD_Yomigana_QuirkCase( LD_Yomigana_Quirks_G_Level, LD_Yomigana_Quirks_G_Mode, LD_Yomigana_L_Kana) 
			else
				% Valid case
				LD_Yomigana_GlobalFurigana, LD_Yomigana_Quirks_G_Mode = LD_Yomigana_ValidCase(LD_Yomigana_L_Kana)
			end
%%%			LD_Yomigana_Kurikaeshi_G_Kanji=LD_Yomigana_L_Kanji % global
		else
			if LD_Yomigana_L_Kanji == LD_Yomigana_L_Kana then 
				% Valid case
				LD_Yomigana_GlobalFurigana, LD_Yomigana_Quirks_G_Mode = LD_Yomigana_ValidCase(LD_Yomigana_L_Kana)
%%%				LD_Yomigana_Kurikaeshi_G_Kanji=LD_Yomigana_L_Kanji % global usefull for kana
			else
				% Case Unknown
				LD_Yomigana_GlobalFurigana, LD_Yomigana_Quirks_G_Level, LD_Yomigana_Quirks_G_Mode = LD_Yomigana_Unknown(LD_Yomigana_Quirks_G_Level, LD_Yomigana_Quirks_G_Mode, LD_Yomigana_L_Kanji, LD_Yomigana_L_Kana)
			end
		end
		return LD_Yomigana_GlobalFurigana, LD_Yomigana_Quirks_G_Level, LD_Yomigana_Quirks_G_Mode
	end		
			
	function LD_Yomigana_ValidCase(LD_Yomigana_L_Kana)
		%LD_Yomigana_GlobalFurigana = LD_Yomigana_L_Kana .. "|"
		%LD_Yomigana_Quirks_G_Mode = 0
		return LD_Yomigana_L_Kana .. "|", 0
	end
		
	function LD_Yomigana_QuirkCase(LD_Yomigana_Quirks_G_Level, LD_Yomigana_Quirks_G_Mode, LD_Yomigana_L_Kana)
		if LD_Yomigana_Quirks_G_Mode < 2 then
			LD_Yomigana_Quirks_G_Level = LD_Yomigana_Quirks_G_Level + 1
		else
			% Forbidden case
			LD_Yomigana_Quirks_G_Level = LD_Yomigana_Quirks_G_Level + 666
		end
		if LD_Yomigana_L_Kana == "" then
			LD_Yomigana_Quirks_G_Mode = 1 
		else
			LD_Yomigana_Quirks_G_Mode = 2 
		end
		return "*" .. LD_Yomigana_L_Kana .. "|", LD_Yomigana_Quirks_G_Level, LD_Yomigana_Quirks_G_Mode
	end
	
	function LD_Yomigana_Unknown(LD_Yomigana_Quirks_G_Level, LD_Yomigana_Quirks_G_Mode, LD_Yomigana_L_Kanji, LD_Yomigana_L_Kana)
		local LD_Yomigana_GlobalFurigana
		LD_Yomigana_Quirks_G_Level = LD_Yomigana_Quirks_G_Level + 9
		if LD_Yomigana_L_Kana == "" then
			% Quirk case
			LD_Yomigana_GlobalFurigana, LD_Yomigana_Quirks_G_Level, LD_Yomigana_Quirks_G_Mode = LD_Yomigana_QuirkCase(LD_Yomigana_Quirks_G_Level, LD_Yomigana_Quirks_G_Mode, LD_Yomigana_L_Kana) 		
%		elseif not(unicode.utf8.find(LD_Yomigana_KurikaeshiSymbols,LD_Yomigana_L_Kanji)== nil) then
%			LD_Yomigana_Kurikaeshi_Support() for the time being
		else
			local Kanji_number = unicode.utf8.byte(LD_Yomigana_L_Kanji)
			if Kanji_number > 12352 and Kanji_number < 12436 and Kanji_number +96 == unicode.utf8.byte(LD_Yomigana_L_Kana) then 
				% "Valid" case hiragana with katakana reading
				LD_Yomigana_Quirks_G_Level = LD_Yomigana_Quirks_G_Level - 9
				LD_Yomigana_GlobalFurigana, LD_Yomigana_Quirks_G_Mode = LD_Yomigana_ValidCase(LD_Yomigana_L_Kana)
			else 
				% Quirk case
				LD_Yomigana_GlobalFurigana, LD_Yomigana_Quirks_G_Level, LD_Yomigana_Quirks_G_Mode = LD_Yomigana_QuirkCase(LD_Yomigana_Quirks_G_Level, LD_Yomigana_Quirks_G_Mode, LD_Yomigana_L_Kana) 
			end
		end	
%		LD_Yomigana_Kurikaeshi_G_Kanji="" % it is unknown....
		return LD_Yomigana_GlobalFurigana, LD_Yomigana_Quirks_G_Level, LD_Yomigana_Quirks_G_Mode
	end
	
	function LD_Yomigana_Dichotomy(LD_Yomigana_Quirks_G_Level, LD_Yomigana_Quirks_G_Mode, LD_Yomigana_L_Kanji, LD_Yomigana_L_Kana, LD_Yomigana_Counter_L_Kanji, LD_Yomigana_Counter_L_Kana, LD_Yomigana_L_Start, LD_Yomigana_L_End)		
		%%%%%%%%%%%%%%%%%%%%%%%%%%%%%%%%%%%%%%%%%%%%%%%%%%
		%
		%	-- First Compute \LD@Yomigana@Counter@S@KanjiA=[(T+1)/2] avec T =\LD@Yomigana@Counter@L@Kanji
		%
		local LD_Yomigana_Counter_S_KanjiA = math.floor ((LD_Yomigana_Counter_L_Kanji + 1) / 2)
		local LD_Yomigana_Counter_S_KanjiB = LD_Yomigana_Counter_L_Kanji-LD_Yomigana_Counter_S_KanjiA
		%
		%	-- now split the Tangostring with  the first [(T+1)/2] tokens in TangoA and the remains in TangoB
		%
		local LD_Yomigana_TangoA = unicode.utf8.sub(LD_Yomigana_L_Kanji, 1, LD_Yomigana_Counter_S_KanjiA)
		local LD_Yomigana_TangoB = unicode.utf8.sub(LD_Yomigana_L_Kanji, -LD_Yomigana_Counter_S_KanjiB,-1)
		%
		%	-- Do the same with Kanas : put the first [(K+1)/2] kanas in KanaA 
		%     -- and the remaining kanas in KanaB avec K =LD_Yomigana_Counter_L_Kana
		%
		local LD_Yomigana_Counter_S_KanaA = math.floor ((LD_Yomigana_Counter_L_Kana + 1) / 2)
		local LD_Yomigana_Counter_S_KanaB = LD_Yomigana_Counter_L_Kana - LD_Yomigana_Counter_S_KanaA
		local LD_Yomigana_Counter_S_KanaC = LD_Yomigana_Counter_S_KanaA
		local LD_Yomigana_Counter_S_KanaD = LD_Yomigana_Counter_S_KanaB
		%
		%	-- now split the Tangostring with  the first [(n+1)/2] tokens in KanaA and the remains in KanaB 
		%
		local LD_Yomigana_KanaA = unicode.utf8.sub(LD_Yomigana_L_Kana, 1, LD_Yomigana_Counter_S_KanaA)
		local LD_Yomigana_KanaB = unicode.utf8.sub(LD_Yomigana_L_Kana, -LD_Yomigana_Counter_S_KanaB,-1)
		local LD_Yomigana_KanaC = LD_Yomigana_KanaA
		local LD_Yomigana_KanaD = LD_Yomigana_KanaB
		%%%%%%%%%%%%%%%%%%%%%%%%%%%%%%%%%%%%%%%%%%%%%%%%%%%%%%%%%%%%%%%%%%%%%%%%%%%%%%%%%%%%	
		local LD_Yomigana_S_Furigana, LD_Yomigana_Quirks_S_Mode
		local LD_Yomigana_Quirks_S_Level = 667
		local LD_Yomigana_Counter_Loop = LD_Yomigana_Counter_L_Kana 
		% we could do the loop with a for statement but then we would have to break it
		repeat 
			local LD_Yomigana_L_FuriganaA, LD_Yomigana_L_FuriganaB, LD_Yomigana_Quirks_L_Level, LD_Yomigana_Quirks_L_Mode
			if 2 * math.floor(LD_Yomigana_Counter_Loop / 2) == LD_Yomigana_Counter_Loop then
				% Even walk
				LD_Yomigana_L_FuriganaA, LD_Yomigana_Quirks_L_Level, LD_Yomigana_Quirks_L_Mode = LD_Yomigana_Process(LD_Yomigana_Quirks_G_Level, LD_Yomigana_Quirks_G_Mode, LD_Yomigana_TangoA, LD_Yomigana_KanaA, LD_Yomigana_Counter_S_KanjiA, LD_Yomigana_Counter_S_KanaA, LD_Yomigana_L_Start, "N") 
				% Test if it is usefull to continue
				if LD_Yomigana_Quirks_L_Level < LD_Yomigana_Quirks_S_Level then 
					% we go for the first solution. Now bring the second Daemon to the fray
					LD_Yomigana_L_FuriganaB, LD_Yomigana_Quirks_L_Level, LD_Yomigana_Quirks_L_Mode = LD_Yomigana_Process(LD_Yomigana_Quirks_L_Level, LD_Yomigana_Quirks_L_Mode, LD_Yomigana_TangoB, LD_Yomigana_KanaB, LD_Yomigana_Counter_S_KanjiB, LD_Yomigana_Counter_S_KanaB, "N", LD_Yomigana_L_End) 
				end
				% Crabwalk in the Even case... \LD@Yomigana@KanaC grows \LD@Yomigana@KanaD shrinkens 
				LD_Yomigana_KanaC =  LD_Yomigana_KanaC .. unicode.utf8.sub(LD_Yomigana_KanaD, 1, 1) 
				LD_Yomigana_Counter_S_KanaC = LD_Yomigana_Counter_S_KanaC + 1
				LD_Yomigana_Counter_S_KanaD = LD_Yomigana_Counter_S_KanaD - 1
				if LD_Yomigana_Counter_S_KanaD == 0 then
					LD_Yomigana_KanaD = ""
				else
					LD_Yomigana_KanaD = unicode.utf8.sub(LD_Yomigana_KanaD, -LD_Yomigana_Counter_S_KanaD)
				end	
			else
				% Odd walk
				LD_Yomigana_L_FuriganaA, LD_Yomigana_Quirks_L_Level, LD_Yomigana_Quirks_L_Mode = LD_Yomigana_Process(LD_Yomigana_Quirks_G_Level, LD_Yomigana_Quirks_G_Mode, LD_Yomigana_TangoA, LD_Yomigana_KanaC, LD_Yomigana_Counter_S_KanjiA, LD_Yomigana_Counter_S_KanaC, LD_Yomigana_L_Start, "N") 
				% Test if it is usefull to continue
				if LD_Yomigana_Quirks_L_Level < LD_Yomigana_Quirks_S_Level then 
					% we go for the first solution, now bring the second Daemon to the fray
					LD_Yomigana_L_FuriganaB, LD_Yomigana_Quirks_L_Level, LD_Yomigana_Quirks_L_Mode = LD_Yomigana_Process(LD_Yomigana_Quirks_L_Level, LD_Yomigana_Quirks_L_Mode, LD_Yomigana_TangoB, LD_Yomigana_KanaD, LD_Yomigana_Counter_S_KanjiB, LD_Yomigana_Counter_S_KanaD, "N", LD_Yomigana_L_End) 
				end
				% Crabwalk in the odd case... \LD@Yomigana@KanaA Shrinkens \LD@Yomigana@KanaB grows 
				LD_Yomigana_KanaB = unicode.utf8.sub(LD_Yomigana_KanaA, -1) ..  LD_Yomigana_KanaB
				LD_Yomigana_Counter_S_KanaB = LD_Yomigana_Counter_S_KanaB + 1
				LD_Yomigana_Counter_S_KanaA = LD_Yomigana_Counter_S_KanaA - 1
				LD_Yomigana_KanaA = unicode.utf8.sub(LD_Yomigana_KanaA, 1, LD_Yomigana_Counter_S_KanaA)
			end
			%%% function LD_Yomigana_CrabWalk_Update()
			if LD_Yomigana_Quirks_L_Level < LD_Yomigana_Quirks_S_Level then
				% We have a winner here, save it's data
				LD_Yomigana_S_Furigana = LD_Yomigana_L_FuriganaA .. LD_Yomigana_L_FuriganaB
				LD_Yomigana_Quirks_S_Mode = LD_Yomigana_Quirks_L_Mode
				LD_Yomigana_Quirks_S_Level = LD_Yomigana_Quirks_L_Level
				if LD_Yomigana_Quirks_G_Level == LD_Yomigana_Quirks_L_Level then
					LD_Yomigana_Counter_Loop = 0 % break the loop : quirk level didn't increase !
				end
			end
			%%%%%%%%%%%%%%%%%%%%%%%%%%%%%%%%%%%%%%%
			LD_Yomigana_Counter_Loop = LD_Yomigana_Counter_Loop  - 1
		until LD_Yomigana_Counter_Loop < 0			
		%%% function LD_Yomigana_Dichotomy_Return()
		return LD_Yomigana_S_Furigana, LD_Yomigana_Quirks_S_Level, LD_Yomigana_Quirks_S_Mode
	end                                                    
	
%	LD_Yomigana_Compute("本","ほん")
%	LD_Yomigana_Compute("本","ん")
%	LD_Yomigana_Compute("日本","にほん")
%	LD_Yomigana_Compute("大丈夫","だいじょうぶ")
%	LD_Yomigana_Compute("運転手","うんてんしゅ")
%	LD_Yomigana_Compute("本部","ほんぶ")
%	LD_Yomigana_Compute("本","ん")
%	LD_Yomigana_Compute("日本","にん")
%	LD_Yomigana_Compute("大丈夫","だじょうぶ")
%	LD_Yomigana_Compute("運転手","だいじょうぶ")
%	LD_Yomigana_Compute("本部","ほん")

}%

\def\LD@Yomigana#1#2{% initializes things and starts the recursive process
	\directlua1{Furigana,Quirks_Level=LD_Yomigana_Compute("#1","#2")}
	\xdef\LD@Yomigana@GlobalFurigana{%
		\directlua1{tex.print(unicode.utf8.sub(Furigana,1,-2))}%
	}%
	\xdef\LD@Yomigana@Quirks@G@Level{%
		\directlua1{tex.print(Quirks_Level)}%
	}%
%
%	\global\let\LD@Yomigana@Scan@Kurikaeshi@Saved\LD@Yomigana@Scan@Kurikaeshi
%	\global\let\LD@Yomigana@Kana@Scan@Katakana@Saved\LD@Yomigana@Kana@Scan@Katakana
}%
%%%%%%%%%%%%%%%%%%%%%%%%%%%%%%%%%%%%%%%%%%%%%%%%%%%%%
%												%
%					Loading JMdict					%
%												%
%%%%%%%%%%%%%%%%%%%%%%%%%%%%%%%%%%%%%%%%%%%%%%%%%%%%%
\def\LD@Reading@Compound@List{}%
\newcount\LD@Stats@Counter
\def\LD@Stats@Yomigana@Process#1#2{{% #1 is the Kanji string and #2 is the Yomigana string
	\edef\LD@Stats@Tail{#1}%
	\def\LD@Reading@Compound{}% N
	\LD@Loop@For\LD@Stats@Loop@Member=#2\WithSeparator |\Do{%
		\LD@String@Split\LD@Stats@Tail{}\LD@Stats@Head\LD@Stats@Tail
		\LD@String@Split\LD@Stats@Loop@Member{}\LD@Token\LD@Test
		\if *\LD@Token
			\def\LD@Stats@Irregular{*}%
		\else
			\def\LD@Stats@Irregular{}%
			\let\LD@Test\LD@Stats@Loop@Member
		\fi
	\let\LD@Temp\LD@Stats@Loop@Member
		\ifx\LD@Test\LD@Empty
			\ifx\LD@Stats@Tail\LD@Empty
			\else
				% empty reading for Kanji....
				\edef\LD@Reading@Compound{\LD@Reading@Compound\LD@Stats@Head}% Y
			\fi
		\else
			\unless\ifx\LD@Reading@Compound\LD@Empty
				% Compound End
				\edef\LD@Reading@Compound{\LD@Reading@Compound\LD@Stats@Head}%
				\LD@Stat@Update{\LD@Reading@Compound ,\LD@Test}%
				\edef\LD@Temp{\LD@Reading@Compound ,\LD@Test |}%
				\edef\LD@Test{\LD@Reading@Compound@List |\LD@Temp}%
				\EA\LD@String@Split\EA\LD@Test\EA{\EA |\LD@Temp}\LD@Trash\LD@Test
				\ifx\LD@Test\LD@Empty
					\xdef\LD@Reading@Compound@List{\LD@Reading@Compound@List\LD@Temp}%
				\fi
				\def\LD@Reading@Compound{}% N
			\else
				% Kanji
				\ifcsname LD@Stats@@\LD@Stats@Head\endcsname
					\EA\let\EA\LD@Stats@CS\CS LD@Stats@@\LD@Stats@Head\EC
					\edef\LD@Test{\LD@Stats@CS,\LD@Temp,}%
					\EA\LD@String@Split\EA\LD@Test\EA{\EA,\LD@Temp,}\LD@Trash\LD@Test
					\unless\ifx\LD@Test\LD@Empty
						% already there...increment it
						\EA\LD@String@Split\EA\LD@Stats@CS\EA{\EA,\LD@Temp,}\LD@Temp@A\LD@Temp@B
						\LD@String@Split\LD@Temp@B ,\LD@Temp@B\LD@Temp@C
						\unless\ifx\LD@Compound@Root\LD@Empty
							\let\LD@Temp@B\LD@Compound@Root
						\fi
						\LD@String@Split\LD@Temp@C ,\LD@Temp@C\LD@Temp@D
						\LD@String@Split\LD@Temp@D |\LD@Number\LD@Temp@D
						\LD@Stats@Counter=\LD@Number\relax
						\advance\LD@Stats@Counter by1\relax
						\ifnum\LD@Stats@Counter>9\relax
							\EA\xdef\CS LD@Stats@@\LD@Stats@Head\EC{\LD@Temp@A,\LD@Temp,\LD@Temp@B,\LD@Temp@C,\the\LD@Stats@Counter|\LD@Temp@D}%
						\else
							\EA\xdef\CS LD@Stats@@\LD@Stats@Head\EC{\LD@Temp@A,\LD@Temp,\LD@Temp@B,\LD@Temp@C,0\the\LD@Stats@Counter|\LD@Temp@D}%
						\fi
					\else
						% first there
						\EA\xdef\CS LD@Stats@@\LD@Stats@Head\EC{\CS LD@Stats@@\LD@Stats@Head\EC,\LD@Temp,\LD@Compound@Root,\LD@Stats@Irregular,01| }%
					\fi
					\def\LD@Reading@Compound{}% N
				\else
					\EA\xdef\CS LD@Stats@@\LD@Stats@Head\EC{,\LD@Temp,\LD@Compound@Root,\LD@Stats@Irregular,01| }%
					\def\LD@Reading@Compound{}% N
				\fi
			\fi
		\fi
	}%
}}%
\def\LD@JMdict@Load@Stats{%
	\def\LD@XML@Entry@Kana@Do{%
		%\global\advance\LD@Entry@Quirks by1\relax
		\global\advance\LD@Entry@Count by1\relax
		{%
		\def\TangoMaxKanjiLevel{0}%
		\def\TangoMaxKanjiGrade{0}%
		\LD@String@Split\LD@Kana@Reading{}\LD@Token\LD@Trash
		\IfKatakana\LD@Token{%
			\def\TangoMaxKanjiLevel{-1}%
			\def\TangoMaxKanjiGrade{-1}%
		}%
		\LD@Akuma@Option@Process\LD@Kana@Info
		\ifx\LD@JLPT\LD@Empty
			\ifx\LD@nf\LD@Empty
				\LD@Stat@Update{6|\TangoMaxKanjiLevel}%
				\LD@Stat@Update{6|G\TangoMaxKanjiGrade}%
			\else
				\LD@Stat@Update{5|\TangoMaxKanjiLevel}%
				\LD@Stat@Update{5|G\TangoMaxKanjiGrade}%
			\fi
			\def\LD@JLPT{{}}%
		\else
			\LD@Stat@Update{\LD@JLPT|\TangoMaxKanjiLevel}%
			\LD@Stat@Update{\LD@JLPT|G\TangoMaxKanjiGrade}%
		\fi}%
	}%
	\def\LD@XML@Entry@Kanji@Do{{%	
		%\unless\ifx\LD@Entry@Sense\LD@Empty
			\LD@Yomigana\LD@Kanji@Reading\LD@Kana@Reading
%			{%
%			\LD@Tango@Info@Get\LD@Kanji@Reading
%			\LD@Akuma@Option@Process\LD@Kanji@Info
%			\ifx\LD@JLPT\LD@Empty
%				\ifx\LD@nf\LD@Empty
%					\LD@Stat@Update{6|\TangoMaxKanjiLevel}%
%					\LD@Stat@Update{6|G\TangoMaxKanjiGrade}%
%				\else
%					\LD@Stat@Update{5|\TangoMaxKanjiLevel}%
%					\LD@Stat@Update{5|G\TangoMaxKanjiGrade}%
%				\fi
%			\else
%				\LD@Stat@Update{\LD@JLPT|\TangoMaxKanjiLevel}%
%				\LD@Stat@Update{\LD@JLPT|G\TangoMaxKanjiGrade}%
%			\fi
%			}%	
			\def\LD@Compound@Root{}% N
			\ifnum\LD@Yomigana@Quirks@G@Level>0\relax
				\global\advance\LD@Entry@Quirks by1\relax
				\LD@String@Length\LD@Kanji@Reading\LD@Counter@Temp@A
				\ifnum\LD@Counter@Temp@A=1\relax
					\LD@Stat@Update{Compound1}%
					\def\LD@Compound@Root{+}% Y
%					\LD@Kanji@Reading\qquad \LD@Kana@Reading\PAR
				\else
					\ifnum\LD@Counter@Temp@A=\LD@Yomigana@Quirks@G@Level\relax
						\LD@Stat@Update{Compound\the\LD@Counter@Temp@A}%
						\def\LD@Compound@Root{+}% Y
%						\LD@Kanji@Reading\qquad \LD@Kana@Reading\PAR
					\fi
				\fi
			\fi
			\global\advance\LD@Entry@Count by1\relax
			\LD@Stats@Yomigana@Process\LD@Kanji@Reading\LD@Yomigana@GlobalFurigana	
		%\fi	
	}}%
	\input LD@Akuma@JMdictParser.tex
	\LD@XML@Parser{JMdict.xml}%
}%
\def\LD@XML@Parser#1{%
	\LD@Time@Benchmark{Parsing #1...}%
	\xdef\LD@XML@SavedTimeStart{\the\LD@Time@Moment}%
	%\begingroup
		%\newlinechar=`\^^M\relax
		%\xdef\LD@Newline{^^M}%
		%\catcode`\^^M=13\relax
		%\catcode`@=11\relax
		%\catcode`]=0\relax
		%\catcode`>=12\relax
		%\def\>{\catcode`\]=11\relax}%
		%\catcode9=13\relax% Tabs
		%\catcode`(=11\relax
		%\catcode`)=11\relax
		%\catcode`.=11\relax
		%\catcode`:=11\relax
		%\catcode`§=11\relax
		%\catcode`*=11\relax
		%\catcode`-=11\relax
		%\catcode`,=11\relax
		%\catcode`¤=0\relax
		\catcode`$=11\relax% Maths
		\catcode`^=11\relax% Superscript
		\catcode`_=11\relax% Subscript
		\catcode`&=11\relax% Alignement
		\catcode`\#=11\relax% Parameters
		\catcode`?=11\relax
		\catcode`!=11\relax
		\catcode`[=11\relax
		\catcode`~=11\relax
		\catcode`/=11\relax
		\catcode`'=11\relax
		\catcode`"=11\relax
		\catcode`+=11\relax
		\catcode`°=11\relax
		\catcode`;=11\relax
		\catcode`µ=11\relax
		\catcode`£=11\relax
		\catcode`¨=11\relax
		\catcode`\{=11\relax
		\catcode`\}=11\relax
		\catcode`\%=11\relax
		\catcode`\\=11\relax
		\LD@XML@Init
		\input #1%
	%\endgroup
	\LD@Time@Moment=\LD@XML@SavedTimeStart\relax	
	\LD@Time@Benchmark [silent]\LD@Temp
	\immediate\write16{Parsing of #1 done in \LD@Temp.}%
	\LD@Font@Kana
	\global\LD@Counter@A=0\relax
	\loop
		\advance\LD@Counter@A by1\relax
		\noindent\the\LD@Counter@A : \CS\the\LD@Counter@A\EC. \CS LD@Stats@@\CS\the\LD@Counter@A\EC\EC\PAR
	\ifnum\LD@Counter@A<\LD@XML@Entries\relax
	\repeat
	% remove the trailing |
	\LD@String@LastChar\LD@Reading@Compound@List\LD@Reading@Compound@List\LD@trash
	\LD@Loop@For\LD@Compound=\LD@Reading@Compound@List\WithSeparator |\Do{%
		\noindent|\LD@Compound| : \LD@Stat@Display{\LD@Compound}\PAR
	}%
	\catcode`\%=14\relax
	\catcode`\\=0\relax
	\catcode`\{=1\relax
	\catcode`\}=2\relax
	\catcode`$=3\relax% Maths
	\catcode`&=4\relax% Alignement
	% 5 is end of line ^M
	\catcode`\#=6\relax% Parameters
	\catcode`^=7\relax% Superscript
	\catcode`_=8\relax% Subscript
	% 9 is ignored char
	% 10 is space
	% 11 is letter
	% 12 is other
	% 13 is active
	% 14 is comment
	% 15 is invalid
}%
\LD@JMdict@Load@Stats

%%%%%%%%%%%%%%%%%%%%%%%%%%%%%%%%%%%%%%%%%%%%%%%%%%%%%
%												%
%					Stats							%
%												%
%%%%%%%%%%%%%%%%%%%%%%%%%%%%%%%%%%%%%%%%%%%%%%%%%%%%%

%\singlecolumn
%\LD@Stats
%\write16{\the\LD@Entry@Count Entries}%

%%%%%%%%%%%%%%%%%%%%%%%%%%%%%%%%%%%%%%%%%%%%%%%%%%%%%
%												%
%					Kanji Reading usage					%
%												%
%%%%%%%%%%%%%%%%%%%%%%%%%%%%%%%%%%%%%%%%%%%%%%%%%%%%%
\bye


\directlua1{
	
	function oltest(a,b)
		if b==nil then
			tex.print("nil")
		else
			tex.print(a..b)
		end
	end
	
	function LD_Yomigana_Compute(LD_Yomigana_L_Kanji, LD_Yomigana_L_Kana) 


		 local function LD_Yomigana_Process(LD_Yomigana_L_Kanji, LD_Yomigana_L_Kana, LD_Yomigana_Counter_L_Kanji, LD_Yomigana_Counter_L_Kana, LD_Yomigana_L_Start, LD_Yomigana_L_End)
			% The Lakedaemonian Recursive Function implementing a kind of Dichotomy.  
	 
		local function LD_Yomigana_OneKanji()
			% One Kanji case. This case does the Grunt Work (it compares with the lectures) and happens a lot. 	
			
			
			local function LD_Yomigana_ValidCase()
				LD_Yomigana_GlobalFurigana = LD_Yomigana_L_Kana .. "|"
				LD_Yomigana_Quirks_G_Mode = 0
			end
		
			function LD_Yomigana_QuirkCase()
		
				LD_Yomigana_GlobalFurigana = "*" .. LD_Yomigana_L_Kana .. "|"
				if LD_Yomigana_Quirks_G_Mode < 2 then
					LD_Yomigana_Quirks_G_Level = LD_Yomigana_Quirks_G_Level + 1
				else
					% Forbidden case
					LD_Yomigana_Quirks_G_Level = LD_Yomigana_Quirks_G_Level + 666
				end
				if LD_Yomigana_L_Kana == "" then
					LD_Yomigana_Quirks_G_Mode = 1 % global
				else
					LD_Yomigana_Quirks_G_Mode = 2 % global
				end

			end
	
			local function LD_Yomigana_Unknown()
				LD_Yomigana_Quirks_G_Level = LD_Yomigana_Quirks_G_Level + 9
				if LD_Yomigana_L_Kana == "" then
					LD_Yomigana_QuirkCase() 
%				elseif not(unicode.utf8.find(LD_Yomigana_KurikaeshiSymbols,LD_Yomigana_L_Kanji)== nil) then
%					LD_Yomigana_Kurikaeshi_Support() for the time being
				else
					local Kanji_number = unicode.utf8.byte(LD_Yomigana_L_Kanji)
					if Kanji_number > 12352 and Kanji_number < 12436 and Kanji_number +96 == unicode.utf8.byte(LD_Yomigana_L_Kana) then 
%				 hiragana with katakana reading
						LD_Yomigana_Quirks_G_Level = LD_Yomigana_Quirks_G_Level - 9
						LD_Yomigana_ValidCase() 
					else 
						LD_Yomigana_QuirkCase() 
					end
				end	
%				LD_Yomigana_Kurikaeshi_G_Kanji="" % it is unknown....
			end
			
			LD_Yomigana_Test = LD_Kanji_Data[LD_Yomigana_L_Kanji .. LD_Yomigana_L_Start .. LD_Yomigana_L_End] 
			if not(LD_Yomigana_Test == nil) then
				if unicode.utf8.find(LD_Yomigana_Test, LD_Yomigana_L_Kana) == nil then
					LD_Yomigana_QuirkCase() % Fix here
				else
					LD_Yomigana_ValidCase()
				end
%%%				LD_Yomigana_Kurikaeshi_G_Kanji=LD_Yomigana_L_Kanji % global
			else
				if LD_Yomigana_L_Kanji == LD_Yomigana_L_Kana then % Fix here
					LD_Yomigana_ValidCase() % Fix here
%%%					LD_Yomigana_Kurikaeshi_G_Kanji=LD_Yomigana_L_Kanji % global usefull for kana
				else
					LD_Yomigana_Unknown() % Fix here
				end
			end
		end
			
			
		local function LD_Yomigana_Dichotomy()
		
		
			local LD_Yomigana_Quirks_L_Level, LD_Yomigana_Quirks_L_Mode, LD_Yomigana_S_Start, LD_Yomigana_S_End
		
			local function LD_Yomigana_Dichotomy_Initiate()
				LD_Yomigana_Quirks_L_Level = LD_Yomigana_Quirks_G_Level
				LD_Yomigana_Quirks_L_Mode = LD_Yomigana_Quirks_G_Mode
				LD_Yomigana_S_Start = LD_Yomigana_L_Start
				LD_Yomigana_S_End = LD_Yomigana_L_End
			end

			local LD_Yomigana_Counter_S_KanjiA, LD_Yomigana_Counter_S_KanjiB
			local LD_Yomigana_Counter_S_KanaA, LD_Yomigana_Counter_S_KanaB, LD_Yomigana_Counter_S_KanaC, LD_Yomigana_Counter_S_KanaD
			local LD_Yomigana_TangoA, LD_Yomigana_TangoB % inconsistant
			local LD_Yomigana_KanaA, LD_Yomigana_KanaB, LD_Yomigana_KanaC, LD_Yomigana_KanaD
			local 	LD_Yomigana_Counter_Loop
		
			local function LD_Yomigata_Dichotomy_Compute()
				%
				%	-- First Compute \LD@Yomigana@Counter@S@KanjiA=[(T+1)/2]
				%
				local T = LD_Yomigana_Counter_L_Kanji
				LD_Yomigana_Counter_S_KanjiA = math.floor ((T+1)/2)
				LD_Yomigana_Counter_S_KanjiB = T-LD_Yomigana_Counter_S_KanjiA
				%
				%	-- now split the Tangostring with  the first [(T+1)/2] tokens in TangoA and the remains in TangoB
				%
				LD_Yomigana_TangoA = unicode.utf8.sub(LD_Yomigana_L_Kanji, 1, LD_Yomigana_Counter_S_KanjiA)
				LD_Yomigana_TangoB = unicode.utf8.sub(LD_Yomigana_L_Kanji, -LD_Yomigana_Counter_S_KanjiB,-1)
				%
				%	-- Do the same with Kanas : put the first [(K+1)/2] kanas in KanaA and the remaining kanas in KanaB
				%
				local K = LD_Yomigana_Counter_L_Kana
				LD_Yomigana_Counter_S_KanaA = math.floor ((K+1)/2)
				LD_Yomigana_Counter_S_KanaB = K-LD_Yomigana_Counter_S_KanaA
				%
				%	-- now split the Tangostring with  the first [(n+1)/2] tokens in KanaA and the remains in KanaB 
				%
				LD_Yomigana_KanaA = unicode.utf8.sub(LD_Yomigana_L_Kana, 1, LD_Yomigana_Counter_S_KanaA)
				LD_Yomigana_KanaB = unicode.utf8.sub(LD_Yomigana_L_Kana, -LD_Yomigana_Counter_S_KanaB,-1)
				%
				%	-- Copy the Data for the Crabwalk
				%
				LD_Yomigana_Counter_Loop = K
				LD_Yomigana_KanaC = LD_Yomigana_KanaA
				LD_Yomigana_KanaD = LD_Yomigana_KanaB
				LD_Yomigana_Counter_S_KanaC = LD_Yomigana_Counter_S_KanaA
				LD_Yomigana_Counter_S_KanaD = LD_Yomigana_Counter_S_KanaB
			end			
		
			local function LD_Yomigana_CrabWalk_Initiate()
				LD_Yomigana_Quirks_G_Mode = LD_Yomigana_Quirks_L_Mode
				LD_Yomigana_Quirks_G_Level = LD_Yomigana_Quirks_L_Level				
			end	
		
		
			local LD_Yomigana_BestFurigana, LD_Yomigana_Quirks_S_Mode, LD_Yomigana_Quirks_S_Level, LD_Yomigana_SavedFurigana

			local function LD_Yomigana_CrabWalk_Update()
				% Updates the situation
				if LD_Yomigana_Quirks_G_Level < LD_Yomigana_Quirks_S_Level then
					% We have a winner here, save it's data
					LD_Yomigana_BestFurigana = LD_Yomigana_SavedFurigana .. LD_Yomigana_GlobalFurigana
					LD_Yomigana_Quirks_S_Mode = LD_Yomigana_Quirks_G_Mode  %%%%%%%%%%%%%%%% strange ! string ?
					LD_Yomigana_Quirks_S_Level = LD_Yomigana_Quirks_G_Level  %%%%%%%%%%%%%%%% strange ! string ?
					if LD_Yomigana_Quirks_G_Level == LD_Yomigana_Quirks_L_Level then
						LD_Yomigana_Counter_Loop = 0 % break the loop
					end
				end
			end
			
			
			local function LD_Yomigana_OddWalk()
				% Summon the first Daemon
%				LD_Yomigana_L_Start = LD_Yomigana_S_Start
%				LD_Yomigana_L_End = "N"
%				LD_Yomigana_L_Kanji = LD_Yomigana_TangoA
%				LD_Yomigana_L_Kana = LD_Yomigana_KanaC
%				LD_Yomigana_Counter_L_Kanji = LD_Yomigana_Counter_S_KanjiA
%				LD_Yomigana_Counter_L_Kana = LD_Yomigana_Counter_S_KanaC
				%LD_Yomigana_GlobalFurigana, LD_Yomigana_Quirks_G_Mode, 	LD_Yomigana_Quirks_G_Level = 
%%%%				tex.print("OL : ".. LD_Yomigana_L_Kana )
%				LD_Yomigana_Process(LD_Yomigana_L_Kanji, LD_Yomigana_L_Kana, LD_Yomigana_Counter_L_Kanji, LD_Yomigana_Counter_L_Kana, 	LD_Yomigana_L_Start, LD_Yomigana_L_End) 
				LD_Yomigana_Process(LD_Yomigana_TangoA, LD_Yomigana_KanaC, LD_Yomigana_Counter_S_KanjiA, LD_Yomigana_Counter_S_KanaC, 	LD_Yomigana_L_Start, "N") 
%%%%				tex.print("LOL : ".. LD_Yomigana_L_Kana.."  LO")
%				LD_Yomigana_Process() % fix here
				% Test if it is usefull to continue
				if LD_Yomigana_Quirks_G_Level < LD_Yomigana_Quirks_S_Level then %%%let's go for the first solution in the first try
					% We save stuff
					LD_Yomigana_SavedFurigana = LD_Yomigana_GlobalFurigana
					% Bring the second Daemon to the fray
%					LD_Yomigana_L_Start = "N"
%					LD_Yomigana_S_End = LD_Yomigana_L_End
%					LD_Yomigana_L_Kanji = LD_Yomigana_TangoB
%					LD_Yomigana_L_Kana = LD_Yomigana_KanaD
%					LD_Yomigana_Counter_L_Kanji = LD_Yomigana_Counter_S_KanjiB
%					LD_Yomigana_Counter_L_Kana = LD_Yomigana_Counter_S_KanaD
%					LD_Yomigana_Process() % fix here
					%LD_Yomigana_Process(LD_Yomigana_L_Kanji, LD_Yomigana_L_Kana, LD_Yomigana_Counter_L_Kanji, LD_Yomigana_Counter_L_Kana, 	LD_Yomigana_L_Start, LD_Yomigana_L_End) 
					LD_Yomigana_Process(LD_Yomigana_TangoB, LD_Yomigana_KanaD, LD_Yomigana_Counter_S_KanjiB, LD_Yomigana_Counter_S_KanaD, "N", LD_Yomigana_L_End) 
					LD_Yomigana_CrabWalk_Update()
				end
				% Crabwalk in the odd case... \LD@Yomigana@KanaA Shrinkens \LD@Yomigana@KanaB grows 
				LD_Yomigana_KanaB = unicode.utf8.sub(LD_Yomigana_KanaA, -1) ..  LD_Yomigana_KanaB
				LD_Yomigana_Counter_S_KanaB = LD_Yomigana_Counter_S_KanaB + 1
				LD_Yomigana_Counter_S_KanaA = LD_Yomigana_Counter_S_KanaA - 1
				LD_Yomigana_KanaA = unicode.utf8.sub(LD_Yomigana_KanaA, 1, LD_Yomigana_Counter_S_KanaA)
			end

			local function LD_Yomigana_EvenWalk()
				% Summon the first Daemon
		%		LD_Yomigana_L_Start = LD_Yomigana_S_Start
		%		LD_Yomigana_L_End = "N"
		%		LD_Yomigana_L_Kanji = LD_Yomigana_TangoA
		%		LD_Yomigana_L_Kana = LD_Yomigana_KanaA
		%		LD_Yomigana_Counter_L_Kanji = LD_Yomigana_Counter_S_KanjiA
		%		LD_Yomigana_Counter_L_Kana = LD_Yomigana_Counter_S_KanaA
%				LD_Yomigana_Process() % fix here
				LD_Yomigana_Process(LD_Yomigana_TangoA, LD_Yomigana_KanaA, LD_Yomigana_Counter_S_KanjiA, LD_Yomigana_Counter_S_KanaA, 	LD_Yomigana_L_Start, "N") 
				% Test if it is usefull to continue
				if LD_Yomigana_Quirks_G_Level < LD_Yomigana_Quirks_S_Level then %%%let's go for the first solution in the first try
					% We save stuff
					LD_Yomigana_SavedFurigana = LD_Yomigana_GlobalFurigana
					% Bring the second Daemon to the fray
				%	LD_Yomigana_L_Start = "N"
				%	LD_Yomigana_S_End = LD_Yomigana_L_End
				%	LD_Yomigana_L_Kanji = LD_Yomigana_TangoB
				%	LD_Yomigana_L_Kana = LD_Yomigana_KanaB
				%	LD_Yomigana_Counter_L_Kanji = LD_Yomigana_Counter_S_KanjiB
				%	LD_Yomigana_Counter_L_Kana = LD_Yomigana_Counter_S_KanaB
%					LD_Yomigana_Process() % fix here
					LD_Yomigana_Process(LD_Yomigana_TangoB, LD_Yomigana_KanaB, LD_Yomigana_Counter_S_KanjiB, LD_Yomigana_Counter_S_KanaB, "N", LD_Yomigana_L_End) 
					LD_Yomigana_CrabWalk_Update()
				end
				% Crabwalk in the even case... \LD@Yomigana@KanaC grows \LD@Yomigana@KanaD shrinkens 
				LD_Yomigana_KanaC =  LD_Yomigana_KanaC .. unicode.utf8.sub(LD_Yomigana_KanaD, 1, 1) 
				LD_Yomigana_Counter_S_KanaC = LD_Yomigana_Counter_S_KanaC + 1
				LD_Yomigana_Counter_S_KanaD = LD_Yomigana_Counter_S_KanaD - 1
				if LD_Yomigana_Counter_S_KanaD == 0 then
					LD_Yomigana_KanaD = ""
				else
					LD_Yomigana_KanaD = unicode.utf8.sub(LD_Yomigana_KanaD, -LD_Yomigana_Counter_S_KanaD)
				end
			end
		

			
			
			local function LD_Yomigana_Dichotomy_Return()
				% Return stuff
				LD_Yomigana_GlobalFurigana = LD_Yomigana_BestFurigana
				LD_Yomigana_Quirks_G_Mode = LD_Yomigana_Quirks_S_Mode
				LD_Yomigana_Quirks_G_Level = LD_Yomigana_Quirks_S_Level
			end
		
	
		
			LD_Yomigana_Dichotomy_Initiate()
			LD_Yomigata_Dichotomy_Compute()
			LD_Yomigana_Quirks_S_Level = 667
			repeat
%			tex.print("\string\\par Loop a : "..LD_Yomigana_Counter_Loop)
				LD_Yomigana_CrabWalk_Initiate()
%				tex.print("\string\\smallskip K1 : " .. LD_Yomigana_TangoA .."\string\\quad K2 : " .. LD_Yomigana_TangoB .."\string\\quad A : " .. LD_Yomigana_KanaA .. " " .. LD_Yomigana_Counter_S_KanaA .. "\string\\quad B : ", LD_Yomigana_KanaB .. " " .. LD_Yomigana_Counter_S_KanaB .. "\string\\quad C : " .. LD_Yomigana_KanaC .. " " .. LD_Yomigana_Counter_S_KanaC .. "\string\\quad D : " .. LD_Yomigana_KanaD .. " " .. LD_Yomigana_Counter_S_KanaD)
				if 2 * math.floor(LD_Yomigana_Counter_Loop/2) == LD_Yomigana_Counter_Loop then
					LD_Yomigana_EvenWalk()
				else
					LD_Yomigana_OddWalk()
				end
%				tex.print("\string\\par Loop b : "..LD_Yomigana_Counter_Loop)
				LD_Yomigana_Counter_Loop = LD_Yomigana_Counter_Loop  - 1
			until LD_Yomigana_Counter_Loop < 0			
			LD_Yomigana_Dichotomy_Return()
		end                                                    
	
%		tex.print("Call \string\\par Kanji : "..LD_Yomigana_L_Kanji.."\string\\qquad Kana : "..LD_Yomigana_L_Kana)
		if LD_Yomigana_Counter_L_Kanji == 1 then
			LD_Yomigana_OneKanji() % fix here
		elseif  unicode.utf8.find(LD_Yomigana_Compounds, "|" .. LD_Yomigana_L_Kanji .. "," .. LD_Yomigana_L_Kana .. "|") == nil then
			LD_Yomigana_Dichotomy() % fix here
		else % Support for Regular Compounds. Return stuff
			LD_Yomigana_GlobalFurigana = ""
			LD_Yomigana_GlobalFurigana = LD_Yomigana_GlobalFurigana .. 	unicode.utf8.rep("*|", LD_Yomigana_Counter_L_Kanji-1)
			LD_Yomigana_GlobalFurigana = LD_Yomigana_GlobalFurigana .. "*" .. 			LD_Yomigana_L_Kana .. "|"
			LD_Yomigana_Quirks_G_Mode = 0
			LD_Yomigana_Kurikaeshi_G_Kana = ""
			LD_Yomigana_Kurikaeshi_G_Kanji = ""
			LD_Yomigana_Kurikaeshi_G_Mode = ""
			
							
		end
			tex.print("out "..LD_Yomigana_Quirks_G_Level)
	end
	
	
	
		% initializes things and starts the recursive process
		% TokenCount of the Kanji String
		local LD_Yomigana_Counter_L_Kanji=unicode.utf8.len(LD_Yomigana_L_Kanji) 
		% TokenCount of the Kana string
		local LD_Yomigana_Counter_L_Kana=unicode.utf8.len(LD_Yomigana_L_Kana) 
		LD_Yomigana_GlobalFurigana="" % global
%%%		LD_Yomigana_Quirks_S_Mode=0 % usefull ?
		LD_Yomigana_Quirks_G_Level=0 % global
		% QuirkMode  : 0 (regular) 1 (Null quirk) 2 (non Null quirk)
		LD_Yomigana_Quirks_G_Mode=0 % global
%%%		LD_Yomigana_Kurikaeshi_G_Kana="" % global
%%%		LD_Yomigana_Kurikaeshi_G_Kanji="" % global	
%%%		LD_Yomigana_Kurikaeshi_G_Mode="" % global	
%		LD_Yomigana_L_Start="Y" %  : First ? Y or N
%		LD_Yomigana_L_End="Y" % : Last ? Y or N
		LD_Yomigana_Process(LD_Yomigana_L_Kanji, LD_Yomigana_L_Kana, LD_Yomigana_Counter_L_Kanji, LD_Yomigana_Counter_L_Kana, "Y", "Y") 
	        oltest("\string\\par Furigana : ",LD_Yomigana_GlobalFurigana)
		 oltest("\string\\qquad Quirks : ",LD_Yomigana_Quirks_G_Level)
	end
	
	LD_Yomigana_Compute("本","ほん")
	LD_Yomigana_Compute("日本","にほん")
	LD_Yomigana_Compute("大丈夫","だいじょうぶ")
	LD_Yomigana_Compute("運転手","うんてんしゅ")
	LD_Yomigana_Compute("本部","ほんぶ")
	LD_Yomigana_Compute("本","ん")
	LD_Yomigana_Compute("日本","にん")
	LD_Yomigana_Compute("大丈夫","だじょうぶ")
	LD_Yomigana_Compute("運転手","だいじょうぶ")
	LD_Yomigana_Compute("本部","ほん")

}%
\bye