%%%%%%%%%%%%%%%%%%%%%%%%%%%%%%%%%%%%%%%%%%%%%%%%%%%%%
%                                                                                                                                            %
%                                                          Fontes                                                                       %
%                                                                                                                                            %
%%%%%%%%%%%%%%%%%%%%%%%%%%%%%%%%%%%%%%%%%%%%%%%%%%%%%

% : fontes true type
%

\def\Font #1@#2pt{\font\Fonteol=ec#1 at #2pt\Fonteol}

\font\LD@Font@Book=cmbx10 scaled 2000\relax 
%
% cursives : georges, brigitte
% 
% Special : haunt=hauntaoe
%
% dingbats : fantastic, creature
%

%  : fonte romaine en 5, 6, 7, 8, 9, 10, 12, 14 et 17 points
% 

\font\seventeenrm=cmr12 scaled 1400
\font\fourteenrm=cmr12 scaled 1200
\font\twelverm=cmr12
\font\elevenrm=cmr10 scaled 1100
\font\tenrm=cmr10
\font\ninerm=cmr9
\font\eightrm=cmr8
\font\sevenrm=cmr7
\font\sixrm=cmr6
\font\fiverm=cmr5

%  : fonte romaine grasse en 5, 6, 7, 8, 9, 10, 12, 14 et 17 points
% 

%\font\seventeenbf=cmbx12 scaled 1400
\font\seventeenbf=cmbx10 scaled 1700
\font\fourteenbf=cmbx10 scaled 1400
\font\twelvebf=cmbx12
\font\elevenbf=cmbx10 scaled 1100
\font\tenbf=cmbx10
\font\ninebf=cmbx9
\font\eightbf=cmbx8
\font\sixbf=cmbx6
\font\fivebf=cmbx10 at 5pt

\font\chiffre=cmbx10 scaled 1600
\font\gras=cmbx10 scaled 1100             % Gras romain (boldface)
\font\TIT=cmbx12 scaled 1600              % Titres
\font\titre=cmbx12 scaled 1200
\font\pti=cmbx12 scaled 1200

%  : fonte italique en 5, 6, 7, 8, 9, 10, 12, 14 et 17 points
% 

\font\seventeenit=cmti10 scaled 1700
\font\fourteenit=cmti10 scaled 1400
\font\twelveit=cmti12
\font\elevenit=cmti10 scaled 1100
\font\tenit=cmti10
\font\nineit=cmti9
\font\eightit=cmti8
\font\sevenit=cmti7
\font\sixit=cmti10 scaled 600
\font\fiveit=cmti10 scaled 500

%  : fonte italique math\'ematique en 6, 7, 8, 9, 10, 11 et 12  points
% 

\font\twelvei=cmmi10 scaled 1200
\font\eleveni=cmmi10 scaled 1100
\font\teni=cmmi10
\font\ninei=cmmi9
\font\eighti=cmmi8
\font\seveni=cmmi7
\font\sixi=cmmi6

% \msafam : famille de symboles amstex en 5, 6, 7, 8, 9, 10, 12, 14 et 17 points
%

\font\seventeenmsa=msam10 at 17pt   
\font\fourteenmsa=msam10 at 14pt
\font\twelvemsa=msam10 at 12pt
\font\tenmsa=msam10
\font\ninemsa=msam10 at 9pt
\font\eightmsa=msam10 at 8pt
\font\sevenmsa=msam7
\font\sixmsa=msam10 at 6pt
\font\fivemsa=msam5

\newfam\msafam
\textfont\msafam=\tenmsa
\scriptfont\msafam=\sevenmsa
\scriptscriptfont\msafam=\fivemsa

% \scalnfam : famille de lettres de ronde en 5, 6, 7, 8, 9, 10, 12, 14 et 17 points
% \scal=\scalnfam en 10 points

\font\seventeenscaln=eusm10 at 17pt
\font\fourteenscaln=eusm10 at 14pt
\font\twelvescaln=eusm10 at 12pt
\font\tenscaln=eusm10
\font\ninescaln=eusm10 scaled 900
\font\eightscaln=eusm10 scaled 800
\font\sevenscaln=eusm10 scaled 700
\font\sixscaln=eusm10 scaled 600
\font\fivescaln=eusm5

\newfam\scalnfam
\textfont\scalnfam=\tenscaln
\scriptfont\scalnfam=\sevenscaln
\scriptscriptfont\scalnfam=\fivescaln

\def\scal{\fam\scalnfam\tenscaln}

% : famille de fontes slanted en 5, 6, 7, 8, 9 et 10  points
% 
\font\tensl=cmsl10
\font\ninesl=cmsl9                 
\font\eightsl=cmsl8
\font\sevensl=cmsl10 at 7pt
\font\sixsl=cmsl10 at 6pt
\font\fivesl=cmsl10 at 5pt


% \gotfam : famille de lettres gothiques en 5, 6, 7, 8, 9 et 10  points
% \got=\gotfam en 10 points

\font\tengot=eufm10                       
\font\ninegot=eufm10 at 9pt
\font\eightgot=eufm10 at 8truept
\font\sevengot=eufm7
\font\sixgot=eufm10 at 6 truept
\font\fivegot=eufm5

\newfam\gotfam
\textfont\gotfam=\tengot
\scriptfont\gotfam=\sevengot
\scriptscriptfont\gotfam=\fivegot
\def\got{\fam\gotfam\tengot}

% \bbfam : famille de lettres évidées pour titres en 5, 6, 7, 8, 9, 10, 12, 14 et 17 points
% \bb=\bbfam en 10 points

\font\seventeenbb=msbm10 at 17pt     
\font\fourteenbb=msbm10 at 14pt
\font\twelvebb=msbm10 at 12pt
\font\tenbb=msbm10                   
\font\ninebb=msbm10 at 9pt
\font\eightbb=msbm10 at 8pt
\font\sevenbb=msbm7
\font\sixbb=msbm10 at 6pt
\font\fivebb=msbm5

\newfam\bbfam
\textfont\bbfam=\tenbb
\scriptfont\bbfam=\sevenbb
\scriptscriptfont\bbfam=\fivebb

\def\bb{\fam\bbfam\tenbb}

%  : fonte italique grasse pour titres en 9, 10, 12, 11, 12, 13, 14 et 17 points
% 

\font\seventeenti=cmbxti10 scaled 1700
\font\fourteenti=cmbxti10 at 14pt
\font\thirteenti=cmbxti10 at 13pt
\font\twelveti=cmbxti10 scaled 1200
\font\eleventi=cmbxti10 at 11pt
\font\tenti=cmbxti10
\font\nineti=cmbxti10 at 9pt
\newfam\bolditfam
\textfont\bolditfam=\tenti
\def\boldit{\fam\bolditfam\tenti}

%  : fonte italique math\'ematique grasse pour titres en 9, 10, 11, 12, 14 et 17 points
% 

\font\seventeenib=cmmib10 scaled 1700
\font\fourteenib=cmmib10 scaled 1400
\font\twelveib=cmmib10 scaled 1200
\font\elevenib=cmmib10 scaled 1100
\font\tenib=cmmib10
\font\nineib=cmmib10 scaled 900

% \scalbfam : famille de lettres de ronde grasses en 5, 6, 7, 8, 9, 10, 12, 14 et 17 points
% \scalb=\scalbfam en 10 points

\font\seventeenscalb=eusb10  at 17pt 
\font\fourteenscalb=eusb10 at 14pt
\font\twelvescalb=eusb10 at 12pt
\font\tenscalb=eusb10      
\font\ninescalb=eusb10 scaled 900
\font\eightscalb=eusb10 scaled 800
\font\sevenscalb=eusb10 scaled 700
\font\sixscalb=eusb10 scaled 600
\font\fivescalb=eusb5

\newfam\scalbfam
\textfont\scalbfam=\tenscalb
\scriptfont\scalbfam=\sevenscalb
\scriptscriptfont\scalbfam=\fivescalb

\def\scalb{\fam\scalbfam\tenscalb}


% \pcapfam : famille de petites capitales en 5, 6, 7, 8, 9, 10, 12, 14 et 17 points
% \pcap=\pcapfam en 10 points


\font\seventeenpcap=cmcsc10 at 17pt
\font\fourteenpcap=cmcsc10 at 17pt
\font\twelvepcap=cmcsc10 at 17pt
\font\tenpcap=cmcsc10
\font\ninepcap=cmcsc9
\font\eightpcap=cmcsc8
\font\sevenpcap=cmcsc10 scaled 700
\font\sixpcap=cmcsc10 scaled 600
\font\fivepcap=cmcsc10 scaled 500

\newfam\pcapfam
\textfont\pcapfam=\tenpcap
\scriptfont\pcapfam=\sevenpcap
\scriptscriptfont\pcapfam=\fivepcap

\def\pcap{\fam\pcapfam\tenpcap}

% : famille de lettres typewritter en 5, 6, 7, 8, 9, et 10 points
% 

\font\tentt=cmtt10
\font\ninett=cmtt9                 
\font\eighttt=cmtt8
\font\seventt=cmtt10 scaled 700
\font\sixtt=cmtt10 scaled 600
\font\fivett=cmtt10 scaled 500

% : famille de symboles pour titres  en 6, 8, 9, 10, 12, 14, et 17 points
% 

\font\seventeensy=cmsy10 scaled 1700
\font\fourteensy=cmsy10 scaled 1400
\font\twelvesy=cmsy10 scaled 1200
\font\tensy=cmsy10
\font\ninesy=cmsy9
\font\eightsy=cmsy8
\font\sixsy=cmsy6


% : famille de ? en 5, 6, 7, 8, 9, 10, 12, 14, et 17 points
% 

\font\seventeenex=cmex10 at 17pt
\font\fourteenex=cmex10 at 14pt
\font\twelveex=cmex10 at 12pt
\font\tenex=cmex10
\font\nineex=cmex10 at 9pt
\font\eightex=cmex10 at 8pt
\font\sevenex=cmex10 at 7pt
\font\sixex=cmex10 at 6pt
\font\fiveex=cmex10 at 5pt

% : famille de  lettres grecques plombees en 5, 6, 7, 8, 9, 10, 11, 12, 13 et 14 points
% 

\font\fourteengp=cmmi10 at 14pt
\font\thirteengp=cmmib10 at 13pt
\font\twelvegp=cmmib10 at 12pt
\font\elevengp=cmmib10 at 11pt
\font\tengp=cmmib10
\font\ninegp=cmmib10 at 9pt
\font\eightgp=cmmib8
\font\sevengp=cmmib7
\font\sixgp=cmmib6
\font\fivegp=cmmib5

% : famille de lettres gothiques diverses en 12  points
% 

\font\yfrak=yfrak scaled 2000
\font\yswab=yswab scaled 1200
\font\twelveygoth=ygoth scaled 1200
\font\twentyygoth=ygoth scaled 2000
\font\thirtyygoth=ygoth scaled 4000

%%%%%%%%%%%%%%%%%%%%%%%%%%%%%%%%%%%%%%%%%%%%%%%%%%%%%
%                                                                                                                                            %
%                                              Paragraphes, fontes et points                                              %
%                                                                                                                                            %
%%%%%%%%%%%%%%%%%%%%%%%%%%%%%%%%%%%%%%%%%%%%%%%%%%%%%




% Paragraphes en 5 points
%

\def\fivepts{%
\textfont0=\fiverm\scriptfont0=\fiverm\scriptscriptfont0=\fiverm
\def\rm{\fam0\fiverm}%
%\textfont1=\fivei\scriptfont1=\fivei
%\scriptscriptfont1=\fivei\def\oldstyle{\fam1\fivei}%
%\textfont2=\fivesy\scriptfont2=\fivey\scriptscriptfont2=\fivesy
%\textfont3=\fiveex\scriptfont3=\fiveex
%\textfont\itfam=\fiveit
%\def\it{\fam\itfam\fiveit}%
%\textfont\slfam=\fivesl
%\def\sl{\fam\slfam\fivesl}%
%\textfont\bbfam=\fivebb \scriptfont\bbfam=\fivebb\scriptscriptfont\bbfam=\fivebb
%%\def\bb{\fam\bbfam\fivebb}%
%\textfont\msafam=\fivemsa\scriptfont\msafam=\fivemsa
%\textfont\scalnfam=\fivescaln
%\def\scaln{\fam\scalnfam\fivescaln}%
%\textfont\bffam=\fivebf\scriptfont\bffam=\fivebf\scriptscriptfont\bffam=\fivebf
%\def\bf{\fam\bffam\fivebf}%
%\textfont\ttfam=\fivett
\abovedisplayskip=1pt plus1pt minus 2pt
\belowdisplayskip=1pt plus0pt minus 2pt
\abovedisplayshortskip= 1pt plus 2pt minus 1pt
\belowdisplayshortskip= 1pt plus 2pt minus 1pt
\smallskipamount=1pt plus 1pt minus 2pt
\medskipamount=2pt plus 2pt minus 2pt
\bigskipamount=3pt plus 3pt minus 3pt
\setbox\strutbox=\hbox{\vrule height 3pt depth 2pt width 0pt}%
\parfillskip=0pt
\normalbaselineskip=3pt\normalbaselines\rm}



% Paragraphes en 7 points
%

\def\sevenpts{%
\textfont0=\sevenrm\scriptfont0=\sixrm\scriptscriptfont0=\fiverm
\def\rm{\fam0\sevenrm}%
\textfont1=\seveni\scriptfont1=\sixi
\scriptscriptfont1=\fivei\def\oldstyle{\fam1\seveni}%
\textfont2=\sevensy\scriptfont2=\sixsy\scriptscriptfont2=\fivesy
\textfont3=\sevenex\scriptfont3=\fiveex
\textfont\itfam=\sevenit
\def\it{\fam\itfam\sevenit}%
\textfont\slfam=\sevensl
\def\sl{\fam\slfam\sevensl}%
\textfont\bbfam=\sevenbb \scriptfont\bbfam=\sixbb\scriptscriptfont\bbfam=\fivebb
\def\bb{\fam\bbfam\sevenbb}%
\textfont\msafam=\sevenmsa\scriptfont\msafam=\sixmsa
\textfont\scalnfam=\sevenscaln
\def\scaln{\fam\scalnfam\sevenscaln}%
\textfont\bffam=\sevenbf\scriptfont\bffam=\sixbf\scriptscriptfont\bffam=\fivebf
\def\bf{\fam\bffam\sevenbf}%
\textfont\ttfam=\seventt
\abovedisplayskip=2pt plus2pt minus 2pt
\belowdisplayskip=2pt plus1pt minus 2pt
\abovedisplayshortskip= 1pt plus 2pt minus 1pt
\belowdisplayshortskip= 1pt plus 2pt minus 1pt
\smallskipamount=2pt plus 1pt minus 2pt
\medskipamount=3pt plus 2pt minus 2pt
\bigskipamount=7pt plus 3pt minus 3pt
\setbox\strutbox=\hbox{\vrule height 5pt depth 2pt width 0pt}%
\normalbaselineskip=9pt\normalbaselines\rm}


% Paragraphes en 8 points
%

\def\eightpts{%
\textfont0=\eightrm\scriptfont0=\sixrm\scriptscriptfont0=\fiverm
\def\rm{\fam0\eightrm}%
\textfont1=\eighti\scriptfont1=\sixi
\scriptscriptfont1=\fivei\def\oldstyle{\fam1\seveni}%
\textfont2=\eightsy\scriptfont2=\sixsy\scriptscriptfont2=\fivesy
\textfont3=\eightex\scriptfont3=\sixex
\textfont\itfam=\eightit
\def\it{\fam\itfam\eightit}%
\textfont\slfam=\eightsl
\def\sl{\fam\slfam\eightsl}%
\textfont\bbfam=\eightbb \scriptfont\bbfam=\sixbb\scriptscriptfont\bbfam=\fivebb
\def\bb{\fam\bbfam\eightbb}%
\textfont\msafam=\eightmsa\scriptfont\msafam=\sixmsa
\textfont\scalnfam=\eightscaln
\def\scaln{\fam\scalnfam\eightscaln}%
\textfont\ttfam=\eighttt
\def\tt{\fam\ttfam\eighttt}%
\textfont\bffam=\eightbf\scriptfont\bffam=\sixbf\scriptscriptfont\bffam=\fivebf
\def\bf{\fam\bffam\eightbf}%
\textfont\pcapfam=\eightpcap
\def\pcap{\fam\pcapfam\eightpcap}%
\abovedisplayskip=2pt plus2pt minus 2pt
\belowdisplayskip=2pt plus1pt minus 2pt
\abovedisplayshortskip= 1pt plus 2pt minus 1pt
\belowdisplayshortskip= 1pt plus 2pt minus 1pt
\smallskipamount=2pt plus 1pt minus 2pt
\medskipamount=3pt plus 2pt minus 2pt
\bigskipamount=7pt plus 3pt minus 3pt
\setbox\strutbox=\hbox{\vrule height 5pt depth 2pt width 0pt}%
\normalbaselineskip=9pt\normalbaselines
\rm}

% Paragraphes en 9 points
%

\def\ninepts{%
\textfont0=\ninerm\scriptfont0=\sixrm\scriptscriptfont0=\fiverm
\def\rm{\fam0\ninerm}%
\textfont1=\ninei\scriptfont1=\sixi
\scriptscriptfont1=\fivei\def\oldstyle{\fam1\ninei}%
\textfont2=\ninesy\scriptfont2=\sixsy\scriptscriptfont2=\fivesy
\textfont3=\nineex\scriptfont3=\sixex
\textfont\itfam=\nineit
\def\it{\fam\itfam\nineit}%
\textfont\slfam=\ninesl
\def\sl{\fam\slfam\ninesl}%
\textfont\bbfam=\ninebb\scriptfont\bbfam=\sixbb\scriptscriptfont\bbfam=\fivebb
\def\bb{\fam\bbfam\ninebb}%
\textfont\msafam=\ninemsa\scriptfont\msafam=\sixmsa\scriptscriptfont\msafam=\fivemsa
\textfont\scalnfam=\ninescaln
\def\scaln{\fam\scalnfam\ninescaln}%
\textfont\ttfam=\ninett
\def\tt{\fam\ttfam\ninett}%
\textfont\bffam=\ninebf\scriptfont\bffam=\sixbf\scriptscriptfont\bffam=\fivebf
\def\bf{\fam\bffam\ninebf}%
\abovedisplayskip=3pt plus2pt minus 2pt
\belowdisplayskip=3pt plus1pt minus 2pt
\abovedisplayshortskip= 2pt plus 2pt minus 1pt
\belowdisplayshortskip= 2pt plus 2pt minus 1pt
\smallskipamount=2pt plus 1pt minus 2pt
\medskipamount=3pt plus 2pt minus 2pt
\bigskipamount=7pt plus 3pt minus 3pt
\setbox\strutbox=\hbox{\vrule height 5pt depth 2pt width 0pt}%
\normalbaselineskip=10.5pt plus.3pt minus.3pt\normalbaselines\rm}


% Paragraphes en 10 points
%

\def\tenpts{%
\textfont0=\tenrm\scriptfont0=\sevenrm\scriptscriptfont0=\fiverm
\def\rm{\fam0\tenrm}%
\textfont1=\teni\scriptfont1=\seveni
\scriptscriptfont1=\fivei\def\oldstyle{\fam1\teni}%
\textfont2=\tensy\scriptfont2=\sevensy\scriptscriptfont2=\fivesy
\textfont3=\tenex\scriptfont3=\sevenex
\textfont\itfam=\tenit
\def\it{\fam\itfam\tenit}%
\textfont\slfam=\tensl
\def\sl{\fam\slfam\tensl}%
\textfont\bbfam=\tenbb\scriptfont\bbfam=\sevenbb\scriptscriptfont\bbfam=\fivebb
\def\bb{\fam\bbfam\tenbb}%
\textfont\msafam=\tenmsa\scriptfont\msafam=\sevenmsa\scriptscriptfont\msafam=\fivemsa
\textfont\scalnfam=\tenscaln
\def\scaln{\fam\scalnfam\tenscaln}%
\textfont\ttfam=\tentt
\def\tt{\fam\ttfam\tentt}%
\textfont\bffam=\tenbf\scriptfont\bffam=\sevenbf\scriptscriptfont\bffam=\fivebf
\def\bf{\fam\bffam\tenbf}%
%\abovedisplayskip=3pt plus2pt minus 2pt
%\belowdisplayskip=3pt plus1pt minus 2pt
%\abovedisplayshortskip= 2pt plus 2pt minus 1pt
%\belowdisplayshortskip= 2pt plus 2pt minus 1pt
%\smallskipamount=2pt plus 1pt minus 2pt
%\medskipamount=3pt plus 2pt minus 2pt
%\bigskipamount=7pt plus 3pt minus 3pt
%\setbox\strutbox=\hbox{\vrule height 5pt depth 2pt width 0pt}%
%\normalbaselineskip=10.5pt plus.3pt minus.3pt\normalbaselines
\rm}

% Paragraphes en 11 points

\def\elevenpts{%
\textfont0=\elevenrm\scriptfont0=\ninerm
\textfont1=\elevenib\scriptfont1=\ninei}


% Paragraphes en 12 points

\def\twelvepts{%
\textfont0=\twelverm\scriptfont0=\tenrm\def\rm{\fam0\twelverm}%
\textfont1=\twelveib\scriptfont1=\teni%
\textfont2=\twelvesy\scriptfont2=\tensy\scriptscriptfont2=\eightsy
\textfont3=\twelveex
\textfont\itfam=\twelveti
\def\it{\fam\itfam\twelveti}%
\textfont\bffam=\twelvebf\scriptfont\bffam=\tenbf\scriptscriptfont\bffam=\eightbf
\def\bf{\fam\bffam\twelvebf}%
\textfont\bbfam=\twelvebb \scriptfont\bbfam=\tenbb
\def\bb{\fam\bbfam\twelvebb}%
\textfont\msafam=\twelvemsa\scriptfont\msafam=\tenmsa
\textfont\scalnfam=\twelvescaln
\normalbaselineskip=15pt\normalbaselines\rm}


% Paragraphes en 14 points

\def\fourteenpts{%
\textfont0=\fourteenrm\scriptfont0=\twelverm\def\rm{\fam0\fourteenrm}%
\textfont1=\fourteenib\scriptfont1=\twelveib%
\textfont2=\fourteensy\scriptfont2=\twelvesy\scriptscriptfont2=\tensy
\textfont3=\fourteenex
\textfont\itfam=\fourteenti
\def\it{\fam\itfam\fourteenti}%
\textfont\bffam=\fourteenbf\scriptfont\bffam=\twelvebf\scriptscriptfont\bffam=\tenbf
\def\bf{\fam\bffam\fourteenbf}%
\textfont\bbfam=\fourteenbb \scriptfont\bbfam=\twelvebb
\def\bb{\fam\bbfam\fourteenbb}%
\textfont\msafam=\fourteenmsa\scriptfont\msafam=\twelvemsa
\textfont\scalnfam=\twelvescaln
\normalbaselineskip=18pt\normalbaselines\rm}

%%%%%% lettres grecques plombées
%

\def\gphuit{%
\textfont0=\eightbf\scriptfont0=\sixbf\scriptscriptfont0=\fivebf
\textfont1=\eightgp\scriptfont1=\sixgp\scriptscriptfont1=\fivegp}

\def\gpdix{%
\textfont0=\tenbf\scriptfont0=\sevenbf\scriptscriptfont0=\fivebf
\textfont1=\tengp\scriptfont1=\sevengp\scriptscriptfont1=\fivegp}

\def\gponze{%
\textfont0=\elevenbf\scriptfont0=\eightbf\scriptscriptfont0=\sixbf
\textfont1=\elevengp\scriptfont1=\eightgp\scriptscriptfont1=\sixgp}

\def\gpdouze{%
\textfont0=\twelvebf\scriptfont0=\tenbf\scriptscriptfont0=\ninebf
\textfont1=\twelvegp\scriptfont1=\tengp\scriptscriptfont1=\ninegp}

\def\gptreize{%
\textfont0=\thirteenbf\scriptfont0=\elevenbf\scriptscriptfont0=\tenbf
\textfont1=\thirteengp\scriptfont1=\elevengp\scriptscriptfont1=\tengp}

 \def\gpquatorze{%
\textfont0=\fourteenbf\scriptfont0=\twelvebf\scriptscriptfont0=\elevenbf
\textfont1=\fourteengp\scriptfont1=\twelvegp\scriptscriptfont1=\elevengp}

%%%%%%%%%%%%%%%%%%%%%%%%%%%%%%%%%%%%%%%%%%%%%%%%%%%%%
%                                                                                                                                            %
%                                                  Changements de fontes                                                    %
%                                                                                                                                            %
%%%%%%%%%%%%%%%%%%%%%%%%%%%%%%%%%%%%%%%%%%%%%%%%%%%%%

\def\ob#1{{\bb #1}}%         math
\def\sc#1{{\scal #1}}%       rond

%\def\og#1{{\bf #1}}%          gras
%\def\oc#1{{\cal #1}}%         rond
%\def\on#1{{\rm #1}}%         romain
%\def\op#1{%                       lettres plombées pauvres
%\setbox\olbox=\hbox{$#1$}\kern-.025em\copy\olbox
%\kern-\wd\olbox\kern.05em
%\copy\olbox\kern-\wd\olbox\kern-.025em\raise .0433em\box\olbox}
%

%%%%%%%%%%%%%%%%%%%%%%%%%%%%%%%%%%%%%%%%%%%%%%%%%%%%%
%                                                                                                                                            %
%                                                    Nouveaux Caractères                                                     %
%                                                                                                                                            %
%%%%%%%%%%%%%%%%%%%%%%%%%%%%%%%%%%%%%%%%%%%%%%%%%%%%%



\def\undefine#1{\let#1\undefined}

\def\newsymbol#1#2#3#4#5{\let\next@\relax
\ifnum#2=\@ne\let\next@\msafam@\else
\ifnum#2=\tw@\let\next@\bbfam@\fi\fi
\mathchardef#1="#3\next@#4#5}

\def\mathhexbox@#1#2#3{\relax
\ifmmode\mathpalette{}{\m@th\mathchar"#1#2#3}%
\else\leavevmode\hbox{$\m@th\mathchar"#1#2#3$}\fi}

\def\hexnumber@#1{\ifcase#1 0\or 1\or 2\or 3\or 4\or 5\or 6\or 7\or 8\or
9\or A\or B\or C\or D\or E\or F\fi}

\def\setboxz@h{\setbox\z@\hbox}
\def\wdz@{\wd\z@}
\def\boxz@{\box\z@}

\edef\msafam@{\hexnumber@\msafam}
\mathchardef\dabar@"0\msafam@39

\edef\bbfam@{\hexnumber@\bbfam}
\def\widehat#1{\setboxz@h{$\m@th#1$}%
\ifdim\wdz@>\tw@ em\mathaccent"0\bbfam@5B{#1}%
\else\mathaccent"0362{#1}\fi}

\def\widetilde#1{\setboxz@h{$\m@th#1$}%
\ifdim\wdz@>\tw@ em\mathaccent"0\bbfam@5D{#1}%
\else\mathaccent"0365{#1}\fi}

\newsymbol\leqq 1335          % superieur ou egal(=)
\newsymbol\leqslant 1336
\newsymbol\lessgtr 1337       % superieur ou inferieur
\newsymbol\backprime 1038     % apostrophe de gauche a droite
\newsymbol\risingdotseq 133A  % egal entre points (bas puis haut)
\newsymbol\fallingdotseq 133B % egal entre points (haut puis bas)
\newsymbol\succcurlyeq 133C   % superieur ou egal tordu
\newsymbol\geqq 133D          % inferieur ou egal(=)
\newsymbol\geqslant 133E
\newsymbol\nmid 232D
\newsymbol\nexists 2040
\newsymbol\smallsetminus 2272
\newsymbol\varnothing 203F
