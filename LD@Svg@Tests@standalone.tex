\magnification 1100\relax
\input eplain
\catcode`@=11\relax
\input LD@Fonts

\newif\ifHtml\Htmltrue
\def\pn{\noindent\HCode{<br />}}






% Debug tikz
\font\ninerm=cmr9% prerequisite Here
\font\texttt=cmr10
\font\scriptsize=cmr7
\font\textsf=cmr10
\def\itshape{} % need this in the er package
\def\RequirePackage#1{}
\let\ConfigureEnv\Configure
%% End Debug Tikz

\def\pgfsysdriver{pgfsys-tex4ht.def}
\input tikz.tex
\usetikzlibrary{petri,er,mindmap,arrows,trees,shapes,snakes,matrix,patterns,topaths,automata,backgrounds,plothandlers,plotmarks,fit,positioning,decorations.text}
\input pgffor.tex
\input pgfbaseimage.tex
\input pgfbasepatterns.tex

%% Debug Tikz
\catcode`@=11\relax
\def\f@size{10pt}% size of tikz fonts
\def\small{\tenrm}
\def\@flushglue{10pt}

%% End Debug Tikz



\csname tex4ht\endcsname



\catcode`@=11\relax


\def\c#1{\HCode{&ccedil;}}%
\overfullrule=0pt
%%%%%%%% Mbox=hbox containing text inside math mode % Mvbox ? Mhbox ?
%
\def\mbox{\@getoptionalarg\LD@System@Mbox}%
\def\LD@System@Mbox#1{{%
	\PauseMathClass
	\mtexttrue
	\HCode{<mtext>}%
	\Configure{$}{\HCode{&nbsp;</mtext><mrow>}\EndPauseMathClass\mtextfalse}{\PauseMathClass\mtexttrue\HCode{</mrow><mtext>&nbsp;}}{}%
	\hbox{#1}%
	\HCode{</mtext>}%
	\EndPauseMathClass	
}}%

%%%% après tex4ht sinon problèmes
\let\Q=\left
\let\W=\right
\let\F=\over
\let\b=\big
\let\B=\Big
\def\d{\,\mbox{\rm d}}                                              % d droit d'intégration
%%%



\let\EA=\expandafter
\let\CS=\csname
\let\EC=\endcsname
\let\Par\par

\ifHtml
\Css{
.tikz{ 
	float: left;
	clear: both;
	padding:20px;
}}
\Css{
.LD{
	float: left;
	padding: 20px;
}}
\fi


\font\LD@Font@Arial=cmr10


\newcount\LD@Svg@Count\global\LD@Svg@Count=1\relax
\long\def\LD@Svg@Test#1\LD@End@Svg@Test{{%
		\noindent\EndP
		\HCode{<div class="tikz" align="center"><h2>ForeignElement\the\LD@Svg@Count</h2>}%
		{%
			\tikzstyle{every picture}=[font=\SvgText,tex4ht node/escape=true]%
			 #1%
		}%
		\HCode{</div>}%	
		\HCode{<div class="LD" align="center"><h2>SvgText \the\LD@Svg@Count</h2>}%
		{%
			%\tikzstyle{every picture}=[font=\SvgText,text only=true]%
			#1%
		}%
		\HCode{</div>}%
		\Par
	\global\advance\LD@Svg@Count by1\relax
}}%



\font\SvgText=cmr10\relax
%\everypar={}

\LD@Svg@Test
\tikzpicture
\node [rectangle,fill=red!20] at (0,0) {A};
\node [rectangle,fill=red!20] at (2,0) {a};
\endtikzpicture
\LD@End@Svg@Test

\LD@Svg@Test
\tikzpicture
\draw (0,0) node[inner sep=0pt,draw] {tight}
(0cm,2em) node[inner sep=5pt,draw] {loose}
(0cm,4em) node[fill=red!20] {default};
\draw (3,0) node[inner sep=0pt,draw] {TIGHT}
(3cm,2em) node[inner sep=5pt,draw] {LOOSE}
(3cm,4em) node[fill=red!20] {DEFAULT};
\draw (6,0) node[inner sep=0pt,draw] {$tight$}
(6cm,2em) node[inner sep=5pt,draw] {$loose$}
(6cm,4em) node[fill=red!20] {$default$};
\draw (9,0) node[inner sep=0pt,draw] {$TIGHT$}
(9cm,2em) node[inner sep=5pt,draw] {$LOOSE$}
(9cm,4em) node[fill=red!20] {$Default$};
\endtikzpicture
\LD@End@Svg@Test


%%%%%%%%%%%% Test SVG 001%%%%%%%%%%%%%%

\LD@Svg@Test
\tikzpicture
\node [rectangle,fill=red!20] at (2,0) {a};
\node [ellipse,fill=blue!20] at (0,0) {$a$};
\endtikzpicture
\LD@End@Svg@Test
%%%%%%%%%%%% Test SVG 001%%%%%%%%%%%%%%
\LD@Svg@Test
\tikzpicture
   \draw (0,0) circle (1mm);  % b.north will lie on the center of this circle
 \pgfmatrix{rectangle}{text}{mymatrix}{\pgfusepath{}}{\pgfpointanchor{b}{north}}{\let\&=\pgfmatrixnextcell}
   {
     \draw (0,0) circle (1cm); \& \node (a) [draw] {Hallo}; \\
     \node (b) [draw] {Welt};  \& \draw (0,0) circle (5mm); \\
   }
   \draw [->] (a) -- (b);
 \endtikzpicture
\LD@End@Svg@Test

\LD@Svg@Test
\tikzpicture
\matrix [matrix of math nodes,ampersand replacement=\&,nodes={circle,draw},nodes in empty cells]
{
a_8 \& \& a_6 \\
a_3 \& \& a_7 \\
a_4 \& a_9 \& \\
};
\endtikzpicture
\LD@End@Svg@Test

\LD@Svg@Test
\tikzpicture
\node [ellipse,fill=red!20] at (2,0) {a};
\node [rectangle,fill=blue!20] at (0,0) {a $a^b$};
\endtikzpicture
\LD@End@Svg@Test



%%%%%%%%%%%% Test SVG %%%%%%%%%%%%%%

\LD@Svg@Test
\EA\def\CS Pos0\EC{0}
\EA\def\CS Pos1\EC{0.15}
\EA\def\CS Pos2\EC{0.37}
\EA\def\CS Pos3\EC{0.63}
\EA\def\CS Pos4\EC{0.85}
\EA\def\CS Pos5\EC{1}
\EA\def\CS Posout0\EC{0}
\EA\def\CS Posout1\EC{0.5}
\EA\def\CS Posout2\EC{1}
\pgfdeclarelayer{background}
\pgfsetlayers{background,main}
%
\tikzpicture
\tikzstyle{Classe}=[rectangle,rounded corners,minimum width=3.5cm,draw]
\tikzstyle{Option}=[ellipse,draw,inner sep=2pt]
\tikzstyle{Diplome}=[starburst,inner sep=1pt,draw]
\tikzstyle{Travail}=[ellipse,inner sep=2pt,draw]
\tikzstyle{Cursus}=[-triangle 60,very thick,draw]
\path (3,2) .. controls (4,-2) and  (10,-2) .. (11,2) \foreach \p in {0,1,...,5} {node[pos={\CS Pos\p\EC}] (Node\p) {}};
\node [Diplome,fill=green!40] (BacS) at (Node0)  {Bac S};
\node [Diplome,fill=green!40] (BacSTL) at  (Node1) {Bac STL};
\node [Diplome,fill=green!40] (BacSTI) at (Node2) {Bac STI};
\node [Diplome,fill=green!40] (BacES) at (Node3) {Bac ES};
\node [Diplome,fill=green!40] (BacSMS) at (Node4) {Bac SMS};
\node [Diplome,fill=green!40] (BacPRO) at (Node5) {Bac Pro};
\node (Cross) at (7,2.5) {};
\node[Classe,fill=blue!30,minimum width=4cm] (HPE1) at (7,2.5) {S.T.S. HPE 1\`ere ann\'ee};
\node[Classe,fill=blue!20,minimum width=4cm] (HPE2) at (7,3.8) {S.T.S. HPE 2\`eme ann\'ee};
\node [Diplome,fill=green!20,inner sep=4pt] (BTSHPE) at (7,5.5) {B.T.S. HPE};
\path (4,7) .. controls (5,8) and  (9,8) .. (10,7) \foreach \p in {0,1,2} {node[pos={\CS Posout\p\EC}] (Nod\p) {}};
\node [Travail,fill=red!60] (VieActive) at (Nod0) {Vie Active};
\node [rectangle,draw,fill=red!20] (LicensePro) at (Nod1) {Licence pro};
\node [rectangle,draw,fill=red!20, text width=2.55cm, text badly centered]  (TitreCertifie) at (Nod2) {Titre~certifi\'e de~niveau~$2$};
\pgfonlayer{background}%
\draw[Cursus,shorten >=1mm] (HPE1) -- (HPE2)-- (BTSHPE);
\draw[very thick] (Node0) .. controls +(1,-0.5) and (7,0) .. (HPE1);
\draw[very thick] (Node1) .. controls +(1.5,-0.5) and (7,-0.2) .. (HPE1);
\draw[very thick] (Node2) .. controls +(0.4,0.25) and (7,-0.5) .. (HPE1);
\draw[very thick] (Node3) .. controls +(-0.4,0.25) and (7,-0.5) .. (HPE1);
\draw[very thick] (Node4) .. controls +(-1.5,-0.5) and (7,0) .. (HPE1);
\draw[very thick] (Node5) .. controls +(-1,-0.5) and (7,0) .. (HPE1);
\draw[very thick,Cursus,shorten >=1mm] (BTSHPE.150) -- (VieActive. -8);
\draw[very thick,Cursus,shorten >=1mm] (BTSHPE) -- (LicensePro);
\draw[very thick,Cursus,shorten >=1mm] (BTSHPE.30) -- (TitreCertifie.south west);
\endpgfonlayer
\endtikzpicture
\LD@End@Svg@Test

%%%%%%%%%%%% Test SVG 002%%%%%%%%%%%%%%

\LD@Svg@Test
\pgfdeclarelayer{background}
\pgfsetlayers{background,main}
%
\tikzpicture[]
\tikzstyle{Classe}=[rectangle,rounded corners,minimum width=3.5cm,draw]
\tikzstyle{Option}=[ellipse,draw,inner sep=2pt]
\tikzstyle{Diplome}=[starburst,inner sep=4pt,draw]
\tikzstyle{Cursus}=[-triangle 60,very thick,draw]
\node[Classe,fill=blue!40,minimum width=4cm] (2nde) at (2,-2.4) {Seconde G\'en\'erale};
\node[Option,fill=red!20] (SI) at (1,-1.7) {\sevenpts SI};
\node[Option,fill=red!20] (SVT) at (2,-1.7) {\sevenpts SVT} ;
\node[Option,fill=red!20] (PC) at (3,-1.7) {\sevenpts PC} ;
\node [Diplome,fill=green!20] (BacS) at (2,1.4) {Bac S};
\draw[Cursus,shorten >=1mm] (2nde.163) .. controls ++(0,0) .. (SI)  .. controls (0.5,0) .. (BacS.210);
\draw[Cursus,shorten >=1mm] (2nde.17) .. controls ++(0,0) .. (PC)  .. controls (3.5,0) .. (BacS.330);
\draw[Cursus,shorten >=0.3cm] (2nde) -- (SVT) -- (BacS);
\node[Classe,fill=blue!30] (1ereS) at (2,-1) {Premi\`ere S};
\node[Classe,fill=blue!20] (TS) at (2,0) {Terminale S};\pgfonlayer{background}
\node[rectangle, rounded corners,fill=red!40,color=green!60,pattern color=green!40,draw,pattern=north west lines, fit=(SI) (SVT) (PC),inner sep=3pt] (options) {};
\endpgfonlayer
\endtikzpicture
\LD@End@Svg@Test

%%%%%%%%%%%% Test SVG 003%%%%%%%%%%%%%%

\LD@Svg@Test
	\tikzstyle{every node}=[rounded corners=0pt,anchor=center,text centered]%
	\tikzpicture[rounded corners=0pt,anchor=center,text centered]
		\scope
			\node [text width=2cm, text height=10pt,rectangle, draw,node distance=18pt,inner xsep=0pt,inner ysep=4pt] (aa) {$\sc P$} ;
			\node [text width=2cm, text height=10pt,rectangle, draw,right of=aa,node distance=2cm,inner xsep=0pt,inner ysep=4pt] (bb) {Vrai} ;
			\node [text width=2cm, text height=10pt,rectangle, draw,right of=bb, node distance=2cm,inner xsep=0pt,inner ysep=4pt] (cc) {Faux} ;
			\node [text width=2cm, text height=10pt,rectangle, draw,below of=aa, node distance=18pt,inner xsep=0pt,inner ysep=4pt] (aaa) {$\overline{\sc P}$} ;
			\node [text width=2cm, text height=10pt,rectangle, draw,right of=aaa,node distance=2cm,inner xsep=0pt,inner ysep=4pt] (bbb) {Faux} ;
			\node [text width=2cm, text height=10pt,rectangle, draw,right of=bbb, node distance=2cm,inner xsep=0pt,inner ysep=4pt] (ccc) {Vrai} ;
		\endscope
	\endtikzpicture
\LD@End@Svg@Test
%%%%%%%%%%%% Test SVG 004%%%%%%%%%%%%%%

\LD@Svg@Test
\tikzpicture
\clip (-1,-1) rectangle (8,4.5);
\draw[-] (-0.2,0) -- (6.5,0) coordinate (x axis);
\draw[-] (0,-0.2) -- (0,4.5) coordinate (y axis);
\draw[-,thick,dotted] (2,2.5) -- (6,4) ;
\draw[-,thick,dotted] (4,1.5) -- (6,4);
\draw[-open triangle 60,thick] (0,0) -- node [pos=0.4,anchor=east] {$\vec j$} (0,1);
\draw[-open triangle 60,thick] (0,0) -- node [pos=0.4,anchor=north] {$\vec i$} (1,0);
\draw[-triangle 60,thick] (0,0) --  (2,2.5) node [anchor=south east] {$M (s)$};
\draw[dashed] (2,2.5) -- (0,2.5) node [color=red,anchor=east] {$b$};
\draw[dashed] (2,2.5) -- (2,0) node [color=red,anchor=north] {$a$};
\draw[-triangle 60,thick] (0,0) --  (4,1.5) node [anchor=north west] {$P (z)$};
\draw[dashed] (4,1.5) -- (0,1.5) node [color=red,anchor=east] {$d$};
\draw[dashed] (4,1.5) -- (4,0) node [color=red,anchor=north] {$c$};
\draw[-triangle 60,thick] (0,0) --  (6,4) node [anchor=south] {$Q (s+z)$};
\draw[dashed] (6,4) -- (0,4) node [color=red,anchor=east] {$b+d$};
\draw[dashed] (6,4) -- (6,0) node [color=red,anchor=north] {$a+c$};
\node [anchor=north east] (O) at (0,0) {$O$};
\endtikzpicture
\LD@End@Svg@Test


%%%%%%%%%%%% Test SVG 005%%%%%%%%%%%%%%

\LD@Svg@Test
\tikzpicture
\clip (-1,-1.9) rectangle (5.5,1.9);
\draw[-] (-0.5,0) -- (5,0) coordinate (x axis);
\draw[-] (0,-1.7) -- (0,1.7) coordinate (y axis);
\node [anchor=north east] (O) at (0,0) {$O$};
\draw[-triangle 45,thick] (0,0) -- node [anchor=south,sloped,pos=0.4] {\vbox to 14pt{}$x\vec i+y\vec j$} %node [anchor=north,sloped,color=red,pos=0.6] {$z=x+iy$}
 (4.3,1.4) node [anchor=south west] {$M (z)$};
\draw[dashed] (4.3,1.4) -- (0,1.4) node [anchor=east] {$iy$};
\draw[-triangle 45,thick] (0,0) -- node [anchor=north,sloped,pos=0.4] {\vbox to 14pt{}$x\vec i-y\vec j$} % node [anchor=north,sloped,color=red,pos=0.4] {$\overline z=x-yi$} 
 (4.3,-1.4) node [anchor=north west] {$P (\overline z)$};
\draw[dashed,color=red] (4.3,-1.4) -- (0,-1.4) node [anchor=east] {$-iy$};
\draw[-triangle 45,thick] (0,0) -- node [anchor=south,pos=0.6] {$x\vec i$} node [color=red,anchor=north,pos=0.6] {$x$} (4.3,0) ;
\draw[-triangle 45,thick] (4.3,0) -- node [anchor=west] {$y\vec j$} (4.3,1.4);
\draw[|-triangle 45,thick] (4.3,0) -- node [anchor=west] {$-y\vec j$} (4.3,-1.4);
\draw[-open triangle 60,thick] (0,0) -- node [pos=0.4,anchor=east] {$\vec j$} (0,1);
\draw[-open triangle 60,thick] (0,0) -- node [anchor=north,pos=0.2] {$\vec i$} (1,0);
\endtikzpicture
\LD@End@Svg@Test


%%%%%%%%%%%% Test SVG 006 %%%%%%%%%%%%%%

\LD@Svg@Test
\tikzpicture
\clip (-1,-0.7) rectangle (6.5,2.7);
\draw[-] (-0.5,0) -- (6,0) coordinate (x axis);
\draw[-] (0,-0.5) -- (0,2.7) coordinate (y axis);
\node [anchor=north east] (O) at (0,0) {$O$};
\draw[-triangle 45,thick] (0,0) -- node [anchor=south,sloped,pos=0.4] {$OM=|z|=\sqrt{x^2+y^2}$} 
 (5.3,2.4) node [anchor=west] {$M (z)$};
\draw[-triangle 45,thick] (0,0) -- node [anchor=south] {$x\vec i$} (5.3,0) ;
\draw[-triangle 45,thick] (5.3,0) -- node [anchor=east] {$y\vec j$} (5.3,2.4);
\draw[-open triangle 60,thick] (0,0) -- node [pos=0.4,anchor=east] {$\vec j$} (0,1);
\draw[-open triangle 60,thick] (0,0) -- node [anchor=north] {$\vec i$} (1,0);
\endtikzpicture
\LD@End@Svg@Test


%%%%%%%%%%%% Test SVG 007 %%%%%%%%%%%%%% 

\LD@Svg@Test
	\tikzstyle{object}=[ellipse,fill=red!20, draw,inner sep=0.5em]
\tikzstyle{operator}=[rectangle,rounded corners,fill=blue!20, draw,inner sep=0.4em]
\tikzstyle{fork}=[diamond,fill=green!20, draw]
\tikzstyle{line}=[draw,line width=0.5ex]
\tikzstyle{thinline}=[draw,line width=0.2ex]
\font\SvgText=cmr8\relax
\tikzstyle{every node}=[inner sep=2pt]
\tikzpicture
\pgfdeclarelayer{background}
\pgfsetlayers{background,main}
\node [object] (a) {Matrice $A$} ;
\node [operator,right of=a, node distance=2.7cm] (b) {$\det(A-\lambda I_n)$} ;
\draw [line] (a)--(b) ; 
\node[object, right of=b,node distance=3cm] (c) {Polynome $P_A$} ;
\draw[line,->] (b) -- (c) ;
\node[fork,node distance=2cm,below of=c] (m) {$\forall i:m_i=n_i$} ;
\node[below of=m,node distance=2.2cm] (prep) {};
\path (m)-- node[midway] (mo) {oui} (prep) ;
\draw [line] (m)--(mo) ; \draw [line,->] (mo)--(prep);
\node[below of=prep,node distance=0.2cm] (p) {$=$} ;
\node[right of=p] (q) {$P$} ;
\node[above of=q,node distance=0.3cm] (abq) {};
\node[right of=q] (r) {$D$} ;
\node[above of=r,node distance=0.3cm] (abr) {};
\node[operator,node distance=1cm,font=\texttt,below of=r] (t) {Inversion} ;
\node[right of=r] (s) {$P^{-1}$} ;
\node[left of=p,label=180:{\qquad}] (pp) {$A$} ;
\node[node distance=0.3cm,left of=pp] (ppp) {} ;
\draw[->,thinline,draw=red] (a) .. controls +(down:4cm) and +(left:2cm) .. (ppp.west) ;
\node[node distance=0.03cm,above of=q] (qq) {} ;
\node[node distance=0.02cm,above of=s] (ss) {} ;
\draw[<-,thinline,red] (ss.south east) .. controls +(315:1cm) and +(right:1cm) .. (t);
\draw[->,thinline,red] (qq.south west) .. controls +(225:1cm) and +(left:1cm) .. (t);
\node[node distance=4cm,snake=saw,thinline,left of=m, text width=2cm] (n) {Trigonaliser $A$} ;
\path (m)-- node[midway] (mn) {non} (n) ;
\draw [line] (m)--(mn) ; \draw [line,->] (mn)--(n);
\node[operator,node distance=2.8cm,right of=c] (d) {Factorisation} ;
\draw [line](c)--(d) ;
\node[object,node distance=3cm,right of=d] (f) {multiplicit\'e $m_i$ } ;
\draw[->,line] (d) -- (f) ;
\draw[->,thinline,draw=red] (f.south) -- (abr.north east) ;
\node[object,node distance=1cm,below of=f] (g) {Valeur propre $\lambda_i$} ;
\draw[->,line] (d) -- (g) ;
\draw[->,thinline, draw=red] (g.south) -- (abr.north east) ;
\node[object,node distance=2.8cm,right of=m] (k) {Dimension $n_i$} ;
\node[operator,node distance=3cm,right of=k,text width=2.4cm,text height=1.8em] (h) {R\'esolution de\pn$(A-\lambda_iI_n)X=0$} ;
\node[object,node distance=1.2cm,below of=h] (j) {Espace propre $E_i$} ;
\draw[->,line] (h) -- (j) ;
\draw [line] (g) -- (h) ;
\draw[->,thinline] (h) -- (k) ;
\draw [line] (k) -- (m) ;
\node[object,node distance=3.5cm,left of=j] (l) {Base $B_i$} ;
\draw[->,thinline] (h) -- (l) ;
\draw[->,line] (j) -- (l) ;
\draw[->,line] (l) -- (k) ;
\draw[->,thinline,draw=red] (l.south west) -- (abq.north east) ;
\pgfonlayer{background}
	\draw [snake=saw,line] (pp.south west) rectangle (s.north east);
	\node [forbidden sign,line width=0.5ex, draw=red,fill=white] at (mn) {\qquad\quad} ;
\endpgfonlayer
\endtikzpicture
\LD@End@Svg@Test

%%%%%%%%%%%% Test SVG 008 %%%%%%%%%%%%%%

\LD@Svg@Test
\tikzpicture
\node [circle,fill=gray,pattern=north west lines,pattern color=gray,draw=red,thick] at (1,4.5) {$\sc D(a,r)$};
\node [circle,fill=gray,pattern=north west lines,pattern color=gray,draw=gray,thick] at (0,3) {$\overline{\sc D}(a,r)$};
\node [circle,draw=gray,thick] at (1,1.5) {$\sc C(a,r)$};
\draw (4,3) circle (2cm);
\node [label={[label distance=-0.3cm]-45:$a$}] at (4,3) {$\times$};
\draw [->] (4,3) -- node [anchor=south,sloped] {rayon $r$}++(200:2cm);
\draw [->] (4,3) -- node [anchor=south,sloped,pos=0.4] {$|z-a|$}++(50:2.5cm) node [label={[label distance=-0.3cm]-45:$z$}] {$\times$};
\endtikzpicture
\LD@End@Svg@Test

%%%%%%%%%%%% Test SVG 009 %%%%%%%%%%%%%%

\LD@Svg@Test
\tikzpicture[]
\tikzstyle{Classe}=[rectangle,rounded corners,draw]
\tikzstyle{Option}=[ellipse,draw,inner sep=2pt]
\tikzstyle{Diplome}=[starburst,inner sep=4pt,draw]
\tikzstyle{Cursus}=[-triangle 60,very thick,draw]
\node[Classe,fill=red!50] (2nde) at (0,0) {
Calculer la suite $(u_n)_{n\in\ob N}$ 
d\'efinie par $u_0=0$, $u_1=1$ et $
u_{n+2}=u_{n+1}+u_n+1\qquad(n\in\ob N).
$};
\endtikzpicture
\LD@End@Svg@Test


%%%%%%%%%%%% Test SVG 010 %%%%%%%%%%%%%%
\LD@Svg@Test
\tikzpicture[]
\tikzstyle{Classe}=[rectangle,rounded corners,draw]
\tikzstyle{Option}=[ellipse,draw,inner sep=2pt]
\tikzstyle{Diplome}=[starburst,inner sep=4pt,draw]
\tikzstyle{Cursus}=[-triangle 60,very thick,draw]
\node[rectangle,fill=blue!40,text width=9cm] (de) at (0,0) {%
Soit $\{I_n\}_{n\in\ob N}$ la suite d\'efinie par 
$$
I_n:=\int_0^1\ln(1+t^n)\d t
$$
pour $n\in\ob N$. \pn
a) D\'emontrer que $\lim_{n\to\infty}I_n=0$.  \pn
b) Lorsque $n\to+\infty$, d\'emontrer que $I_n\sim K/n$ 
pour une certaine constante $K>0$. 
};
\node[ellipse,fill=blue!40] (de) at (-1,0) { $K>0$. 
};
\endtikzpicture
\LD@End@Svg@Test

%%%%%%%%%%%% Test SVG 011 %%%%%%%%%%%%%%
\LD@Svg@Test
\tikzpicture[domain=0:360,smooth,variable=\x,scale=1,baseline=(current bounding box.north)]
	\draw[very thin,color=black!20,step=0.5] (-1.1,-1.1) grid (1.1,1.1);
	\draw[->] (-1.1,0) -- (1.1,0) node[below] {\eightpts$x$};
	\draw[->] (0,-1.1) -- (0,1.1) node[left] {\eightpts$y$};
	\draw[color=blue] plot ({cos(\x)^3},{sin(\x)^3}) ;
	\node [anchor=north] at (current bounding box.south) {\eightpts Figure Astroide.};
\endtikzpicture
\LD@End@Svg@Test

%%%%%%%%%%%% Test SVG 012 %%%%%%%%%%%%%%

\LD@Svg@Test
\tikzpicture[domain=0:360,smooth,variable=\x,scale=1]%,baseline=(0,0)] %(current bounding box.north)]
	\scope
	\clip (-1.2,-1.2) rectangle (1.5,1.5);
	\draw[very thin,color=black!20,step=0.5] (-1.1,-1.1) grid (1.5,1.5);
	\draw[->] (-1.1,0) -- (1.1,0) node[above] {\eightpts$x$};
	\draw[->] (0,-1.1) -- (0,1.1) node[right] {\eightpts$y$};
	\draw[color=blue] plot ({cos(\x)^3},{sin(\x)^3}) ;
	\draw[color=red] (1.41421,1.41421) circle (1.5 cm);
	\fill [color=red] (1.41421,1.41421) circle (1pt);
	\node [anchor=north,color=red] at (1.41421,1.41421) {\eightpts$C$};
	\draw [color=green](0.3535,0.3535) -- (1.41421,1.41421) node[midway,anchor=south,color=green]{\eightpts$R$};
	\draw [color=black,->](0.3535,0.3535)-- (-0.3535,1.0605)node[pos=1,anchor=north,color=black]{\eightpts$\vec T$};
	\draw [color=black,->](0.3535,0.3535)-- (-0.3535,-0.3535)node[pos=1,anchor=north,color=black]{\eightpts$\vec N$};
	\fill [black] (0.3535,0.3535) circle (1pt);
	\node [anchor=south,color=blue] at (0.3535,0.3535) {\eightpts$M$};
	\endscope
	%\node [anchor=north] at (current bounding box.south) {\eightpts \Figure [{Rayon, centre et cercle de courbure},Index=Courbes!Astroide@]{Courbure}} ;
\endtikzpicture
\LD@End@Svg@Test

%%%%%%%%%%%% Test SVG 012 %%%%%%%%%%%%%%
\LD@Svg@Test
\tikzpicture
\draw (0,0) node[inner sep=0pt,draw] {tight}
(0cm,2em) node[inner sep=5pt,draw] {loose}
(0cm,4em) node[fill=red!20] {default};
\endtikzpicture
\LD@End@Svg@Test

\LD@Svg@Test
\tikzpicture
\draw (0,0) node[inner sep=0pt,draw] {tight}
(0cm,2em) node[inner sep=5pt,draw] {loose}
(0cm,4em) node[fill=red!20] {default};
\draw (3,0) node[inner sep=0pt,draw] {TIGHT}
(3cm,2em) node[inner sep=5pt,draw] {LOOSE}
(3cm,4em) node[fill=red!20] {DEFAULT};
\draw (6,0) node[inner sep=0pt,draw] {$tight$}
(6cm,2em) node[inner sep=5pt,draw] {$loose$}
(6cm,4em) node[fill=red!20] {$default$};
\draw (9,0) node[inner sep=0pt,draw] {$TIGHT$}
(9cm,2em) node[inner sep=5pt,draw] {$LOOSE$}
(9cm,4em) node[fill=red!20] {$Default$};
\endtikzpicture
\LD@End@Svg@Test

\LD@Svg@Test
\tikzstyle{every node}=[dart, shape border uses incircle,inner sep=1pt, draw]
\tikzpicture
\foreach \a/\b/\c in {A/0/0, B/45/0, C/0/45, D/45/45}
\node [shape border rotate=\b, rotate=\c] at (\b/36,-\c/36) {\a};
\endtikzpicture
\LD@End@Svg@Test


\LD@Svg@Test
\tikzpicture[every node/.style={draw,circle}]
\draw[help lines] (0,0) grid (2,5);
\scope[node distance=5mm]
\node (a) at (1,1) {a};
\node [left=of a] {1}; \node [right=of a] {2};
\node [above=of a] {3}; \node [below=of a] {4};
\node [above left=of a] {5}; \node [above right=of a] {6};
\node [below left=of a] {7}; \node [below right=of a] {8};
\endscope
\scope[node distance=5mm and 5mm]
\node (b) at (1,4) {b};
\node [left=of b] {1}; \node [right=of b] {2};
\node [above=of b] {3}; \node [below=of b] {4};
\node [above left=of b] {5}; \node [above right=of b] {6};
\node [below left=of b] {7}; \node [below right=of b] {8};
\endscope
\endtikzpicture
\LD@End@Svg@Test


\LD@Svg@Test
\tikz \draw (0,0) .. controls +(right:3.5cm) and +(right:3.5cm) .. (0,3)
\foreach \p in {0,0.125,...,1} {node[pos=\p]{\p}};
\LD@End@Svg@Test

\LD@Svg@Test
\tikzpicture[shorten >=1pt,node distance=2cm,auto]
\draw[help lines] (0,0) grid (3,2);
\node[state] (q_0) {$q_0$};
\node[state] (q_1) [above right of=q_0] {$q_1$};
\node[state] (q_2) [below right of=q_0] {$q_2$};
\node[state] (q_3) [below right of=q_1] {$q_3$};
\path[->] (q_0) edge node {0} (q_1)
edge node [swap] {1} (q_2)
(q_1) edge node {1} (q_3)
edge [loop above] node {0} ()
(q_2) edge node [swap] {0} (q_3)
edge [loop below] node {1} ();
\endtikzpicture
\LD@End@Svg@Test

\LD@Svg@Test
% Define block styles
\tikzstyle{decision} = [diamond, draw, fill=blue!20, 
    text width=4.5em, text badly centered, node distance=3cm, inner sep=0pt]
\tikzstyle{block} = [rectangle, draw, fill=blue!20, 
    text width=5em, text centered, rounded corners, minimum height=4em]
\tikzstyle{line} = [draw, -latex']
\tikzstyle{cloud} = [draw, ellipse,fill=red!20, node distance=3cm,
    minimum height=2em]
\tikzpicture[node distance = 2cm, auto]
    % Place nodes
    \node [block] (init) {initialize model};
    \node [cloud, left of=init] (expert) {expert};
    \node [cloud, right of=init] (system) {system};
    \node [block, below of=init] (identify) {identify candidate models};
    \node [block, below of=identify] (evaluate) {evaluate candidate models};
    \node [block, left of=evaluate, node distance=3cm] (update) {update model};
    \node [decision, below of=evaluate] (decide) {is best candidate better?};
    \node [block, below of=decide, node distance=3cm] (stop) {stop};
    % Draw edges
    \path [line] (init) -- (identify);
    \path [line] (identify) -- (evaluate);
    \path [line] (evaluate) -- (decide);
    \path [line] (decide) -| node [near start] {yes} (update);
    \path [line] (update) |- (identify);
    \path [line] (decide) -- node {no}(stop);
    \path [line,dashed] (expert) -- (init);
    \path [line,dashed] (system) -- (init);
    \path [line,dashed] (system) |- (evaluate);
\endtikzpicture
\LD@End@Svg@Test

\LD@Svg@Test
\tikzpicture[scale=3,cap=round]
  % Local definitions
  \def\costhirty{0.8660256}

  % Colors
  \colorlet{anglecolor}{green!50!black}
  \colorlet{sincolor}{red}
  \colorlet{tancolor}{orange!80!black}
  \colorlet{coscolor}{blue}

  % Styles
  \tikzstyle{axes}=[]
  \tikzstyle{important line}=[very thick]
  \tikzstyle{information text}=[rounded corners,fill=red!10,inner sep=1ex]

  % The graphic
  \draw[style=help lines,step=0.5cm] (-1.4,-1.4) grid (1.4,1.4);

  \draw (0,0) circle (1cm);

  \scope[style=axes]
    \draw[->] (-1.5,0) -- (1.5,0) node[right] {$x$};
    \draw[->] (0,-1.5) -- (0,1.5) node[above] {$y$};

    \foreach \x/\xtext in {-1, -.5/-0.5, 1}
      \draw[xshift=\x cm] (0pt,1pt) -- (0pt,-1pt) node[below,fill=white]
            {$\xtext$};

    \foreach \y/\ytext in {-1, -.5/-0.5, .5/0.5, 1}
      \draw[yshift=\y cm] (1pt,0pt) -- (-1pt,0pt) node[left,fill=white]
            {$\ytext$};
  \endscope

  \filldraw[fill=green!20,draw=anglecolor] (0,0) -- (3mm,0pt) arc(0:30:3mm);
  \draw (15:2mm) node[anglecolor] {$\alpha$};

  \draw[style=important line,sincolor]
    (30:1cm) -- node[left=1pt,fill=white] {$\sin \alpha$} +(0,-.5);

  \draw[style=important line,coscolor]
    (0,0) -- node[below=2pt,fill=white] {$\cos \alpha$} (\costhirty,0);

  \draw[style=important line,tancolor] (1,0) --
    node [right=1pt,fill=white]
    {
      $\displaystyle \tan \alpha \color{black}=
      {{\color{sincolor}\sin \alpha}\over\color{coscolor}\cos \alpha}$
    } (intersection of 0,0--30:1cm and 1,0--1,1) coordinate (t);

  \draw (0,0) -- (t);

  \draw[xshift=1.85cm] node [right,text width=6cm,style=information text]
    {
      The {\color{anglecolor} angle $\alpha$} is $30^\circ$ in the
      example ($\pi/6$ in radians). The {\color{sincolor}sine of
        $\alpha$}, which is the height of the red line, is
      $$
  \sin \alpha = 1/2.
      $$
      By the Theorem of Pythagoras we have ${\color{coscolor}\cos^2 \alpha} +
      {\color{sincolor}\sin^2\alpha} =1$. Thus the length of the blue
      line, which is the {\color{coscolor}cosine of $\alpha$}, must be
      $$
      {\color{coscolor}\cos\alpha} = \sqrt{1 - 1/4} = \textstyle
      0.5 \sqrt 3.
      $$%
      This shows that {\color{tancolor}$\tan \alpha$}, which is the
      height of the orange line, is
      $$
      {\color{tancolor}\tan\alpha} = {\color{sincolor}\sin
          \alpha\over\color{coscolor}\cos \alpha} = 1/\sqrt 3.
      $$
    };
\endtikzpicture
\LD@End@Svg@Test




\def\A{\sc A }
\def\B{\sc B }
\def\pa#1{\left(#1\right)}

\LD@Svg@Test
\tikzpicture[
    grow=right,
    level 1/.style={sibling distance=3.5cm,level distance=5.2cm},
    level 2/.style={sibling distance=3.5cm, level distance=6.7cm},
    edge from parent/.style={very thick,draw=blue!40!black!60,
        shorten >=5pt, shorten <=5pt},
    edge from parent path={(\tikzparentnode.east) -- (\tikzchildnode.west)},
    kant/.style={text width=2cm, text centered, sloped},
 every node/.style={text ragged, inner sep=2mm},
    punkt/.style={rectangle, rounded corners, shade, top color=white,
    bottom color=blue!50!black!20, draw=blue!40!black!60, very
    thick }
    ]

\node[punkt, text width=5.5em] {Country~\B}
    %Lower part lv1
    child {
        node[punkt] [rectangle split, rectangle split, rectangle split parts=3,
         text ragged] {
            \bf{Scenario  1}
                  \nodepart{second}
            $\mbox{Country \B}\colon    s\bar{Q}$
                  \nodepart{third}
            $\mbox{Country \A}\colon\pa{1-s}\bar{Q}$
        }
        edge from parent
            node[kant, below, pos=.6] {Unchanged parity}
    }
    %Upper part, lv1
    child {
        node[punkt, text width=6em] {Country~\A}
        %child 1
        child {
            node [punkt,rectangle split, rectangle split,
            rectangle split parts=3] {
                \bf{Scenario  2}
                \nodepart{second}
                $\mbox{Country \B}\colon s\bar{Q}+2\alpha\Delta E -sc$
                \nodepart{third}
                $\mbox{Country \A}\colon\pa{1-s}\bar{Q}-\alpha\Delta E -
                \pa{1-s}c$
            }
            edge from parent
                node[below, kant,  pos=.6] {Unchanged parity}
        }
        %child 2
        child {
            node [punkt, rectangle split, rectangle split parts=3]{
                \bf{Scenario 3}
                \nodepart{second}
                $\mbox{Country \B}\colon s\bar{Q}-2sc$
                \nodepart{third}
                $\mbox{Country \A}\colon\pa{1-s}\bar{Q}-2\pa{1-s}c$
            }
            edge from parent
                node[kant, above] {Devalues}}
            edge from parent{
                node[kant, above] {Devalues}}
    };
\endtikzpicture
\LD@End@Svg@Test


\bye



