%% LD@Layout.tex : layout of the document
%
%  Dependencies : 
%
%%%%%%%%%%%%%%%%%%%%%%%%%%%%%%%%%%%%%%%%%%%%%%%%%%%%%
%                                                                                                                                            %
%                                                  Paramètres d'une page                                                    %
%                                                                                                                                            %
%%%%%%%%%%%%%%%%%%%%%%%%%%%%%%%%%%%%%%%%%%%%%%%%%%%%%


%%% différents formats de pages
\def\formatpage{
\hsize150truemm
\vsize 240truemm
\hoffset=5truemm
\voffset=-3truemm
\pretolerance=500
\tolerance=1000
\brokenpenalty=5000
\parindent3mm
}
%\formatpage

\def\olspept{
%\magnification 1200
\hsize180truemm\vsize 270truemm\hoffset=-10truemm\voffset=-13truemm
\pretolerance=500\tolerance=1000\brokenpenalty=5000
\parindent3mm
}


\def\dimstand{\hsize 150truemm\vsize 240truemm\hoffset=5truemm\voffset=0truemm}

\def\dimart{\hsize126truemm\vsize186truemm\hoffset16truemm\voffset=24truemm}

%%



\interfootnoteskip=0pt
\let\note=\numberedfootnote
\everyfootnote={\eightpoint\leftskip=5truemm\rightskip5truemm}


%%


\newcount\chstart
\chstart=\pageno
\headline={\ifnum\pageno=\chstart {\hfill} \else {\hss \tenrm --\ \folio\ --\hss}\fi}
\footline={\hfill}
\normalbaselines
\frenchspacing

%%
\newif\ifMidFolio
\MidFoliotrue            % numéro de page au mileu du haut de page par d\'efaut

\newif\ifpagetitre \pagetitretrue
\newtoks\hautpagetitre \hautpagetitre={\hfill}
\newtoks\baspagetitre \baspagetitre={\hfill}
\newtoks\auteurcourant \auteurcourant={\hfill}
\newtoks\titrecourant \titrecourant={\hfill}
\newtoks\hautpagegauche
\newtoks\hautpagedroite
\newtoks\hautpagemilieu
\hautpagemilieu={\tenrm\hfil -- \folio\ -- \hfil}
\hautpagegauche={\ifMidFolio\the\hautpagemilieu\else\tenrm\folio\hfil\the\auteurcourant\hfil\fi}
\hautpagedroite={\ifMidFolio\the\hautpagemilieu\else\hfil\the\titrecourant\hfil\tenrm\folio\fi}
\newtoks\baspagegauche \baspagegauche={\hfil}
\newtoks\baspagedroite \baspagedroite={\hfil}
\headline={\ifpagetitre\the\hautpagetitre
\else\ifodd\pageno\the\hautpagedroite\else\the\hautpagegauche\fi\fi }
\footline={\ifpagetitre\the\baspagetitre
\else\ifodd\pageno\the\baspagedroite
\else\the\baspagegauche\fi\fi \global\pagetitrefalse}

\def\nopagenumber{\headline={\hfill}\footline={\hfill}}
%\def\numpgsuiv{ \chstart=\pageno\pageno=\chstart}

\def\pageblanche{\vfill\eject\pagetitretrue
\null\vfill\eject
\pagetitretrue
}


\def\hautspages#1#2{\auteurcourant={\ninepcap#1}\titrecourant={\nineit#2}\MidFoliofalse}

\ifnum\chstart=\pageno \pagetitretrue\fi
%\ifnum\pageno=\chstart \pagetitretrue \fi

%%%%%%%%%%%%%%%%%%%%%%%%%%%%%%%%%%%%%%%%%%%%%%%%%%%%%
%                                                                                                                                            %
%                                                 Notes en marge gauche                                                     %
%                                                                                                                                            %
%%%%%%%%%%%%%%%%%%%%%%%%%%%%%%%%%%%%%%%%%%%%%%%%%%%%%


\def\leftnote#1{\vadjust{\setbox1=\vtop{\hsize 20mm\parindent=0pt\eightpoint
\baselineskip=9pt\rightskip=4mm plus 4mm\vskip-4.7mm#1}\hbox{\kern-2cm\smash{\box1}}}}



%%%%%%%%%%%%%%%%%%%%%%%%%%%%%%%%%%%%%%%%%%%%%%%%%%%%%
%                                                                                                                                            %
%                                                 Notes en bas de page                                                       %
%                                                                                                                                            %
%%%%%%%%%%%%%%%%%%%%%%%%%%%%%%%%%%%%%%%%%%%%%%%%%%%%%


% Pour que les accents se placent correctement en mode math en corps 8 et 6
\skewchar\eighti='177 \skewchar\sixi='177
\skewchar\eightsy='60 \skewchar\sixsy='60

%
% [Format : \note{texte de la note} avec numerotation automatique]
%
%
\let\footnotea=\footnote
\catcode`@=11
\message{output routines,}
\newinsert\footins
\def\footnote#1{\let\@sf\empty % parameter #2 (the text) is read later
  \ifhmode\edef\@sf{\spacefactor\the\spacefactor}\/\fi
\hbox{$^{(#1)}$}\@sf\vfootnote{#1.}}
\def\vfootnote#1{\insert\footins\bgroup
  \interlinepenalty\interfootnotelinepenalty
  \splittopskip\ht\strutbox % top baseline for broken footnotes
  \splitmaxdepth\dp\strutbox \floatingpenalty\@MM
  \leftskip\z@skip \rightskip\z@skip \spaceskip\z@skip \xspaceskip\z@skip
  \textindent{#1}\footstrut\futurelet\next\fo@t}
\def\fo@t{\ifcat\bgroup\noexpand\next \let\next\f@@t
  \else\let\next\f@t\fi \next}
\def\f@@t{\bgroup\aftergroup\@foot\let\next}
\def\f@t#1{#1\@foot}
\def\@foot{\strut\egroup}
\def\footstrut{\vbox to\splittopskip{}}
\skip\footins=\bigskipamount % space added when footnote is present
\count\footins=800 % footnote magnification factor (1 to 1)
\dimen\footins=8in % maximum footnotes per page

%
%\anote=\footnote de plain (2 arguments)
%
\def\anote#1#2{\footnotea{\hbox{$^{#1}$}}{\eightpts#2}}
%\catcode`@=12 % the "@" signs are no longer letters

%reference de note
 \def\defrefnote#1{\definexref{#1}{{\the\footnotenumber}}{refnotes}}
 \def\refnote#1{\ref{#1}}



%\def\note#1#2{\footnote{#1}{\noindent\eightpts #2\vskip-10pt\par}}
%\def\note#1#2{\hbox{$^{(#1)}$}\footnote{}{\eightpts$#1.$#2\vskip-10pt\par}}
\def\pclm#1#2{\noi\bf#1\sl\ #2}

\def\footnoterule{\kern -2pt\hrule width 7truecm\kern 2.4pt}

%reference croisee a des numeros de notes
\def\xnotedef#1{\definexref{#1}{\noexpand\number\footnotenumber}{Note}}%
\def\noteref{\ref}


%
% Fin notes
%



%%%%%%%%%%%%%%%%%%%%%%%%%%%%%%%%%%%%%%%%%%%%%%%%%%%%%
%                                                                                                                                            %
%                                  Numérotation des équation (à gauche par défaut)                           %
%                                                                                                                                            %
%%%%%%%%%%%%%%%%%%%%%%%%%%%%%%%%%%%%%%%%%%%%%%%%%%%%%

 % choix numerotation gauche ou droite: defaut=gauche, \numadroite{oui} -> a droite
 
  
\newif\ifNumGauche
\NumGauchetrue

\ifNumGauche\let\eqno=\leqno\let\eqalignno=\leqalignno
\else\let\leqno=\eqno\let\leqalignno=\eqalignno
\fi




% Layout

\def\AFiveLayout{%
\global\hsize135truemm\global\vsize192truemm\voffset=-17truemm
\hoffset=-19truemm
\pretolerance=500\tolerance=1000\brokenpenalty=5000
\parindent0mm
\paperwidth=148.5truemm
\paperheight=210truemm
\pdfpagewidth=148.5truemm
\pdfpageheight=210truemm}%
%
\def\TangoLayoutLandscape{%
\hsize270truemm\vsize 187truemm\hoffset=-13truemm\voffset=-17truemm
\pretolerance=500\tolerance=1000\brokenpenalty=5000
\parindent0mm
\pdfpagewidth=297truemm
\pdfpageheight=210truemm
\paperwidth=297truemm
\paperheight=210truemm
}%
\def\LD@AFour@Book{%
	\hsize170truemm\relax
	\vsize 250truemm\relax
	\hoffset=-5truemm\relax
	\voffset=-13truemm\relax
	\pretolerance=500\relax
	\tolerance=1000\relax
	\brokenpenalty=5000\relax
	\parindent3mm% fix this for \item ?
	\paperwidth=210truemm\relax
	\paperheight=297truemm\relax
	\pdfpagewidth=210truemm\relax
	\pdfpageheight=297truemm\relax
}%
\endinput
