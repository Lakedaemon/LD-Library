%%%%%%%%%%%%%%%%%%%%%%%%%%%%%%%%%%%%%%%%%%
%																%
%					\LD@Core@IncludeOnce #1  						%
%																%
%				Ensure that Library #1 is included only once					%
%																%
%%%%%%%%%%%%%%%%%%%%%%%%%%%%%%%%%%%%%%%%%%
\def\LD@Core@IncludeOnce#1{%
	\expandafter\ifx\csname LD@Core@IncludeOnce@#1\endcsname\endinput
		\endinput 
	\else
		\expandafter\let\csname LD@Core@IncludeOnce@#1\endcsname=\endinput
	\fi
}%
\LD@Core@IncludeOnce{LD@Core}%
%%%%%%%%%%%%%%%%%%%%%%%%%%%%%%%%%%%%%%%%%%
%																%
%							Define alias cs	  						%
%																%
%																%
%%%%%%%%%%%%%%%%%%%%%%%%%%%%%%%%%%%%%%%%%%
\let\EA=\expandafter
\let\CS=\csname
\let\EC=\endcsname
\let\Par\par
\def\LD@Empty{}%
\def\LD@Relax{\relax}%
\def\LD@Space{ }%
%%%%%%%%%%%%%%%%%%%%%%%%%%%%%%%%%%%%%%%%%%
%																%
%						Expandafter shortcuts							%
%																%
%																%
%%%%%%%%%%%%%%%%%%%%%%%%%%%%%%%%%%%%%%%%%%
\let\LD@EaB\expandafter
\def\LD@EaBC#1#2{%
	\expandafter\expandafter\expandafter#1\expandfter#2%
}%
\def\LD@EabC#1#2{%
	\expandafter#1\expandfter#2%
}%
\def\LD@EaBCD#1#2#3{%
	\expandafter\expandafter\expandafter\expandafter\expandafter\expandafter\expandafter
	#1\expandafter\expandafter\expandafter#2\expandfter#3%
}%
\def\LD@EaBcD#1#2#3{%
	\expandafter\expandafter\expandafter#1\expandafter#2\expandfter#3%
}%
\def\LD@EabCD#1#2#3{%
	\expandafter\expandafter\expandafter#1\expandafter\expandafter\expandafter#2%
	\expandfter#3%
}%
\def\LD@EabcD#1#2#3{%
	\expandafter#1\expandafter#2\expandfter#3%
}%
%%%%%%%%%%%%%%%%%%%%%%%%%%%%%%%%%%%%%%%%%%
%																%
%							Define registers			  				%
%																%
%																%
%%%%%%%%%%%%%%%%%%%%%%%%%%%%%%%%%%%%%%%%%%
% Tokens
\newtoks\LD@ToksOne
\newtoks\LD@ToksTwo

% Boxes
\newbox\LD@Box
\newbox\LD@Box@One

% Dimensions
\newdimen\LD@Dimen
\newdimen\LD@Dim@One

% To use with char, catcode and `
\newcount\LD@CharOne
\newcount\LD@CharTwo

% Counters
\newcount\LD@Counter
\newcount\LD@Counter@Temp@A
\newcount\LD@Counter@Temp@B



% Temporary Counter : These are global counters because they could be used at different level of grouping. And we don't want to saturate the save stack with it. 
% Warning : don't ever call an undead or a living \LD@Macro between the moment you set it and the moment you use it (it could change and produce weird effect if the \LD@Macro uses the same TempCounter)
\newcount\LD@TempCounter@A
\global\LD@TempCounter@A=0\relax

% Temporary Macros : These are global counters because they could be used at different level of grouping. And we don't want to saturate the save stack with it. 
% Warning : don't ever call an undead or living \LD@Macro between the moment you set it and the moment you use it (it could change and produce weird effect if the \LD@Macro uses the same TempCounter)
%\gdef\LD@Temp@A{}%
%\gdef\LD@Temp@Trash%

% user count 
\newcount\LD@Zero
\newcount\LD@One
\newcount\LD@Two
\newcount\LD@Three
\newcount\LD@Four
\newcount\LD@Five

% Maths Library

%Time Macro
\newcount\LD@Time@Counter@Moment
\newcount\LD@Time@Counter@Lapse


% XML Macro
\newwrite\LD@XML@Output@One
\newwrite\LD@XML@Output@Two
\newwrite\LD@XML@Output@Three

% Yomigata Macro

% To Parse File 
\newread\LD@ParseFile@InputFile

%%%%%%%%%%%%%%%%%%%%%%%%%%%%%%  	Obsolete

% Local Api number Count : don't use globally
\newcount\LD@Counter@A
\newcount\LD@Counter@B
\newcount\LD@Counter@C
\newcount\LD@Counter@D
\newcount\LD@Counter@E
% global Api number Count
\newcount\LD@Counter@GA
\newcount\LD@Counter@GB
\newcount\LD@Counter@GC
\newcount\LD@Counter@GD
\newcount\LD@Counter@GE

%Time Macro
\newcount\LD@Time@Moment
\newcount\LD@Time@Lapse

% XML Macro
\newwrite\LD@XML@OutputOne
\newwrite\LD@XML@OutputTwo
\newwrite\LD@XML@OutputThree
%%%%%%%%%%%%%%%%%%%%%%%%%%%%%%%%%%%%%%%%%%
%																%
%					\LD@String@IfSubString #1#2#3#4					%
%																%
%			executes #3 if #2 is a substring of #1, executes #4 otherwise			%
%																%
%%%%%%%%%%%%%%%%%%%%%%%%%%%%%%%%%%%%%%%%%%
% to do : sanitize the code
\def\LD@String@IfSubString#1#2#3#4{%
	\edef\LD@SubString@Token{#2}%
	\EA\EA\EA\def\EA\EA\EA\LD@SubString@Test\EA\EA\EA{\EA#1\LD@SubString@Token}%
	\EA\LD@String@Split\EA\LD@SubString@Test\EA{\LD@SubString@Token}\LD@SubString@Trash\LD@SubString@Test
	\ifx\LD@SubString@Test\LD@Empty
		#4%
	\else
		#3%
	\fi
}%
%\def\LD@String@IfSubString#1#2#3#4{%
%	\EA\def\EA\LD@IfSubString@Process\EA##1#2##2\LD@Empty{%
%		\def\LD@String@Test{##2}%
%		\ifx\LD@String@Test\LD@Empty
%			#3%
%		\else
%			#4%
%		\fi
%	}%
%	\EA\EA\EA\LD@IfSubString@Process \EA#1#2\LD@Empty
%	\EA\EA\EA\def\EA\EA\EA\LD@String@Test\EA\EA\EA{\EA#1#2}%
%	\EA\LD@String@Split\EA\LD@String@Test\EA{#2}\LD@String@Trash\LD@String@Test
%	\ifx\LD@String@Test\LD@Empty
%		#3%
%	\else
%		#4%
%	\fi
%}%
%%%%%%%%%%%%%%%%%%%%%%%%%%%%%%%%%%%%%%%%%%%%%%%%%%%%%%%%%%%%%%
%														%
%						Options support						%
%														%
%%%%%%%%%%%%%%%%%%%%%%%%%%%%%%%%%%%%%%%%%%%%%%%%%%%%%%%%%%%%%%
\def\LD@Option@Process#1{%
	\unless\ifx\@optionalarg\LD@Empty
		\def\LD@Option@@{}%
		\LD@Loop@For\LD@Option@Member=\@optionalarg\WithSeparator #1\Do{%
			\EA\def\EA\LD@Temp\EA{\LD@Option@Member =}%
			\LD@String@Split\LD@Temp =\LD@Option@Key\LD@Option@Value
			\trim\LD@Option@Key
			\ifx\LD@Option@Value\LD@Empty
				\EA\def\EA\LD@Option@@\EA{\LD@Option@Key}%
			\else
				\LD@String@Split\LD@Option@Value =\LD@Option@Value\LD@Trash %%%%% strange construction \LD@Temp needed ? global against local def ?
				\EA\def\EA\LD@Temp\EA{\LD@Option@Value}%
				\EA\let\CS LD@Option@@\LD@Option@Key\EC\LD@Temp
			\fi
		}%
	\fi
}%
%%%%%%%%%%%%%%%%%%%%%%%%%%%%%%%%%%%%%%%%%%%%%%%%%%%%%
%                                                                                                                                            %
%                                                 Maths : Euclidian division (Undead)                                  %
%                                                                                                                                            %
%%%%%%%%%%%%%%%%%%%%%%%%%%%%%%%%%%%%%%%%%%%%%%%%%%%%%
\def\LD@Maths@EuclidianDiv#1#2#3#4{%
	\LD@Counter@Temp@B=#1\relax
	\divide\LD@Counter@Temp@B by#2\relax
	\LD@Counter@Temp@A=\LD@Counter@Temp@B\relax
	\multiply\LD@Counter@Temp@B by#2\relax
	\advance\LD@Counter@Temp@B by-#1\relax
	\LD@Counter@Temp@B=-\LD@Counter@Temp@B\relax
	#3=\LD@Counter@Temp@A\relax
	\edef#4{\the\LD@Counter@Temp@B}%	
}%
%%%%%%%%%%%%%%%%%%%%%%%%%%%%%%%%%%%%%%%%%%%%%%%%%%%%%%%%%%%%%%
%														%
%						Time  Library							%
%														%
%%%%%%%%%%%%%%%%%%%%%%%%%%%%%%%%%%%%%%%%%%%%%%%%%%%%%%%%%%%%%%
\def\LD@Time@Date{\the\year\backslash\the\month\backslash\the\day}%
\def\LD@Time@ToUndredth#1#2:#3:#4,#5.{% #1 must be a counter
	\global#1=#2\relax
	\global\multiply#1 by 60\relax
	\global\advance#1 by #3\relax
	\global\multiply#1 by 60\relax
	\global\advance#1 by #4\relax
	\global\multiply#1 by 100\relax
	\global\advance#1 by #5\relax
}%
\def\LD@Time@ToTime#1#2{% #1 must be a counter
	\LD@Maths@EuclidianDiv#1{100}#1\LD@Temp@A
	\LD@Maths@EuclidianDiv#1{60}#1\LD@Temp@B
	\edef\LD@Temp@B{\LD@Temp@B s\LD@Temp@A}%
	\LD@Maths@EuclidianDiv#1{60}#1\LD@Temp@A
	\edef#2{\the#1h\LD@Temp@A m\LD@Temp@B}%
}%
\def\LD@Time@Display#1#2{%
	\LD@Time@ToTime#1\LD@Time@Temp@A
	\immediate\write16{Done in \LD@Time@Temp@A}%
	\immediate\write16{#2}%
}%
\def\LD@Time@Write{%
	\immediate\write18{echo \percentchar time\percentchar.>C:/TeX/Output/LD@Time@Stamp.tex}%
}%
\def\LD@Time@Start#1{%
	\immediate\write16{now XeTeXing...}%
	\LD@Time@Write
	\EA\LD@Time@ToUndredth\EA\LD@Time@Moment\input LD@Time@Stamp.tex\relax
}%
\def\LD@Time@Benchmark{\@getoptionalarg\@LD@Time@Benchmark}%
\def\@LD@Time@Benchmark#1{%
	\LD@Time@Lapse=\LD@Time@Moment\relax
	\LD@Time@Write
	\EA\LD@Time@ToUndredth\EA\LD@Time@Moment\input LD@Time@Stamp.tex\relax
	\advance\LD@Time@Lapse by-\LD@Time@Moment\relax
	\LD@Time@Lapse=-\LD@Time@Lapse\relax
	\advance\LD@Time@Lapse by-5\relax % to counter the IO lag.
	\ifx\@optionalarg\LD@Empty
		\LD@Time@Display\LD@Time@Lapse{#1}%
	\else
		\LD@Time@ToTime\LD@Time@Lapse#1%
	\fi
}%
%%%%%%%%%%%%%%%%%%%%%%%%%%%%%%%%%%%%%%%%%%%%%%%%%%%%%%%%%%%%%%
%														%
%						Stats								%
%														%
%%%%%%%%%%%%%%%%%%%%%%%%%%%%%%%%%%%%%%%%%%%%%%%%%%%%%%%%%%%%%%
\def\LD@Stat@Update#1{%
\begingroup
	\ifcsname LD@Stat@#1\endcsname
		\LD@Counter@A=\CS LD@Stat@#1\EC\relax
		\advance\LD@Counter@A by1\relax
		\EA\xdef\CS LD@Stat@#1\EC{\the\LD@Counter@A}% increment
	\else
		\EA\xdef\CS LD@Stat@#1\EC{1}% initialisation and increment
	\fi
\endgroup
}%
\def\LD@Stat@Display#1{%
	\ifcsname LD@Stat@#1\endcsname
		\LD@Counter@A=\CS LD@Stat@#1\EC\relax
		\ifnum\LD@Counter@A<10\relax
			0\the\LD@Counter@A
		\else
			\the\LD@Counter@A
		\fi	
	\else
		\LD@Stat@Zero% =0
	\fi
}% Fails Beautifully
\def\LD@Stat@Zero{}% defines what to display/execute when the count is =0
%%%%%%%%%%%%%%%%%%%%%%%%%%%%%%%%%%%%%%%%%%%%%%%%%%%%%%%%%%%%%%
%														%
%						Data								%
%														%
%%%%%%%%%%%%%%%%%%%%%%%%%%%%%%%%%%%%%%%%%%%%%%%%%%%%%%%%%%%%%%
\def\LD@Data@Set#1#2#3{%
	\def\LD@Data@Extract##1#1=##2##3\LD@Data@ExtractEnd{%
		\xdef#3{##2}% is x necessary ? Fix this !
	}%
	\ifcsname Data#2\endcsname
		\EA\EA\EA\LD@Data@Extract\CS Data#2\EC\LD@Data@ExtractEnd
	\else
		\let#3\relax
	\fi
}% Fails Beautifully
\def\LD@Data@Def#1#2#3{%
	\def\LD@Data@Extract##1#1=##2##3\LD@Data@ExtractEnd{%
		\def#3{##2}%
	}%
	\ifcsname Data#2\endcsname
		\EA\EA\EA\LD@Data@Extract\CS Data#2\EC\LD@Data@ExtractEnd
	\else
		\let#3\relax
	\fi
}% Fails Beautifully
\endinput