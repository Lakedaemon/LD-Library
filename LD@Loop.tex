\LD@Core@IncludeOnce{LD@Loop}%
%
%
% Caution ! Some commands will have weird behavior and/or produce errors if you feed them with explicit argument instead of control sequences
% (this is due to an expandafter). For these commands : never code \Command{arg} but code \def\temp{arg}\Command\temp
% 
% Such commands are : \LD@String@Split, \LD@String@Length, \LD@String@LastChar
%
%
%
%%%%%%%%%%%%%%%%%%%%%%%%%%%%%%%%%%%%%%%%%%%%%%%%%%%%%
%												%
%					Dead : Loop	 					%
%												%
%%%%%%%%%%%%%%%%%%%%%%%%%%%%%%%%%%%%%%%%%%%%%%%%%%%%%
% #2 must NOT be {} but it can be an empty \cs 
\def\LD@Loop@For#1=#2\WithSeparator#3\Do#4{%
	\EA\def\EA\LD@Loop@For@Tail\EA{#2}%
	\unless\ifx\LD@Loop@For@Tail\LD@Empty
		\EA\def\EA\LD@Loop@For@Tail\EA{\LD@Loop@For@Tail #3}%
		\def\LD@Loop@Gobble##1#3##2\LD@Loop@Empty{%
			\def#1{##1}%
			\def\LD@Loop@For@Tail{##2}%
		}%
		\def\LD@Loop@For@Iterate{%
			\ifx\LD@Loop@For@Tail\LD@Empty
				\let\LD@Loop@For@Next\relax
			\else	
				\EA\LD@Loop@Gobble\LD@Loop@For@Tail\LD@Loop@Empty
				#4\relax
				\let\LD@Loop@For@Next\LD@Loop@For@Iterate
			\fi
			\LD@Loop@For@Next
		}%
		\LD@Loop@For@Iterate
	\fi
}%



















% Loops

% <ded> loops as long as there is a token in the #1 edef expanded string
\def\ForEach#1:=#2\do#3{%
	\edef\LD@ForEach@Tail{#2}%
	\def\LD@Gobble##1##2\LD@Empty{%
		\def#1{##1}%
		\def\LD@ForEach@Tail{##2}%
	}%
	\def\LD@ForEach@Iterate{%
		\ifx\LD@ForEach@Tail\LD@Empty
			\let\LD@ForEach@Next=\relax
		\else	
			\EA\LD@Gobble\LD@ForEach@Tail\LD@Empty
			#3\relax
			\let\LD@ForEach@Next=\LD@ForEach@Iterate
		\fi
		\LD@ForEach@Next
	}%
	\LD@ForEach@Iterate
}%
\def\LD@For#1:=#2\With#3\Do#4{%
	\edef\LD@For@Tail{#2}%
	\unless\ifx\LD@For@Tail\LD@Empty
		\edef\LD@For@Tail{\LD@For@Tail #3}%
		\def\LD@Gobble##1#3##2\LD@Empty{%
			\def#1{##1}%
			\def\LD@For@Tail{##2}%
		}%
		\def\LD@For@Iterate{%
			\ifx\LD@For@Tail\LD@Empty
				\let\LD@For@Next=\relax
			\else	
				\EA\LD@Gobble\LD@For@Tail\LD@Empty
				#4\relax
				\let\LD@For@Next=\LD@For@Iterate
			\fi
			\LD@For@Next
		}%
		\LD@For@Iterate
	\fi
}%
\def\LD@ForOne#1:=#2\With#3\Do#4{%
	\xdef\LD@ForOne@Tail{#2}%
	\unless\ifx\LD@ForOne@Tail\LD@Empty
			\xdef\LD@ForOne@Tail{\LD@ForOne@Tail #3}%
		\gdef\LD@GobbleOne##1#3##2\LD@Empty{%
			\def#1{##1}%
			\xdef\LD@ForOne@Tail{##2}%
		}%
		\gdef\LD@ForOne@Iterate{%
			\ifx\LD@ForOne@Tail\LD@Empty
				\global\let\LD@ForOne@Next=\relax
			\else	
				\EA\LD@GobbleOne\LD@ForOne@Tail\LD@Empty
				#4\relax
				\global\let\LD@ForOne@Next=\LD@ForOne@Iterate
			\fi
			\LD@ForOne@Next
		}%
		\LD@ForOne@Iterate
	\fi
}%

\def\LD@Ignore#1\LD@Relax{}%



% usage :
%
% \LD@ParseFile{Letter:/Path/File.ext}{#1...#9 as in the parameter definition of a macro}{code to execute for each line}{catcode changes : \catcode`\\=11,...}%
%

\def\LD@ParseFile#1#2#3#4{%
	\begingroup
		\endlinechar=-1\relax% what does this do ? 
		\CS def\EC\LD@ParseFile@Code #2\LD@Empty{{#3}}% why is the group necessary here ?
		\openin\LD@ParseFile@InputFile=#1\relax
		\def\LD@ParseFile@Iterate{%
		\ifeof\LD@ParseFile@InputFile
			\let\LD@ParseFile@Next=\relax
		\else
			\read\LD@ParseFile@InputFile to\LD@ParseFile@InputLine
			\unless\ifx\LD@ParseFile@InputLine\LD@Empty
	 			\EA\LD@ParseFile@Code\LD@ParseFile@InputLine\LD@Empty
			\fi
			\let\LD@ParseFile@Next\LD@ParseFile@Iterate
		\fi
		\LD@ParseFile@Next
		}%
		{#4\LD@ParseFile@Iterate}%
		\closein\LD@ParseFile@InputFile
	\endgroup
}%
\endinput